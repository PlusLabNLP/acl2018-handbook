\section[Restaurants and Attractions]{Restaurants and Attractions}
\thispagestyle{emptyheader}
\setheaders{Local Guide}{Local Guide}

ACL 2014 is located in Baltimore's newest upscale neighborhood, Harbor
East, which is nestled between touristy Inner Harbor to the west and
historic Fells Point to the east. Head west along the waterfront to
the Inner Harbor for the Aquarium, Science Center, and dozens of chain
restaurants. Stay in Harbor East for good but often quite expensive
restaurants and bars. Head further east along the waterfront past
Caroline Street to Fells Point for cobblestone streets and centuries
old bars and restaurants. Baltimore's Little Italy is just a couple
blocks north from the conference hotel.

\newcommand{\gradstudent}{\$\xspace}
\newcommand{\postdoc}{\$\$\xspace}
\newcommand{\professor}{\$\$\$\xspace}
\newcommand{\industry}{\$\$\$\$\xspace}

\subsection*{Restaurants, Bars, and Cafes by Neighborhood}

Throughout this guide, we provide distances away from the Marriott
Waterfront conference hotel. For restaurants serving full meals, we
also provide approximate relative food costs on a four point (\gradstudent to
\industry) scale. We highlight some of our very favorite spots as {\it
  editor favorites}.

\paragraph*{Harbor East}
The conference hotel is located in Harbor East, a fast developing, upscale neighborhood that replaced decaying industrial warehouses about a decade ago. In addition to a few nice restaurants, the neighborhood offers plenty of shopping, a movie theater, and easy access to Baltimore's free circulator bus and water taxi. 

\begin{itemize}
\item{Lebanese Taverna (719 S. President, 0.1 miles away): upscale Lebanese food; good for lunch or dinner and plenty of space for large groups. \professor.}
\item{Cinghiale (822 Lancaster, 0.1 miles away): upscale Italian fare and extensive wine list. \industry.}
\item{Whole Foods (1001 Fleet, 0.2 miles away); predictable grocery store fare; hot and cold food bars for a quick meal. \gradstudent.}
\item{Gordon Biersch (1000 Lancaster, 0.2 miles away): new, large bar, good for watching sports or a quick beer near the conference hotel. \postdoc.}
\item{Teavolve (1401 Aliceanna, 0.3 miles away): great for working (free wifi), morning coffee, and lunch. \gradstudent.}
\item{RA Sushi Bar Restaurant (1390 Lancaster, 0.3 miles away): Good lunch specials, great for large groups. \postdoc.} 
\item{Fleet Street Kitchen (1012 Fleet, 0.3 miles away): Upscale, beautiful interior serving excellent local meat and produce. \professor. \it{Editor favorite}.}
\end{itemize}

\paragraph*{Little Italy}
Baltimore's historic Italian neighborhood is small but bustling with delis, restaurants, and cafes. Several old warehouses nearby have also been converted into retail and office space. Take a walk up Albemarle and High Streets after dark, when they are lit with charming street lights.

\begin{itemize}
\item{Vaccaro's (222 Albemarle, 0.3 miles away): Italian bakery; stop by for a late night coffee and dessert. \gradstudent.}
\item{Chiapperelli's (237 S. High, 0.3 miles away): Our favorite restaurant in Little Italy. Good for lunch or dinner, and plenty of space for big groups. \postdoc. \it{Editor favorite}.}
\item{Heavy Seas Alehouse (1300 Bank, 0.4 miles away): Local brewpub (the brewery itself a few miles away); upscale bar food and outdoor beer garden. \professor.}
\item{Amicci's (231 S. High, 0.3 miles away). Classic, reasonably priced Italian fare. \postdoc.}
\item{La Tavola (248 Albemarle, 0.3 miles away). One of the nicer, but pricier options in the neighbhorhood. \professor.}
\item{Ciao Bella (236 S High, 0.3 miles away): You can probably find an online coupon, but the food and service are not as high quality as other spots in the neighborhood. \postdoc.}
\item{Mo's Fisherman's Wharf (219 S. President, 0.3 miles away): Tourist trappy and not particularly nice, but an easy walk from the conference center. \postdoc.}
\item{My Thai (323 S.Central, 0.4 miles away): Solid Thai food, good for big groups. \postdoc.}
\end{itemize}

\paragraph*{Fells Point}
Fells Point is a quintessential Baltimore neighborhood. Stroll along cobblestoned Thames Street for a picturesque view of the harbor, check out the centuries old rowhouses on Lancaster and Ann Streets, and dine out in one of the plethora of neighborhood restaurants. Locals claim that Fells Point has the highest number of bars per capita of any neighborhood in the United States! Be warned that many restaurants are small and won't be able to accommodate very large groups.

\begin{itemize}
\item{One-Eyed Mike's (708 S Bond, 0.4 miles away): Small, cozy spot for a meal or drink. \postdoc.}
\item{Blue Moon Cafe (1621 Aliceanna, 0.5 miles away): Legendary-ish all-day (and -night) breakfast place. Long lines, delicious food. \gradstudent.}
\item{Brick Oven Pizza (800 S. Broadway, 0.5 miles away): Pizza by the slice, BYOB, eat-in. \gradstudent.}
\item{Bertha's (734 S. Broadway, 0.5 miles away): Touristy seafood restaurant with surprisingly great food. Eat Bertha's mussels. \postdoc.}
\item{Sticky Rice (1634 Aliceanna, 0.5 miles away): Hipster Sushi. Come early for a great happy hour. \postdoc.}
\item{Birds of a Feather (1712 Aliceanna, 0.6 miles away): Quiet, local scotch bar. \postdoc.}
\item{Max's Taphouse (737 S Broadway, 0.6 miles away): Bro taphouse. Over 100 beers on tap. \postdoc.}
\item{The Horse You Came In On Saloon (1626 Thames, 0.6 miles away): Lively bar with live music everyday. Edgar Allan Poe had his last drink here before he was found dying in a nearby ditch. \postdoc.}
\item{Bond Street Social (901 S Bond, 0.6 miles away): Upscale bar/restaurant with great outdoor seating. \professor.}
\item{Kooper's Tavern (1702 Thames, 0.6 miles away): A favorite of locals and tourists alike! Bar/restaurant with excellent burgers and beer selection. \postdoc. \it{Editor favorite}.}
\item{Duda's Tavern (1600 Thames, 0.6 miles away): Small bar/restaurant, local hangout, great crabcakes and friendly service. \postdoc. \it{Editor favorite}.}
\item{Mezze (1606 Thames, 0.6 miles away): Good tapas with some outdoor seating. Make a reservation for large groups. \professor.}
\item{Bar (Lancaster, 0.6 miles away): Local favorite. The service is bad, beer is cheap, food is unavailable, and pool table is broken, but Bar is always a fun night out. \gradstudent. \it{Editor favorite}.}
\item{Thames Street Oyster House (1728 Thames, 0.7 miles away): Delicious but pricey seafood. \professor.}
\item{Wharf Rat (801 S. Ann, 0.7 miles away ): Exceptionally cozy bar that serves not-great bar food but has an excellent beer selection and a pool table in the back. \gradstudent.}
\item{Nanami (907 S. Ann, 0.7 miles away): Small, cozy sushi on the water. \postdoc.}
\item{Tortilleria Sinaloa (1716 Eastern, 0.7 miles away): There's a lot of good Mexican food on Eastern Ave., but this is one of the best. Stop by for a tamale for lunch. \gradstudent. \it{Editor favorite}.}
\item{Darbar (1911 Aliceanna, 0.7 miles away): Fells Point's only Indian restaurant. Good for large groups at lunch or dinner. \postdoc.}
\item{John Steven, Ltd (1800 Thames, 0.7 miles away); Small bar with decent food. Just on the edge of Fells Point rowdiness. \postdoc.}
\item{Daily Grind (1720 Thames, 0.7 miles away): Morning coffee or a quick lunch; free wifi and plenty of space to work. \gradstudent.}
\item{Ale Mary's (1939 Fleet, 0.8 miles away): Worth the walk! Small bar/restaurant. Try the tater tots and Chicken Chesapeake. \postdoc. \it{Editor favorite}.}
\item{Red Star Bar and Grill (906 S Wolfe, 0.8 miles away): Very good bar food and nice, historic building. Worth the extra walk. \postdoc.}
\item{Spirits Tavern (1901 Bank, 0.9 miles away): Located in a former funeral home, Spirits is a real neighborhood gem. Come to meet some friendly locals. \gradstudent. \it{Editor favorite}.}
\end{itemize}

\paragraph*{Inner Harbor}
Baltimore's Inner Harbor was revitalized in the 1970s and 80s and has since served as the city's tourist hub. However, it has declined some in recent years as new development has preferred other waterfront neighborhoods, including Harbor East. The neighborhood does still offer an excellent waterfront view of the city and the habitual array of chain restaurants. We recommend checking out Yelp or TripAdvisor for details on the restaurants listed below, as the editors plead the 5th on opinion-giving.

\begin{itemize}
\item{McCormick \& Schmick's (711 Eastern, 0.2 miles away): Expensive dining very near the conference hotel. \professor.}
\item{Hard Rock Cafe (601 E Pratt, 0.4 miles away): Wait, this restaurant didn't go out of business in the 90s? \postdoc.}
\item{Blu Bambu (621 E Pratt, 0.4 miles away): Mediocre Asian fast food. \postdoc.}
\item{Phillips Seafood (601 E Pratt, 0.5 miles away): Baltimore's `famous' seafood restaurant. We challenge you to find a local who has dined here. \professor.}
\item{Dick's Last Resort (621 E Pratt, 0.5 miles away): [Read like a letter of recommendation:] Aptly named. Nice view of the Inner Harbor? \postdoc.}
\item{Fogo de Chao (600 E Pratt, 0.5 miles away): Meat-on-a-stick, Brazilian-style steakhouse chain. \professor.}
\item{Chipotle (621 E Pratt, 0.5 miles away): Quick, cheap burritos. \gradstudent.}
\item{Five Guys (201 E Pratt, 0.7 miles away): Good fast food burgers. \gradstudent.}
\item{The Capital Grille (500 E Pratt, 0.5 miles away): Good, expensive steaks (so they say). \industry.}
\item{Ruth's Chris Steak House (600 Water, 0.6 miles away): Good, expensive steaks (so they say). \industry.}
\item{Ruth's Chris Steak House (711 Eastern, 0.4 miles away): Sample size of 2 on this one. Remember your significance testing! \industry.}
\item{Cheesecake Factory (201 E Pratt, 0.7 miles away): You can order more than just Cheesecake, but you'd be hard pressed to find an entree with fewer than 2,000 calories. \postdoc.}
\item{Bubba Gump Shrimp Co. (301 Light, 0.9 miles away): Life is like a box of so-so seafood. \postdoc.}
\item{Rusty Scupper. (402 Key Hwy, 1.6 miles away): Expensive seafood dining very far from the conference hotel. \professor.}
\end{itemize}


\paragraph*{Mount Vernon \& Downtown} 
Just north of the Inner Harbor, the Mount Vernon neighborhood originally was inhabited by Baltimore's most wealthy and fashionable families (and today, graduate students). It features lovely brownstone architecture, the original Washington Monument, the Peabody Conservatory, the Walters Art Museum, as well as a great selection of bars and restaurants. Station North, just north of Mount Vernon, is home to some more artsy and divy venues.

\begin{itemize}
\item{Iggie's (818 N. Calvert, 1.5 miles away): Great thin-crust pizza, unconventional menu, BYOB. \postdoc. \it{Editor favorite}.}
\item{Stang of Siam (1301 Calvert, 1.8 miles away): Lovely thai restaurant with good food. Understanding of spiciness randomized daily. \postdoc.}
\item{Helmand (806 N Charles, 1.8 miles away): Afghan cuisine, often listed amongst best restaurants in Baltimore. Owned by the brother of Afghan president Hamid Karzai. \professor.}
\item{Brewer's Art (1106 N Charles, 1.9 miles away): Great brewpub with a fantastic selection of beers. Brewer's was voted America's Best Bar 2008, and is the largest brewery in Baltimore. It features an upscale restaurant section as well as a divy basement. \postdoc. \it{Editor favorite}.}
\item{Red Maple (930 N Charles, 2.1 miles away): Likely Baltimore's second-best night club. Possibly Baltimore's worst night club. Allegedly serves food. \postdoc.}
\item{Club Charles (1724 N Charles, 2.3 miles away): Divy bar lit in gloomy pink. Rather mediocre drinks, but occasional trapeze performances. \gradstudent.}
\item{Joe Squared (133 W North, 2.5 miles away): Pizza and beer place, hosts local live music of varying quality on most nights.  Build your own pizza from a wealth of delicious, partially roof-grown toppings. \postdoc.}
\item{Alewife (21 N Eutaw, 1.5 miles away): Located near the Hippodrome and Lexington Market, this restaurant-bar sports an ever-changing draft list of 40 well-selected craft beers, along with a near-interminable list of bottled beers. Their burger is one of the best in the city. \postdoc. {\it Editor favorite}.}
\end{itemize}


\paragraph*{Federal Hill}
Federal Hill is another popular Baltimore neighborhood for eating, drinking, and residing. Located near both the Orioles and Ravens stadiums, folks in this part of town are even more fanatical about our local teams than most Baltimore residents. From the conference hotel, you can reach it by taking the free water taxi straight across the water to the south. Federal Hill Park provides excellent views of the city, and the neighborhood is a close second in the number of bars per capita ranking. 

\begin{itemize}
\item{Little Havana (1325 Key Highway, 2.5 miles away on land, 0.7 miles on water): Take the water taxi across the harbor and enjoy cuban cuisine on the water. Good for large groups. \postdoc.}
\item{Thai Arroy (1019 Light, 1.7 miles away on land, 0.9 miles away on water): Great, no-frills Thai food (and it's BYOB!). This restaurant is small and won't be able to easily accommodate large groups. \postdoc.}
\item{Cross Street Market (1065 S Charles, 1.8 miles away on land, 1.0 miles on water): Classic Baltimore city market. Buy some fresh flowers, produce, and a burner phone before grabbing a beer, crabcakes, oysters, shrimp, and sushi from Nick's, on the west end of the market. Well-worth the trip around or across the harbor. \gradstudent. \it{Editor favorite}.}
\item{Byblos (1033 Light, 1.7 miles away on land, 0.9 miles away on water): Inexpensive lebanese food with the sweetest owners we've ever met. (And it's BYOB!). \gradstudent.}
\item{Mother's Federal Hill Grille (1113 S Charles, 1.8 miles away on land, 1.0 miles on water): One of Baltimore's staple bar/restaurants. Mother's is huge but often full. It's a great place to celebrate local sports teams' wins, linger over weekend brunch, or grab a quick lunch or happy hour beer. \postdoc.}
\end{itemize}

\paragraph*{Hampden}
Formerly a blue collar neighborhood settled to provide housing for mills workers, Hampden is now one of Baltimore's premier hipster and artist neighborhoods. Plenty of small, kooky stores, good food, and local color are to be found around W 36th street, known as ``The Avenue.'' Hampden is a bit of a trek from the conference venue, but well worth an afternoon stroll if you have the time.

\begin{itemize}
\item{Grano Pasta Bar (1031 W 36th, 4.2 miles away): Tiny Grano is an amazing, family-run no-frills pasta bar. Freshly made pasta and delicious sauces. BYOB. Big Grano is Tiny Grano's big brother, in fully-grown Italian restaurant form. \gradstudent. \it{Editor favorite}.}

\item{Luigi's Italian Deli (846 W 36th, 4.3 miles away): Delicious, richly topped Italian sandwiches. \gradstudent.}

\item{De Kleine Duivel (3602 Hickory, 4.2 miles away): Simple bar specializing in Belgian beers. Good selection and plenty of space for groups. \gradstudent.}

\item{Corner BYOB (850 W 36th, 4.4 miles away): Upscale (and cash-only) restaurant that features an adventurous eaters club which serves (amongst other things) python meat. Says BYOB on every available surface in the restaurant. May be BYOB. \professor.}

\item{Spro (851 W 36th, 4.4 miles away): Hampden's solid take on the fancy coffee shop. \gradstudent.}

\item{Woodberry Kitchen (2010 Clipper Park Road, 4.8 miles away): Excellent restaurant by Hampden. Local, seasonal New American cuisine. Great for dinner and weekend brunch, great cocktails. From the Inner Harbor, take the light rail North to Woodberry station, or walk down from Hampden. Reservations highly recommended.  \professor. \it{Editor favorite}.}

\item{Artifact Coffee (1500 Union, 4.5 miles away): Woodberry Kitchen's coffee shop cousin. Great coffee, pastries, and small plates. Well worth a drop-in if you're in the area. \postdoc.}

\end{itemize}

\paragraph*{Foodie Eats}

Baltimore is home to a number of great restaurants, serving anything from local New American fare to Italian or Afghani cuisine. Though not all are nearby, these places are certainly worth a dinner trip.

\begin{itemize}
\item{Woodberry Kitchen (2010 Clipper Park Road, 4.8 miles away): Excellent restaurant by Hampden. Local, seasonal New American cuisine. Great for dinner and weekend brunch, fantastic cocktails. From the Inner Harbor, take the light rail North to Woodberry station, or walk down from Hampden. Reservations highly recommended.  \professor. \it{Editor favorite}.}
\item{Waterfront Kitchen (1417 Thames, 0.5 miles away): Situated right on the water of the harbor, this restaurant serves American cuisine with an emphasis on local ingredients. \professor.}
\item{Helmand (806 N Charles, 1.8 miles away): Afghan cuisine, often listed amongst best restaurants in Baltimore. Owned by the brother of Afghan president Hamid Karzai. \professor.}
\item{Salt (2127 E Pratt, 1.3 miles away): Upscale New American restaurant in Upper Fells Point. \professor.}
\item{Cinghiale (822 Lancaster, 0.1 miles away): Upscale Italian fare and extensive wine list. \industry.}
\end{itemize}

\paragraph*{Fancy [Expensive] Eats}

\begin{itemize}
\item{McCormick \& Schmick's (711 Eastern, 0.2 miles away): Expensive dining very near the conference hotel. \professor.}
\item{Ruth's Chris Steak House (600 Water, 0.6 miles away): Good, expensive steaks (so they say). \industry.}
\item{Black Olive (814 Bond, 0.4 miles away): Mediterranean cuisine, with an emphasis on seafood. \professor.}
\item{The Prime Rib (1101 N Calvert, 1.7 miles away): The best steakhouse in Baltimore, according to some. \industry.}
\end{itemize}

\subsection*{Transportation}

\paragraph*{From BWI to Conference Hotel}
The cheapest is to take the Light Rail Camden Line to Camden Yards and then walk, taxi, or take the circulator bus one mile east through downtown to the conference hotel. A slightly more expensive option is the SuperShuttle, a shared van service that costs about \$15 per person. It should cost about \$35 to reach the conference hotel in a private taxi. Note that shuttle and taxi drivers should also be tipped 10-15\%.

\paragraph*{From Penn Station to Conference Hotel}
The cost and time involved in traveling from Baltimore's main train station to the conference hotel will vary depending on traffic. The trip is about 3 miles and should cost no more than \$15 in light traffic. However, during both morning and evening rush hour, the trip could be frustratingly slow. Be warned!

\paragraph*{City Public Transportation}
\begin{itemize}
\item{Circulator: Baltimore city offers a free bus service, the Charm City Circulator. Several routes connect the neighborhoods surrounding the harbor with one another and Mount Vernon. More information can be found at \url{http://www.charmcitycirculator.com/}. The next stop and bus ETA can be found on \url{www.nextbus.com}.}
\item{Water Taxi: the Baltimore Water Taxi connects a number of spots along the harbor by boat. Its service includes the free Harbor Connector that connects the conference venue in Harbor East with Federal Hill on the other side of the harbor.}
\end{itemize}

\paragraph*{To/From DC}
Take the free Orange line Circulator bus (which picks up just around the corner from the conference hotel) to Camden Yards and then hop on the Camden Line MARC train to DC's Union Station. Another option is to take a taxi to Penn Station and then the Penn MARC train to DC's Union Station. 

\subsection*{Cultural}

\paragraph*{Art Museums}
\begin{itemize}
\item{American Visionary Art Museum (800 Key Highway, 1.8 miles away, or 0.8 miles via the ferry): Located in a former whiskey warehouse, Baltimore's AVAM offers an avant-garde art collection that is worth experiencing.\it{Editor's favorite}.}
\item{Walters Art Museum (600 N Charles, 1.6 miles away): Building on the 1931 donation by one of Baltimore's richest industrialists, this museum offers a collection of art and artifacts spanning a wide range of cultures and eras.}
\item{Baltimore Museum of Art (10 Art Museum Dr, 4.0 miles away): The BMA houses several world-class collections of art, both classical and contemporary. Located right on the Johns Hopkins University Campus, it also sports a beautiful sculpture garden.}
\end{itemize}

\paragraph*{History}
\begin{itemize}
\item{Fort McHenry (2400 E Fort): See the fort where Francis Scott Key wrote the Star Spangled Banner 200 years ago this September! The Fort is planning a big celebration of the end of the war of 1812 and has upgraded its grounds substantially in recent years. Be sure to leave time for a walk around the one mile waterfront path.}
\item{Reginald F. Lewis Museum of Maryland African American History \& Culture (830 E Pratt, 0.3 miles away): Located very close to the conference hotel, the Reginald F. Lewis Museum is an excellent museum highlighting the accomplishments of African Americans, with a focus on those from Maryland. Admission is reasonably priced (\$6 for students with a valid ID, \$8 for adults).}
\item{Baltimore's Historic Ships: See a variety of historic ships, including the US Constellation and a submarine. The ships are located throughout the Inner Harbor. Visit \url{http://www.historicships.org/} for more information.}
\end{itemize}

\paragraph*{Others}
\begin{itemize}
\item{Chessie Paddle Boating (Inner Harbor Pier 1, 0.8 miles away): Rent a dragon paddle boat and drive yourself around the Inner Harbor waters.}
\item{Maryland Science Center (601 Light, 1.1 miles away): Very nice museum, particularly for kids. It's in the Inner Harbor at the intersection of Key Highway and Light Street.}
\item{Urban Pirates Cruise (913 S Ann, 0.7 miles away): Family-friendly cruises by day adult-fun cruises by night, and friendly, energetic crews all day long. Leaves from the end of Ann Street in Fells Point, just next to Nanami restaurant.}
\item{Spirit Dinner Cruises from the Inner Harbor (561 Light St, 0.9 miles away): Overpriced dinner provides a nice view of the city from the water.}
\item{Orioles Major League Baseball Game (333 W Camden, 1.2 miles away): The Orioles will play the Chicago White Sox on Monday, Tuesday, Wednesday during ACL and the Rays on Friday and Saturday. Take the orange circulator bus from President Street Circle to Eutaw Street to see MLB's most beautiful stadium. Tickets are always available day-of from the box office. Go O's!}
\item{Live Theater at Centerstage (700 N Calvert, 1.4 miles away): ‘Wild with Happy' is showing at Centerstage the week of ACL. Check out the Centerstage website for tickets and showtimes.}
\end{itemize}

\subsection*{Working Out}
There's no need to let your fitness suffer at ACL 2014! Go for a run, play beach volleyball, or check out a local crossfit or yoga spot. 

\paragraph*{Running Routes}
The 7 mile Harbor Promenade provides a great, flat, scenic, and easily navigable running route. It goes right by the conference hotel. 

\begin{itemize}
\item{Westward: Head west along the promenade and turn around at the Aquarium (1 mile out and back), Maryland Science Center (2 miles out and back), the Harborview Water Taxi stop (3 miles out and back), or continue along Key Highway to the Under Armour headquarters (5.5 miles out and back) or to Hull Street, Fort Avenue, and Ft McHenry (7.5 miles to the Fort and back, 8.5 including a loop around scenic Ft McHenry).}
\item{Eastward: Run east along the promenade and turn around at Caroline (1 mile out and back), Henderson's Wharf (2 miles out and back), Captain James Landing (3 miles out and back), the Boston Street Pier Park/Safeway (4 miles out and back), or the Canton Waterfront Park (5 miles out and back).}
\item{Northward: Please don't run very far north.}
\item{Hills!: If you're looking for a hillier route, go east on Eastern Avenue to Patterson Park, a large city park with lots of recreation opportunities, including an outdoor public swimming pool and tennis courts.}
\end{itemize}

\paragraph*{Crossfit}
Out-of-town crossfitters are welcome to drop in at CrossFit Harbor East, located at 510 S. Eden Street. Fill out the waiver online before coming to class to save time.

\paragraph*{Swimming}
The Inner Harbor is the head of the Patapsco River. Go for a swim just outside of the hotel (just kidding, don't do this). 

\paragraph*{Yoga}
The Fells Point Charm City Yoga studio is located at 1807 Thames and Sanctuary Bodyworks is at 701 South Ann. Both studios welcome drop-ins; check out their schedules online! 

\paragraph*{Beach Volleyball}
Play volleyball on sand courts located right by Baltimore's beautiful waterfront. Check out \url{www.baltimorebeach.com} for more information.

\paragraph*{Elite Fitness tours}
Elite Fitness Tours offers a running tour through some of Baltimore's most historic neighborhoods, including Federal Hill and Mt. Vernon. Check out their website for tour information. 
