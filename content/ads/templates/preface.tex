\documentclass[11pt]{article}
\usepackage[utf8]{inputenc} 
\usepackage[T1]{fontenc} % fonts to encode unicode
\usepackage{times}
\sloppy
\hyphenpenalty 10000

\setlength\topmargin{-5mm} \setlength\oddsidemargin{-0cm}
\setlength\textheight{24.7cm} \setlength\textwidth{16cm}
\setlength\columnsep{0.6cm}  \newlength\titlebox \setlength\titlebox{2.00in}
\setlength\headheight{5pt}   \setlength\headsep{0pt}
\setlength\footskip{1.0cm}
\setlength\leftmargin{0.0in}
\pagestyle{empty}

\setlength{\parindent}{0in}
\setlength{\parskip}{2ex}

\begin{document}

\begin{center}
  {\Large \bf Message from the General Chair}
\end{center}

It is my delightful duty as General Chair to sit at 
my kitchen table here in San Francisco and write these words welcoming you to the 58th
Annual Meeting of the Association for Computational Linguistics.

Our conference this year is of course very different than in the past;
I'll be attending the conference from my kitchen table as well.
This is our first experience of ACL as a virtual conference, a 
shift due to a great trial to all of us, the COVID-19 virus.

Our hope in designing this year's conference was to draw strength from this tragedy and come together
as a community. We wanted the conference to offer a beacon of inclusion,
making it much easier for people all over the globe, whatever
their resources or backgrounds, to come to share their knowledge
and learn from each other, in a safe, welcoming, and exciting environment.
And we wanted the conference to offer a message of sustainability,
proving that even without the environmental costs of thousands of people flying around the globe,
and despite the lack of face-to-face cameraderie that helps bind us together,
we could nonetheless send our words and thoughts and
around the globe and build something together in another way.

Our challenge was to do so in a few months, and with little prior experience of our own.
I am so proud of our program chairs Joyce Chai, Natalie Schluter, and Joel Tetreault,
and the entire organizing committee, for rising to the challenge and putting together this wonderful meeting.

We have many people to thank.  Joyce, Natalie, and Joel, as is our ACL custom,
bore the brunt of the organizational burden, and managed beautifully despite all the
simultaneous demands of the whirlwinds of their daily
work and home lives.  The unflappable and wise Priscilla Rasmussen.
The amazing 52-person organizing committee, who all turned on a dime
to make the conference work virtually: Local Chairs (Jianfeng Gao, Luke Zettlemoyer), Tutorial Chairs (Agata Savary, Yue Zhang),
Workshop Chairs (Milica Ga\v{s}i\'{c}, Dilek Hakkani-Tur, Saif M. Mohammad, Ves Stoyanov),
Student Research Workshop Chairs (Rotem Dror, Jiangming Liu, Shruti Rijhwani, Yizhong Wang),
Faculty Advisors to the Student Research Workshop (Omri Abend, Sujian Li, Zhou Yu),
Conference Handbook Chair (Nanyun Peng),
Demonstration Chairs (Asli Celikyilmaz, Shawn Wen),
Diversity and Inclusion Chairs (Cecilia Ovesdotter Alm, Vinodkumar Prabhakaran),
Diversity and Inclusion Sub-Committee Chairs (Academic Inclusion Chairs: Aakanksha Naik, Emily Prud'hommeaux, Alla Rozovskaya;
Accessibility Chairs: Sushant Kafle, Masoud Rouhizadeh, Naomi Saphra;
Childcare Chairs: Khyathi Chandu, Stephen Mayhew;
Financial Access Chairs: Allyson Ettinger, Ryan Georgi, Tirthankar Ghosal;
Socio-cultural Inclusion Chairs: Shruti Palaskar, Maarten Sap),
Local Sponsorship Chairs (Hoifung Poon, Kristina Toutanova),
Publication Chairs (Steven Bethard, Ryan Cotterell, Rui Yan),
Virtual Infrastructure Chairs (Hao Fang, Sudha Rao),
Virtual Infrastructure Committee (Yi Luan, Hamid Palangi, Lianhui Qin, Yizhe Zhang),
Publicity Chairs (Emily M. Bender, Esther Seyffarth),
Sustainability Chairs (Ananya Ganesh, Klaus Zechner),
Student Volunteer Coordinator (Marjan Ghazvininejad),
Website Chairs (Sudha Rao, Yizhe Zhang)

The ACL Executive Committee gave excellent guidance and advice. Extra-special thanks to
ACL Officers Nitin Madnani, Matt Post,  and David Yarowsky.
We drew heavily on the infrastructure pioneered by Sasha Rush and
the ICLR organization committee at ICLR 2020, together with lots
of advice from the organizers of other virtual conferences and the
ACM.

We are, as always, extremely grateful to our sponsors, listed on the previous page.

And finally, thanks to you, the thousands of members of our community
who made this conference possible by writing papers, recording talks, reviewing and area chairing the papers,
being invited speakers, and perhaps most important, by reading

\bigskip
\noindent Dan Jurafsky\\
\noindent ACL 2020 General Chair\\
\noindent July 2020\\


\pagebreak

\begin{center}
  {\Large \bf Message from the Program Chairs}
\end{center}

Welcome to the 58th Annual Meeting of the Association for Computational Linguistics! ACL 2020 has a special historical significance as this is a particularly exciting period for our field: our field has grown dramatically, NLP research is now ubiquitous in products, and the barrier to entry to the field has lowered considerably.  Finally, ACL 2020 is the first ever virtual conference in the community's history. As the world combats the COVID-19 pandemic we are very grateful for all of your support and contributions which make ACL 2020 exciting and memorable. 

ACL 2020 received 3,429 submissions--an all-time record for ACL-related conferences!  This number represents more than a two-fold increase in submissions from just two years ago.
The submissions were assigned to one of 25 topic tracks. This year, we introduced four new tracks:  (1) \textbf{Ethics and NLP}. Research to assess the associated ethical assumptions and consequences of our NLP applications is crucial as these NLP applications become more and more pervasive and impactful in our society.  (2) \textbf{Interpretation and Analysis of Models for NLP}.  As the community strives to push performance boundaries, understanding behaviors of state-of-the-art models becomes critical. (3) \textbf{Theory and Formalism (Linguistic and Mathematical)}. The creation of this track reflects that theoretical research in NLP belongs at ACL and ensures a group of dedicated reviewers for the fair assessment of theory papers.  
(4) \textbf{Theme: Taking Stock of Where We've Been and Where We're Going}. The last few years have witnessed unprecedented growth since the field began over sixty years ago. This track is designed to invite submissions that can provide insight for the community to assess how much we have accomplished today with respect to the past and where the field should be heading.

To meet the reviewer demands of a growing conference without compromising review quality, we initiated a large-scale reviewer recruiting effort.  All authors, except for those who explicitly chose to opt-out due to various reasons,  were required to review if called upon. We asked all authors to fill out both a global profile and a local profile form that would allow the review system to best detect conflicts of interest (COIs) and to match submissions to reviewers. We thank the overwhelming support from the community. This effort led to a pool of more than 11K candidate reviewers, from which 2,519 primary reviewers were called upon and participated in the review process.  Together with Senior Area Chairs (SACs), Area Chairs (ACs), primary reviewers, and secondary reviewers, we have the  largest ever program committee in the history of ACL with 3319 members,  marking a 47\% increase over ACL 2019 (2,256 members).

In addition, we launched a new pilot mentoring program.  It is of central importance for our community to mentor and train our new reviewers in order to keep up with the community's rapid growth, both in terms of submissions and in terms of new members of the community, and in order to maintain review quality.  In this mentoring program, we pair Area Chairs with mentees (often a Ph.D. student, or a junior researcher who has just graduated) during the review process. The goal is to provide
 mentoring to new reviewers. The response was very positive. Over 280  ACs and 290  junior reviewers  participated in the program.   The results of this pilot will inform ACL on constructing more scalable mentoring efforts in the future.

After the review process, 779 papers were accepted which includes  571 long papers and 208 short papers. The acceptance rate is  22.7\% based on 3,429 submissions.\footnote{Removing the 29 desk rejects and 312 withdrawals, the acceptance rate becomes 25.2\%} As in previous years, the acceptance rate for long papers is higher than that for short papers (25.4\% vs. 17.6\%). Overall, ACL continues to be a highly competitive conference. From the accepted papers, and based on the nominations from Senior Area Chairs, five award-winning papers were selected by a best paper committee, including one best paper and one best theme paper.

Continuing the tradition, ACL 2020 will also feature 31 papers that were published at {\em Transactions of the Association for Computational Linguistics} (TACL) and, for the first time in ACL history, 7 papers from the journal of {\em Computational Linguistics} (CL).  Another highlight of our program is the two exciting keynote talks: one by Professor Kathleen McKeown from Columbia University, and the other one by Professor Josh Tenenbaum from MIT. 

Putting together a program for the virtual conference is a new challenge this year. We are fortunate that we were able to learn a lot from ICLR which had a virtual meeting ahead of us. One main issue was making the program accessible to attendees/authors from different time zones.  Inspired by the ICLR model, we structured the program with pre-recorded video presentations and live Q\&A sessions for individual papers. We thank the authors for providing us their time-slot preferences in a timely manner.  Our plenary sessions include live-streamed keynote talks and Q\&As, award ceremonies, and business meetings.  

ACL 2020 would not be possible without the support from the community. There are many people we would like to thank for their significant contributions!

\begin{itemize}
\item Our awesome 40 Senior Area Chairs who were instrumental in every aspect of the review process. For many of them, the scope of their responsibilities was equivalent to chairing a mini-conference. We could always count on them for their input to final decisions, selection of best papers, and outstanding reviewers.

\item The 299 Area Chairs who led paper review discussions, wrote meta-reviews, and mentored junior reviewers. 
 
\item Our 2,519 primary reviewers and 458 secondary reviewers who provided valuable feedback to the authors. Special thanks to those who stepped in at the last minute to serve as emergency reviewers. 

\item Our fantastic Best Paper Committee: Christy Doran (chair), Chris Callison-Burch, Yvette Graham, Julia Hirschberg, Rebecca Hwa, Min Yen Kan, Emily Pitler, Dragomir Radev, Philip Resnik, and Yulia Tsvetkov for selecting five award-winning papers under a tight schedule. 

\item ACL Executive Review Committee.  In particular, Amanda Stent and Arya McCarthy for making the COI detection software available and Graham Neubig for the automatic reviewer-paper assignment software. These tools were instrumental in assigning papers to reviewers. 

\item Our student assistants Shane Storks, Sayan Gosh, Tianchun Huang, Sky Wang, and Tianrong Zhang who helped check the compliance of every single submission. 

\item Our 7,711 authors who submitted their work for review at ACL 2020. Although we were only able to accept a fraction of the submissions, their hard work makes this conference exciting and our community strong.

\item TACL editors-in-chief Mark Johnson, Ani Nenkova, and Brian Roark, TACL Editorial
Assistant Cindy Robinson, and CL Editor-in-Chief Hwee Tou Ng for coordinating TACL and CL presentations with us. 

\item The Program co-Chairs of ACL 2019, Anna Korhonen and David Traum; of NAACL 2019, Christy Doran and Thamar Solorio; of EMNLP 2019, Jing Jiang, Vincent Ng, and Xiaojun Wan for generously sharing their experience, documentation, and advice in organizing ACL conferences and for answering our questions, often on short notice.  

\item Our Publication Chairs, Steven Bethard, Ryan Cotterell, and Rui Yan, for a smooth transition to the production of the final proceedings.

\item Matt Post, the ACL Anthology Director, for his always fast response to our questions. 

\item Our Publicity Chair, Emily Bender, and our Web Chairs, Sudha Rao and Yizhe Zhang, for effectively communicating conference updates and other useful information.

\item Infrastructure Chairs, Hao Feng and Sudha Rao, for taking a heavy load of moving our program online; and Hamid Palangi and Lianhui Qin for coordinating presentations with SlideLive. 

\item Rich Gerber at SoftConf, who was always quick to respond to our emails and resolve any difficulties we encountered with the START system. 

\item Priscilla Rasmussen for helpful discussion and insight into organizing an ACL at this scale. 

\item ICLR chairs, especially  Alexander Rush, Shakir Mohamed, and Kyunghyun Cho, for sharing with us many invaluable tips for running a virtual conference. 

\item ACL Executive Committee, especially Hinrich Schütze, the ACL president, and Barbara Di Eugenio, the liaison for conferences to help us sort through policy issues.

\item Our students, interns, postdocs, colleagues, and families.  Sorry for ignoring you the past year.  We're back!

\item And last but not least, our General Chair Dan Jurafsky.  He has been open-minded and supportive, giving us the flexibility to innovate while providing an invaluable sounding board, and of course, successfully led the massive turn-around of ACL as a physical conference into a virtual one in just a few short months.

\end{itemize}

Our deepest gratitude to all of you. We hope you will enjoy this new conference experience. 
 
\noindent
Joyce Chai, University of Michigan \\
Natalie Schluter, Google Brain and IT University of Copenhagen \\
Joel Tetreault, Dataminr

ACL 2020 Program Committee Co-Chairs

\end{document}