\chapter{Local Guide}
\setheaders{Local Guide}{Local Guide}

\emph{This guide was originally written by Tony Wirth, with additions by 
  Jey Han Lau.}
\index{Lau, Jey Han}
\index{Wirth, Tony}

Melburnians are only too happy to tell you that their city is the cultural
capital of Australia, or the sporting capital of Australia.
Though true, Melbourne's real international claim is that its food is 
world class and its coffee is even better.
A combination of strong competition and a multicultural
population that frequently dines out
has kindled a vibrant caf{\'e} and restaurant scene.
Below are some recommendations for places to visit.

\begin{description}
\item[Food] \url{https://www.broadsheet.com.au/melbourne}
\item[Caf{\'e}] \url{https://www.beanhunter.com/melbourne}
\item[Things To Do] \url{http://www.visitvictoria.com/Regions/Melbourne}
%\item[Google Maps Places] \url{https://goo.gl/maps/6Q96QDCHbJD2} 
%        \parbox{}{\centering
%        \includegraphics[width=0.3in, 
%height=0.3in]{content/local-guide/gmaps.pdf}}
\end{description}



\paragraph{Central Business District}

The CBD, as we like to call it, aka Melbourne 3000 -- or Downtown if you want to
sound like a tourist -- has the highest density of restaurants in 
greater Melbourne. In general, the streets to the east of Elizabeth 
Street, especially, East of Russell Street, are best for evenings.
For an outstanding gelato experience, with experimental flavours, drop 
into Gelateria Primavera. Chinatown has its spine along Little Bourke 
Street, including the long-standing Golden Orchids, while Wonderbao, to 
the north of the CBD, is the latest trend.

Closest to the conference venue,
the most vibrant area
is near the corner of
Katherine Place and Flinders Lane.
Further east along Flinders Lane,
you can find Dukes coffee roasters.
Degraves Street, either side of Flinders Lane, is packed with caf{\'e}s,
and is very popular with visitors and locals alike.

There are several quality \emph{rooftop bars} in the CBD,
including Bomba, Siglo, Campari
House, Rooftop Bar (Curtin House), Madame Brussels, and Red Hummingbird.
For a newish view of the city, and of local commuters rushing for
their trains, head to the delightful Arbory. Either side, along the 
Yarra River, Ponyfish Island and Riverland are gems.

Explore Melbourne's {\em laneways}: some of them have amazing places to 
eat,
some have fantastic artwork, some are just a little scary looking.

\paragraph{South Melbourne}

Some of the great caf{\'e}s of Melbourne, St Ali, and Chez Dr{\'e}
are just a few blocks south of the conference venue.

\paragraph{Fitzroy}

Taking the Number 96 tram north for about 25 minutes, you'll arrive in
Fitzroy.
Apart from the CBD, this inner suburb has the highest concentration of
restaurants.
There are classic atmospheric pubs, such as The Napier, Labour in
Vain, and The Standard, as well as the newer Naked for Satan with its 
rooftop
Naked in the Sky.
The caf{\'e}s are top notch, with Industry Beans
featuring in an article by {\em The Huffington Post} on the most hipster
neighborhoods in the world.
In general, Brunswick Street and Smith Street reward aimless wandering.

 

\paragraph{Bayside}

Port Melbourne is a short ride down the 109 tram. Bay Street has several good
eateries, including bakery Noisette.
St Kilda is the southern terminus of the Number 96 tram. The Sunday 
esplanade
market is a Melbourne classic, the scenic railway at Luna Park is the 
oldest
continuously running roller coaster in the world, and the continental 
cakes
along Acland Street have kept the area buzzing for decades.
