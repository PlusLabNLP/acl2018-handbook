\chapter{Local Guide}
\thispagestyle{emptyheader}
\setheaders{Local Guide}{Local Guide}

\emph{This guide was originally written by Tony Wirth, with additions by 
  Jey Han Lau.}
\index{Lau, Jey Han}
\index{Wirth, Tony}

Melburnians are only too happy to tell you that their city is the cultural
capital of Australia, or the sporting capital of Australia.
Though true, Melbourne's real international claim is that its food is 
world class and its coffee is even better.
A combination of strong competition and a multicultural
population that frequently dines out
has kindled a vibrant caf{\'e} and restaurant scene.
Below are some recommendations for places to visit.

\begin{description}
\item[Food] \url{https://www.broadsheet.com.au/melbourne}
\item[Caf{\'e}] \url{https://www.beanhunter.com/melbourne}
\item[Things To Do] \url{http://www.visitvictoria.com/Regions/Melbourne}
%\item[Google Maps Places] \url{https://goo.gl/maps/6Q96QDCHbJD2} 
%        \parbox{}{\centering
%        \includegraphics[width=0.3in, 
%height=0.3in]{content/local-guide/gmaps.pdf}}
\end{description}



\paragraph{Central Business District}

The CBD, as we like to call it, aka Melbourne 3000 -- or Downtown if you want to
sound like a tourist -- has the highest density of restaurants in 
greater Melbourne. In general, the streets to the east of Elizabeth 
Street, especially, East of Russell Street, are best for evenings.
For an outstanding gelato experience, with experimental flavours, drop 
into Gelateria Primavera. Chinatown has its spine along Little Bourke 
Street, including the long-standing Golden Orchids, while Wonderbao, to 
the north of the CBD, is the latest trend.

Closest to the conference venue,
the most vibrant area
is near the corner of
Katherine Place and Flinders Lane.
Further east along Flinders Lane,
you can find Dukes coffee roasters.
Degraves Street, either side of Flinders Lane, is packed with caf{\'e}s,
and is very popular with visitors and locals alike.

There are several quality \emph{rooftop bars} in the CBD,
including Bomba, Siglo, Campari
House, Rooftop Bar (Curtin House), Madame Brussels, and Red Hummingbird.
For a newish view of the city, and of local commuters rushing for
their trains, head to the delightful Arbory. Either side, along the 
Yarra River, Ponyfish Island and Riverland are gems.

Explore Melbourne's {\em laneways}: some of them have amazing places to 
eat,
some have fantastic artwork, some are just a little scary looking.

\paragraph{South Melbourne}

Some of the great caf{\'e}s of Melbourne, St Ali and Chez Dr{\'e},
are just a few blocks south of the conference venue.

\paragraph{Fitzroy}

Taking the Number 96 tram north for about 25 minutes, you'll arrive in
Fitzroy.
Apart from the CBD, this inner suburb has the highest concentration of
restaurants.
There are classic atmospheric pubs, such as The Napier, Labour in
Vain, and The Standard, as well as the newer Naked for Satan with its 
rooftop
Naked in the Sky.
The caf{\'e}s are top notch, with Industry Beans
featuring in an article by {\em The Huffington Post} on the most hipster
neighborhoods in the world.
In general, Brunswick Street and Smith Street reward aimless wandering.

 

\paragraph{Bayside}

Port Melbourne is a short ride down the 109 tram. Bay Street has several good
eateries, including bakery Noisette.
St Kilda is the southern terminus of the Number 96 tram. The Sunday 
esplanade
market is a Melbourne classic, the scenic railway at Luna Park is the 
oldest
continuously running roller coaster in the world, and the continental 
cakes
along Acland Street have kept the area buzzing for decades.


\paragraph{Getting Around}

Melbourne is famous for its trams, and has one of the most extensive
tram networks in the world. The CBD is designated a Free Tram Zone,
meaning that you can ride any tram for free, although beware, as the
Free Tram Zone finishes one stop short of the conference venue and 
ticket inspectors frequent the fringes of the zone trying to catch out
those who ride without a ticket, and are infamous for their intolerance
(including tourists). If you wish to catch a tram beyond the Free Tram
Zone, you will need to purchase a Myki ticket from a newsagent or one of
the many vending machines at tram stops, and ``touch on'' each time you
get on a tram. Note that if you touch on within the Free Tram Zone, you
will be charged. No, not the most user-friendly system in the world, but
there are clear announcements in the trams of whether you are in the
Free Tram Zone or not.




\paragraph{How not to Blend in}

Ways of standing out as a tourist in Melbourne include:
\begin{itemize}
\item buying coffee at Starbucks (Melbournians do pride themselves on
  their local caf\'e culture!)
\item meaning anything other than ``Aussie rules football'' when
  referring to ``football'' (Melbournians are, in large part, famously
  one-eyed when it comes to football codes)... in fact using the term
  ``football'' at all, as Australians love to abbreviate everything,
  including ``footie''
\item not having an immediate response and breaking into impassioned
  dialogue/song when asked ``who do you barrack for'' (referring, of
  course, to the footie team you support)
\item tipping --- tipping culture is very limited in Melbourne, and it
  is only at high-end restaurants where there is really any expectation
  of a tip, and even here it is optional. When eating out in large
  groups, high-end restaurants will sometimes charge a group surcharge,
  meaning even less reason to tip. Certainly there is no need to tip in
  taxis or at caf\'es (other than in the form of loose change in the
  tipping jar).
\item not having a humorous/whimsical come-back at the ready at all
  times --- Melbournians are generally a very friendly, laid-back bunch
  who try not to take themselves too seriously (except when it comes to
  footie, of course), and like to light-heartedly ``take the piss'' when
  the opportunity arises
\item comparing Sydney with Melbourne favourably in any way ---
  Melbournians are very proud of their city, and fiercely territorial
  when it comes to comparisons with Sydney
\item asking about ``Australian'' eating options --- Melbourne is
  proudly multicultural and very proud of its ``foodie'' culture, and
  ``Australian'' cuisine is representative of that: a melting pot of the
  myriad of different cuisines of the many migrant groups who make up
  its population; if there is an ``Australian'' cuisine, it is in the
  blending/fusion of different cuisines. Particular cuisines where
  Melbourne excels include Chinese, Japanese, Korean, Vietnamese,
  Malaysian, Indonesian, Italian, and Greek, with many fantastic options
  within easy access of the conference venue.
\end{itemize}


%%% Local Variables:
%%% mode: latex
%%% TeX-master: t
%%% End:
