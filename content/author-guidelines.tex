\chapter{ACL Author Guidelines}
\thispagestyle{emptyheader}
\setheaders{ACL Author Guidelines}{ACL Author Guidelines}

\section*{Preserving Double Blind Review}\vspace{2em}

The following rules and guidelines are meant to protect the integrity of double-blind review and ensure that submissions are reviewed fairly. The rules make reference to the anonymity period, which runs from 1 month before the submission deadline up to the date when your paper is either accepted, rejected, or withdrawn.

\begin{itemize}

    \item You may not make a non-anonymized version of your paper available online to the general community (for example, via a preprint server) during the anonymity period. By a version of a paper we understand another paper having essentially the same scientific content but possibly differing in minor details (including title and structure) and/or in length (e.g., an abstract is a version of the paper that it summarizes).
    \item If you have posted a non-anonymized version of your paper online before the start of the anonymity period, you may submit an anonymized version to the conference. The submitted version must not refer to the non-anonymized version, and you must inform the program chair(s) that a non-anonymized version exists. You may not update the non-anonymized version during the anonymity period, and we ask you not to advertise it on social media or take other actions that would further compromise double-blind reviewing during the anonymity period.
    \item Note that, while you are not prohibited from making a non-anonymous version available online before the start of the anonymity period, this does make double-blind reviewing more difficult to maintain, and we therefore encourage you to wait until the end of the anonymity period if possible. Alternatively, you may consider submitting your work to the Computational Linguistics journal, which does not require anonymization and has a track for "short" (i.e., conference-length) papers.

\end{itemize}

\pagebreak

\section*{Citation and Comparison}\vspace{2em}

If you are aware of previous research that appears sound and is relevant to your work, you should cite it even if it has not been peer-reviewed, and certainly if it influenced your own work. However, refereed publications take priority over unpublished work reported in preprints. Specifically:
    
\begin{itemize}
    \item You are expected to cite all refereed publications relevant to your submission, but you may be excused for not knowing about all unpublished work (especially work that has been recently posted and/or is not widely cited).
    \item In cases where a preprint has been superseded by a refereed publication, the refereed publication should be cited in addition to or instead of the preprint version.
\end{itemize}

Papers (whether refereed or not) appearing less than 3 months before the submission deadline are considered contemporaneous to your submission, and you are therefore not obliged to make detailed comparisons that require additional experimentation and/or in-depth analysis.

