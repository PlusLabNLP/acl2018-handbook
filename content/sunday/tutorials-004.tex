\begin{bio}
\small

{\bfseries Dan Roth} is a Professor in the Department of Computer Science at the University of Illinois at Urbana-Champaign and the Beckman Institute of Advanced Science and Technology (UIUC) and a University of Illinois Scholar. He is a fellow of AAAI, ACL and the ACM. Roth has published broadly in machine learning, natural language processing, knowledge representation and reasoning and received several paper, teaching and research awards. He has developed several machine learning based natural language processing systems that are widely used in the computational linguistics community and in industry and has presented invited talks and tutorials in several major conferences. Over the last few years he has worked on Entity Linking and Wikification. He has given several tutorials at ACL/NAACL/ECL and other forums.

{\bfseries Heng Ji} is the Edward G. Hamilton Development Chair Associate Professor in Computer Science Department of Rensselaer Polytechnic Institute. Her research interests focus on Natural Language Processing, especially on Cross-source Information Extraction and Knowledge Base Population. She coordinated the NIST TAC Knowledge Base Population task in 2010, 2011 and 2014 and has published several papers on entity linking and Wikification.

{\bfseries Ming-Wei Chang} is a researcher at Microsoft Research. His research interests are in machine learning and natural language understanding. He currently focuses on using large-scale structured and unstructured data for semantic understanding. Specially, he is interested in developing algorithms for entity linking that are effective for short and noisy text.

{\bfseries Taylor Cassidy} is a Postdoctoral Researcher at U.S. Army Research Laboratory \& IBM Research. His research interests include Cross-lingual Entity Linking and Wikification for social media.
\end{bio}

\begin{tutorial}{Wikification and Beyond: The Challenges of Entity and Concept Grounding}
  {Dan Roth (UIUC), Heng Ji (RPI), Ming-Wei Chang (MSR), Taylor Cassidy (ARL, IBM)}
  {Sunday, June 22, 2014, 9:00 -- 12:30pm}
  {\TutLocD}

Contextual disambiguation and grounding of concepts and entities in natural language text are essential to moving forward in many natural language understanding related tasks and are fundamental to many applications. The Wikification task aims at automatically identifying concept mentions appearing in a text document and linking them to (or “grounding them in”) concept referents in a knowledge base (KB) (e.g., Wikipedia). For example, consider the sentence, "The Times report on Blumenthal (D) has the potential to fundamentally reshape the contest in the Nutmeg State.". A Wikifier should identify the key entities and concepts (Times, Blumental, D and the Nutmeg State), and disambiguate them by mapping them to an encyclopedic resource revealing, for example, that “D” here represents the Democratic Party, and that “the Nutmeg State” refers Connecticut. Wikification may benefit both human end-users and Natural Language Processing (NLP) systems. When a document is Wikified a reader can more easily comprehend it, as information about related topics and relevant enriched knowledge from a KB is readily accessible. From a system-to-system perspective, a Wikified document conveys the meanings of its key concepts and entities by grounding them in an encyclopedic resource or a structurally rich ontology.

The primary goals of this tutorial are to review the framework of Wikification and motivate it as a broad paradigm for cross-source linking for knowledge enrichment. We will present and discuss multiple dimensions of the task definition, present the basic building blocks of a state-of-the-art Wikifier system, share some key lessons learned from the analysis of evaluation results, and discuss recently proposed ideas for advancing work in this area along with some of the key challenges. We will also suggest some research questions brought up by new applications, including interactive Wikification, social media, and censorship. The tutorial will be useful for both senior and junior researchers with interests in cross-source information extraction and linking, knowledge acquisition, and the use of acquired knowledge in natural language processing and information extraction. We will try to provide a concise roadmap of recent perspectives and results, as well as point to some of our Wikification resources that are available to the research communities.

\end{tutorial}
