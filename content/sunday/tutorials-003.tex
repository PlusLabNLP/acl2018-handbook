\begin{bio}
{\bfseries Florian Metze} received his PhD from Universitat Karlsruhe (TH) in 2005. He worked as a Senior Research Scientist at Deutsche Telekom Laboratories (T-Labs) and joined Carnegie Mellon University's faculty in 2009. His interests includes speech and audio processing, and user interfaces.

{\bfseries Koichi Shinoda} received his D.\ Eng. from Tokyo Institute of Technology in 2001. In 1989, he joined NEC Corporation. From 1997 to 1998, he was a visiting scholar with Bell Labs, Lucent Technologies. He is currently a Professor at the Tokyo Institute of Technology. His research interests include speech recognition, video information retrieval, and human interfaces.
\end{bio}

\begin{tutorial}{Semantics for Large-Scale Multimedia: New Challenges for NLP}
  {Florian Metze (CMU) and Koichi Shinoda (TokyoTech)}
  {Sunday, June 22, 2014, 9:00 -- 12:30pm}
  {\TutLocC}

Thousands of videos are constantly being uploaded to the web, creating
a vast resource, and an ever-growing demand for methods to make them
easier to retrieve, search, and index. As it becomes feasible to
extract both low-level as well as high-level (symbolic) audio, speech,
and video features from this data, these need to be processed further,
in order to learn and extract meaningful relations between these. The
language processing community has made huge process in analyzing the
vast amounts of very noisy text data that is available on the
Internet. While it is very difficult to create semantic units of
low-level image descriptors or non-speech sounds by themselves, it is
comparatively easy to ground semantics in the word output of a speech
recognizer, or text data that is loosely associated with a video. This
creates an opportunity for NLP researchers to use their unique skills,
and make significant contributions to solve tasks on data that is even
noisier than web text, but (we argue) even more interesting and
challenging.

This tutorial aims to present to the NLP community the state of the
art in audio and video processing, by discussing the most relevant
tasks at NIST's TREC Video Retrieval Evaluation (TRECVID) workshop
series. We liken "Semantic Indexing" (SIN) task, in which a system
must identify occurrences of concepts such as "desk", or "dancing" in
a video to the word spotting approach. We then proceed to explain more
recent, and challenging tasks, "Multimedia Event Detection" (MED) and
"Multimedia Event Recounting" (MER), which can be compared to
transcription and summarization tasks. Finally, we will present an
easy way to get started in multi-media analysis using Virtual Machines
from the ``Speech Recognition Virtual Kitchen'', which will enable
tutorial participants to perform hands-on experiments during the
tutorial, and at home.

\end{tutorial}
