%\thispagestyle{myheadings}
\section{Keynote Address: Josh Tenenbaum}
\label{keynote-2}
\index{Tenenbaum, Josh}

\begin{center}
\begin{Large}
{\bfseries\Large Cognitive and computational building blocks for more human-like language in machines}
\vspace{1em}\par
\end{Large}

% \daydateyear, 9:00am--10:00am \vspace{1em}\\
\PlenaryLoc \\
\vspace{1em}\par
%\includegraphics[height=100px]{content/tuesday/popovic-headshot.jpg}
\end{center}

\noindent

{\bf Abstract:} Humans learn language building on more basic conceptual and computational resources that we can already see precursors of in infancy.  These include capacities for causal reasoning, symbolic rule formation, rapid abstraction, and commonsense representations of events in terms of objects, agents and their interactions.  I will talk about steps towards capturing these abilities in engineering terms, using tools from hierarchical Bayesian models, probabilistic programs, program induction, and neuro-symbolic architectures.  I will show examples of how these tools have been applied in both cognitive science and AI contexts, and point to ways they might be useful in building more human-like language, learning and reasoning in machines.

\vspace{2ex}\centerline{\rule{.5\linewidth}{.5pt}}\vspace{2ex}
\setlength{\parskip}{1ex}\setlength{\parindent}{0ex}

{\bf Biography:} Josh Tenenbaum is Professor of Computational Cognitive Science at MIT in the Department of Brain and Cognitive Sciences, the Computer Science and Artificial Intelligence Laboratory (CSAIL) and the Center for Brains, Minds and Machines (CBMM).  He received his PhD from MIT in 1999, and taught at Stanford from 1999 to 2002. His long-term goal is to reverse-engineer intelligence in the human mind and brain, and use these insights to engineer more human-like machine intelligence.  His current research focuses on the development of common sense in children and machines, the neural basis of common sense, and models of learning as Bayesian program synthesis. His work has been published in Science, Nature, PNAS, and many other leading journals, and recognized with awards at conferences in Cognitive Science, Computer Vision, Neural Information Processing Systems, Reinforcement Learning and Decision Making, and Robotics. He is the recipient of the Distinguished Scientific Award for Early Career Contributions in Psychology from the American Psychological Association (2008), the Troland Research Award from the National Academy of Sciences (2011), the Howard Crosby Warren Medal from the Society of Experimental Psychologists (2016), the R&D Magazine Innovator of the Year award (2018), and a MacArthur Fellowship (2019). He is a fellow of the Cognitive Science Society, the Society for Experimental Psychologists, and a member of the American Academy of Arts and Sciences.

Website: {\tt https://web.mit.edu/cocosci/josh.html}

\newpage
