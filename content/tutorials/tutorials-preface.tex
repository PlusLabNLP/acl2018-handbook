\section{Message from the Tutorial Co-Chairs}\vspace{2em}
%\setheaders%
%    {Message from the Tutorial Co-Chairs}%
%    {Message from the Tutorial Co-Chairs}
\thispagestyle{emptyheader}
%\renewcommand{\large}{\fontsize{9}{11}\selectfont}
% that's a hack to make this part nicely fill the pages

\setlength{\parskip}{.7ex}
%\setlength{\parindent}{0pt}

Welcome to the Tutorials Session of ACL 2020.

The ACL tutorials session is organized to give conference attendees a comprehensive introduction by expert researchers to some topics of importance drawn from our rapidly growing and changing research field.

This year, as has been the tradition over the past few years,  the call, submission, reviewing and selection of tutorials were coordinated jointly for multiple conferences: ACL, AACL-IJCNLP, COLING and EMNLP.
We formed a review committee of 19 members, including the ACL tutorial chairs (Agata Savary and Yue Zhang), the EMNLP tutorial chairs (Banjamin van Durme and Aline Villavicencio), the COLING tutorial chairs (Daniel Beck and Lucia Specia), the AACL-IJCNMP tutorial chairs (Timothy Baldwin and Fei Xia) and 11 external reviewers (Emily Bender, Erik Cambria, Ga\"{e}l Dias, Stefan Evert, Yang Liu, Jo\~{a}o Sedoc, Xu Sun, Yulia Tsvetkov, Taro Watanabe, Aaron Steven White and Meishan Zhang). 
A reviewing process was organised so that each proposal receives 3 reviews. The selection criteria included clarity, preparedness, novelty, timeliness, instructors' experience, likely audience, open access to the teaching materials, diversity (multilingualism, gender, age and geolocation) and the compatibility of preferred venues.
A total of 43 tutorial submissions were received, of which 8 were selected for presentation at ACL.

We solicited two types of tutorials, including cutting-edge themes and introductory themes. The 8 tutorials for ACL include of 3 introductory tutorials and 5 cutting-edge tutorials. The introductory tutorials are dedicated to reviewing, ethics and commonsense reasoning in NLP. The cutting-edge discussions address interpretability of neural NLP, multi-modal information extraction and dialogue, stylized text generation, and open-domain question answering.

We would like to thank the tutorial authors for their contributions and flexibility while organising the conference virtually.  We are also grateful to the 11 external reviewers for their generous help in the decision process. Finally, our thanks go to the conference organizers for effective collaboration, and in particular to the general chair Dan Jurafsky, the website chairs Sudha Rao and Yizhe Zhang, the publicity chair Emily Bender, the ACL anthology director Matt Post.

We hope you enjoy the tutorials.

\bigskip
\noindent ACL 2020 Tutorial Co-chairs  \\*  % start a new line in the same paragraph
\noindent Agata Savary  \\*  % start a new line in the same paragraph
\noindent Yue Zhang
\index{Savary, Agata}
\index{Zhang, Yue}
