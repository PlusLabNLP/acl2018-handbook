\begin{tutorial}{Multi-lingual Entity Discovery and Linking}
  {tutorial-024}
  {\daydateyear, \tutorialafternoontime}
  {\TutLocH}
\end{tutorial}

The primary goals of this tutorial are to review the framework of cross-lingual
EL and motivate it as a broad paradigm for the Information Extraction task. We
will start by discussing the traditional EL techniques and metrics and address
questions relevant to the adequacy of these to across domains and languages. We
will then present more recent approaches such as Neural EL, discuss the basic
building blocks of a state-of-the-art neural EL system and analyze some of the
current results on English EL. We will then proceed to Cross-lingual EL and
discuss methods that work across languages. In particular, we will discuss and
compare multiple methods that make use of multi-lingual word embeddings. We
will also present EL methods that work for both name tagging and linking in
very low resource languages. Finally, we will discuss the uses of cross-lingual
EL in a variety of applications like search engines and commercial product
selling applications. Also, contrary to the 2014 EL tutorial, we will  also 
focus on Entity Discovery which is an essential component of EL.


\vspace{2ex}\centerline{\rule{.5\linewidth}{.5pt}}\vspace{2ex}
\setlength{\parskip}{1ex}\setlength{\parindent}{0ex}



{\bfseries Avirup Sil} is a Research Staff Member and the chair
of the NLP community at IBM Research AI.
His research interests are in multi-lingual information
extraction from large text collection
(cross-lingual entity extraction, disambiguation
and slot filling), machine learning
and knowledge representation. Avi has published
several papers on Entity Linking and
his systems at IBM have obtained top scores
in TAC-KBP annual multi-lingual entity linking
evaluations. Avi is an area chair for Information
Extraction at NAACL 2018 and also
for COLING 2018. He is also organizing
the workshop on the “Relevance of Linguistic
Structure in Neural NLP” at ACL 2018.

{\bfseries Heng Ji} is the Edward G. Hamilton Development
Chair Professor in Computer Science
Department of Rensselaer Polytechnic Institute.
Her research interests focus on Natural
Language Processing, especially on Crosssource
Information Extraction and Knowledge
Base Population. She coordinated the
NIST TAC Knowledge Base Population task
since 2010 and has published many papers
on entity discovery and linking. Heng has
co-taught the “Wikification and Beyond: The
Challenges of Entity and Concept Grounding”
tutorial with Dan Roth at ACL 2014.

{\bfseries Dan Roth} is the Eduardo D. Glandt Distinguished
Professor at the Department of Computer
and Information Science, University
of Pennsylvania. He is a fellow of AAAS,
AAAI, ACL, and the ACM and the winner of
the IJCAI-2017 John McCarthy Award, for
“major conceptual and theoretical advances
in the modeling of natural language understanding,
machine learning, and reasoning.”
Roth has published broadly in machine learning,
natural language processing, knowledge
representation and reasoning, and has developed
several machine learning based natural
language processing systems that are widely
used in the computational linguistics community
and in industry. Over the last few
years he has worked on Entity Linking and
Wikification. He has taught several tutorials
at ACL/NAACL/ECL and other forums.
Dan has co-taught the “Wikification and Beyond:
The Challenges of Entity and Concept
Grounding” tutorial with Heng Ji at ACL
2014.

{\bfseries Silviu-Petru Cucerzan} is a Principal Researcher
at Microsoft Research and the Bing Knowledge
Graph group. His research has focused
on topics at the intersection of NLP
and IR with concrete applications to industry,
including multilingual spelling correction,
question answering, entity recognition and
linking, query suggestion, vertical search,
and ads selection. Many of the technologies
developed by Silviu have been shipped with
Microsoft products. The NEMO entity linking
system developed by Silviu has scored
the top performance during the four consecutive
years it participated in the TAC-KBP
evaluations organized by NIST and LDC.



