\begin{bio}
{\bfseries Avirup Sil} 

{\bfseries Heng Ji}

{\bfseries Dan Roth}

{\bfseries Silviu-Petru Cucerzan}

\end{bio}

\begin{tutorial}{Multi-lingual Entity Discovery and Linking}
  {tutorial-024}
  {\daydateyear, \tutorialafternoontime}
  {\TutLocH}


The primary goals of this tutorial are to review the framework of cross-lingual EL and motivate it as a broad paradigm for the Information Extraction task. We will start by discussing the traditional EL techniques and metrics and address questions relevant to the adequacy of these to across domains and languages. We will then present more recent approaches such as Neural EL, discuss the basic building blocks of a state-of-the-art neural EL system and analyze some of the current results on English EL. We will then proceed to Cross-lingual EL and discuss methods that work across languages. In particular, we will discuss and compare multiple methods that make use of multi-lingual word embeddings. We will also present EL methods that work for both name tagging and linking in very low resource languages. Finally, we will discuss the uses of cross-lingual EL in a variety of applications like search engines and commercial product selling applications. Also, contrary to the 2014 EL tutorial, we will also focus on Entity Discovery which is an essential component of EL.

The tutorial will be useful for both senior and junior researchers (in academia and industry) with interests in cross-source information extraction and linking, knowledge acquisition, and the use of acquired knowledge in natural language processing and information extraction. We will try to provide a concise road-map of recent approaches, perspectives, and results, as well as point to some of our EL resources that are available to the research community.

\end{tutorial}
