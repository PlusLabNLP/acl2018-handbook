
\begin{tutorial}
  {}
  {tutorials-042}
  {\daydateyear, \tutorialtimec \\\tutorialtimezone}
  {\TutLocF}
  {\TutLevelF}
\end{tutorial}


\tutorialabstract{}{}{}{}{tutorials-042}

\vspace{2ex}\centerline{\rule{.5\linewidth}{.5pt}}\vspace{2ex}
\setlength{\parskip}{1ex}\setlength{\parindent}{0ex}

{\small

{\bfseries Maarten Sap} is a PhD student in the Paul G.
Allen School of Computer Science \& Engineering at the University of Washington. His research focuses primarily on social applications of
NLP, specifically on endowing machines with social intelligence, social commonsense, or theory
of mind.

{\bfseries Vered Shwartz} is a postdoctoral researcher at
the Allen Institute for Artificial Intelligence (AI2)
and the Paul G. Allen School of Computer Science \& Engineering at the University of Washington, working on lexical semantics, multiword expressions, and commonsense reasoning. She coorganized the ACL 2018 Student Research Workshop, the SemEval 2018 shared task on hypernymy discovery, and the AAAI 2020 Workshop
on Reasoning for Complex Question Answering,
Special Edition on Commonsense Reasoning.

{\bfseries Antoine Bosselut} is a PhD student in the Paul
G. Allen School of Computer Science \& Engineering at the University of Washington and a student
researcher at the Allen Institute for Artificial Intelligence (AI2). His research interests are in integrating commonsense knowledge and reasoning
into downstream applications for text generation,
summarization, and conversational dialogue. He
organized the West Coast NLP (WeCNLP) in 2018
and 2019 and the NeuralGen workshop at NAACL
2019.

{\bfseries Yejin Choi} is an associate professor at the Paul
G. Allen School of Computer Science \& Engineering at the University of Washington and also
a senior research manager at AI2 overseeing the
project Mosaic. Her research interests include
language grounding with vision, physical and social commonsense knowledge, language generation with long-term coherence, conversational AI,
and AI for social good. She was a recipient of
Borg Early Career Award (BECA) in 2018, among
the IEEEs AI Top 10 to Watch in 2015, a corecipient of the Marr Prize at ICCV 2013, and a
faculty advisor for the Sounding Board team that
won the inaugural Alexa Prize Challenge in 2017.
She was on the steering committee of the NeuralGen workshop at NAACL 2019.

{\bfseries Dan Roth} is the Eduardo D. Glandt Distinguished Professor at the Department of Computer
and Information Science, University of Pennsylvania, and a Fellow of the AAAS, the ACM, AAAI,
and the ACL. In 2017 Roth was awarded the John
McCarthy Award, the highest award the AI community gives to mid-career AI researchers. He was
the Editor-in-Chief of the Journal of Artificial Intelligence Research (JAIR) and a program co-chair
of AAAI, ACL and CoNLL. Dan has presented
several tutorials in conferences including at ACL,
on entity linking, temporal reasoning, transferable
representation learning, and more.

}