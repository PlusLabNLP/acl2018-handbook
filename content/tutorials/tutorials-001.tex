
\begin{tutorial}
  {100 Things You Always Wanted to Know about Semantics \& Pragmatics But Were Afraid to Ask}
  {tutorial-012}
  {\daydateyear, \tutorialmorningtime}
  {\TutLocA}
\end{tutorial}

Meaning is a fundamental concept in Natural Language Processing (NLP), given
its aim to build systems that mean what they say to you, and understand what
you say to them.  In order for NLP to scale beyond partial, task-specific
solutions, it must be informed by what is known about how humans use language
to express and understand communicative intents.  The purpose of this tutorial
is to present a selection of useful information about semantics and pragmatics,
as understood in linguistics, in a way that's accessible to and useful for NLP
practitioners with minimal (or even no) prior training in linguistics. The
tutorial content is based on a manuscript in progress I am co-authoring with
Prof. Alex Lascarides of the University of Edinburgh.

\vspace{2ex}\centerline{\rule{.5\linewidth}{.5pt}}\vspace{2ex}
\setlength{\parskip}{1ex}\setlength{\parindent}{0ex}

%\begin{bio}
{\bfseries Emily M. Bender} is a Professor in the Department of Linguistics and Adjunct Professor in the Paul G. Allen School of Computer Science \& Engineering at the University of Washington. She is also the past chair (2016–2017) of NAACL. Her research interests lie in multilingual grammar engineering, computational semantics, and the incorporation of linguistic knowledge in natural language processing. She is the primary developer of the Grammar Matrix grammar customization system, which is developed in the context of the DELPH-IN Consortium (Deep Linguistic Processing with HPSG Initiative). More generally, she is interested in the intersection of linguistics and computational linguistics, from both directions: bringing computational methodologies to linguistic science and linguistic science to natural language processing.
  
%\end{bio}



