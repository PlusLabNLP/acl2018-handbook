\begin{bio}
  {\bfseries Scott Wen-tau Yih} is a Researcher in the Machine
  Learning Group at Microsoft Research Redmond. His research interests
  include natural language processing, machine learning and
  information retrieval. Yih received his Ph.D. in computer science at
  the University of Illinois at Urbana-Champaign. His work on joint
  inference using integer linear programming (ILP) [Roth \& Yih, 2004]
  helped the UIUC team win the CoNLL-05 shared task on semantic role
  labeling, and the approach has been widely adopted in the NLP
  community. After joining MSR in 2005, he has worked on email spam
  filtering, keyword extraction and search \& ad relevance. His recent
  work focuses on continuous semantic representations using neural
  networks and matrix/tensor decomposition methods, with applications
  in lexical semantics, knowledge base embedding and question
  answering. Yih received the best paper award from CoNLL-2011 and has
  served as area chairs (HLT-NAACL-12, ACL-14) and program co-chairs
  (CEAS-09, CoNLL-14) in recent years.

  {\bfseries Xiaodong He} is a Researcher of Microsoft Research,
  Redmond, WA, USA. He is also an Affiliate Professor in Electrical
  Engineering at the University of Washington, Seattle, WA, USA. His
  research interests include deep learning, information retrieval,
  natural language understanding, machine translation, and speech
  recognition. Dr. He has published a book and more than 70 technical
  papers in these areas, and has given tutorials at international
  conferences in these fields. In benchmark evaluations, he and his
  colleagues have developed entries that obtained No. 1 place in the
  2008 NIST Machine Translation Evaluation (NIST MT) and the 2011
  International Workshop on Spoken Language Translation Evaluation
  (IWSLT), both in Chinese-English translation, respectively. He
  serves as Associate Editor of IEEE Signal Processing Magazine and
  IEEE Signal Processing Letters, as Guest Editors of IEEE TASLP for
  the Special Issue on Continuous-space and related methods in natural
  language processing, and Area Chair of NAACL2015. He also served as
  GE for several IEEE Journals, and served in organizing committees
  and program committees of major speech and language processing
  conferences in the past. He is a senior member of IEEE and a member
  of ACL. 

  {\bfseries Jianfeng Gao} is a Principal Researcher of Microsoft
  Research, Redmond, WA, USA. His research interests include Web
  search and information retrieval, natural language processing and
  statistical machine learning. Dr. Gao is the primary contributor of
  several key modeling technologies that help significantly boost the
  relevance of the Bing search engine. His research has also been
  applied to other MS products including Windows, Office and Ads. In
  benchmark evaluations, he and his colleagues have developed entries
  that obtained No. 1 place in the 2008 NIST Machine Translation
  Evaluation in Chinese-English translation. He was Associate Editor
  of ACM Trans on Asian Language Information Processing, (2007 to
  2010), and was Member of the editorial board of Computational
  Linguistics (2006 – 2008). He also served as area chairs for
  ACL-IJCNLP2015, SIGIR2015, SIGIR2014, IJCAI2013, ACL2012, EMNLP2010,
  ACL-IJCNLP 2009, etc. Dr.\ Gao recently joined Deep Learning
  Technology Center at MSR-NExT, working on Enterprise Intelligence.
\end{bio}

\begin{tutorial}
  {tutorial-final-009}
  {\daydateyear, \tutorialafternoontime}
  {\TutLocE}

Deep learning techniques have demonstrated tremendous success in the
speech and language processing community in recent years, establishing
new state-of-the-art performance in speech recognition, language
modeling, and have shown great potential for many other natural
language processing tasks. The focus of this tutorial is to provide an
extensive overview on recent deep learning approaches to problems in
language or text processing, with particular emphasis on important
real-world applications including language understanding, semantic
representation modeling, question answering and semantic parsing, etc.

In this tutorial, we will first survey the latest deep learning
technology, presenting both theoretical and practical perspectives
that are most relevant to our topic. We plan to cover common methods
of deep neural networks and more advanced methods of recurrent,
recursive, stacking and convolutional networks. In addition, we will
introduce recently proposed continuous-space representations for both
semantic word embedding and knowledge base embedding, which are
modeled by either matrix/tensor decomposition or neural networks.

Next, we will review general problems and tasks in text/language
processing, and underline the distinct properties that differentiate
language processing from other tasks such as speech and image object
recognition. More importantly, we highlight the general issues of
natural language processing, and elaborate on how new deep learning
technologies are proposed and fundamentally address these issues. We
then place particular emphasis on several important applications,
including (1) machine translation, (2) semantic information retrieval
and (3) semantic parsing and question answering. For each task, we
will discuss what particular architectures of deep learning models are
suitable given the nature of the task, and how learning can be
performed efficiently and effectively using end-to-end optimization
strategies.

\end{tutorial}
