ACL has always prided itself as a venue that allows for the open
exchange of ideas and the freedom of thought and expression. In
keeping with these beliefs, to codify them and to ensure that ACL
becomes immediately aware of any deviation from these principles, ACL
has instituted an anti-harassment policy in coordination with NAACL.

Harassment and hostile behavior is unwelcome at any ACL conference;
including speech or behavior that intimidates, creates discomfort, or
interferes with a person's participation or opportunity for
participation in the conference. We aim for ACL conferences to be
environments where harassment in any form does not happen, including
but not limited to harassment based on race, gender, religion, age,
color, national origin, ancestry, disability, sexual orientation, or
gender identity. Harassment includes degrading verbal comments,
deliberate intimidation, stalking, harassing photography or recording,
inappropriate physical contact, and unwelcome sexual attention.

If you are a victim of harassment or hostile behaviour at an ACL
conference, or otherwise observe such behaviour toward someone else,
please contact any of the following people:

\begin{itemize}
\item Any current member of the ACL board
\item Hal Daume III \index{Daume, Hal}
\item Julia Hirschberg \index{Hirschberg, Julia}
\item Su Jian \index{Jian, Su}
\item Priscilla Rasmussen \index{Rasmussen, Priscilla}

Please be assured that if you approach us, your concerns will be kept
in strict confidence, and we will consult with you on the actions
taken by the Board.

The full policy and its implementation is defined at:

\url{http://naacl.org/policies/anti-harassment.html}
