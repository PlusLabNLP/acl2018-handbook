\clearpage
\section*{Special Cultural Event: \daydate, 8:00 PM}
\setheaders{Special Cultural Event}{\daydateyear}

ACL 2014 welcomes all attendees to an outstanding professional production of the acclaimed satirical play about language - {\it The Memo} - written by dissident playwright V\'{a}clav Havel, the first president of the Czech Republic.

This special reprise performance of a recent sold-out production of Baltimore's award-winning Single Carrot Theater will be held in the Arellano Theater of Johns Hopkins University.

Tickets are free at the ACL registration desk to the first 100 requesting attendees, thanks to the generous full sponsorship of the Center for Language and Speech Processing at Johns Hopkins University.

The play will be preceded by a 7:15PM welcome reception and optional tour of the Johns Hopkins University Homewood campus and its Center for Language and Speech Processing, one of the two largest NLP/MT/Speech programs in the United States.  Transportation details will be provided to ticket holders at the conference.

\noindent\makebox[\linewidth]{\rule{4in}{0.4pt}}

\noindent {\bf The Memo} \\
\noindent By Václav Havel \\

\noindent Translated from Czech by Paul Wilson \\
\noindent Directed by Stephen Nunns \\

A comic workplace classic about linguistics and the perils of standardized communication.

Company President Mr. Gross receives a memorandum but can’t read it because it’s written in Ptydepe*, the newly invented language to which all correspondence must adhere. If he can’t figure out what it says, he’ll lose his job and certainly his mind. This satirical take on bureaucracy and office malarkey is an incisive look at 20th century Communist Czechoslovakia, but it could just as easily be today’s America.

The play focuses on two artificial languages, Ptydepe and Chorukor, at the heart of the play's satire. Ptydepe was constructed along strictly scientific lines, without the ambiguities of natural languages. To avoid the possibilities for confusion that arise with homonyms, Ptydepe was created according to the postulate that all words must be formed from the least probable combinations of letters. Specifically, it makes use of the so-called "sixty percent dissimilarity" rule; which states that any Ptydepe word must differ by at least sixty percent of its letters from any other word consisting of the same number of letters. Length of words, like everything else in Ptydepe, is determined scientifically. The vocabulary of Ptydepe uses entropy encoding: shorter words have more common meanings. Therefore, the shortest word in Ptydepe, {\it gh}, corresponds to what is believed to be the most general term in natural language, {\it whatever}. (The longest word in Ptydepe, which contains 319 letters, is a name for a nonexistent member of the genus {\it Apus}). Havel's younger brother, computer scientist Ivan M. Havel, helped in its formulation. [Thanks to Single Carrot Theater and Wikipedia]

This performance is approximately 2 hours and 15 minutes, including a
15 minute intermission.  Members of the Johns Hopkins CLSP faculty
loved this play when they saw it recently and wanted to share a final
performance of this extremely entertaining production with you.
