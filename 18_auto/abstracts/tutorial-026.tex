Developing an intelligent dialogue system that not only emulates human conversation, but also can answer questions of topics ranging from latest news of a movie star to Einstein's theory of relativity, and fulfill complex tasks such as travel planning, has been one of the longest running goals in AI. The goal remains elusive until recently when we started observing promising results in both the research community and industry as the large amount of conversation data is available for training and the breakthroughs in deep learning (DL) and reinforcement learning (RL) are applied to conversational AI. In this tutorial, we start with a brief introduction to the recent progress on DL and RL that is related to natural language processing and conversational AI. Then, we describe in detail the state-of-the-art neural approaches developed for three types of dialogue systems. The first is a question answering (QA) agent. Equipped with rich knowledge drawn from various data sources including Web documents and pre-complied knowledge graphs (KG's), the QA agent can provide concise direct answers to user queries.  The second is a task-oriented dialogue system that can help users accomplish tasks ranging from meeting scheduling to vacation planning. The third is a social chat bot which can converse seamlessly and appropriately with humans, and often plays roles of a chat companion and a recommender. In the final part of the tutorial, we review attempts to developing open-domain conversational AI systems that combine the strengths of different types of dialogue systems.
