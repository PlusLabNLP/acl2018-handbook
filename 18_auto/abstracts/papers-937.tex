We present a set of experiments to demonstrate that deep recurrent neural networks (RNNs) learn internal representations that capture soft hierarchical notions of syntax from highly varied supervision. We consider four syntax tasks at different depths of the parse tree; for each word, we predict its part of speech as well as the first (parent), second (grandparent) and third level (great-grandparent) constituent labels that appear above it. These predictions are made from representations produced at different depths in networks that are pretrained with one of four objectives: dependency parsing, semantic role labeling, machine translation, or language modeling. In every case, we find a correspondence between network depth and syntactic depth, suggesting that a soft syntactic hierarchy emerges. This effect is robust across all conditions, indicating that the models encode significant amounts of syntax even in the absence of an explicit syntactic training supervision.
