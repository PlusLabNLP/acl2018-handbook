Supervised domain adaptation---where a large generic corpus and a smaller in-domain corpus are both available for training---is a challenge for neural machine translation (NMT). Standard practice is to train a generic model and use it to initialize a second model, then continue training the second model on in-domain data to produce an in-domain model. We add an auxiliary term to the training objective during continued training that minimizes the cross entropy between the in-domain model's output word distribution and that of the out-of-domain model to prevent the model's output from differing too much from the original out-of-domain model. We perform experiments on EMEA (descriptions of medicines) and TED (rehearsed presentations), initialized from a general domain (WMT) model. Our method shows improvements over standard continued training by up to 1.5 BLEU.
