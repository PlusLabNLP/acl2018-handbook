Most Reading Comprehension methods limit themselves to queries which can be answered using a single sentence, paragraph, or document. Enabling models to combine disjoint pieces of textual evidence would extend the scope of machine comprehension methods, but currently no resources exist to train and test this capability. We propose a novel task to encourage the development of models for text understanding across multiple documents and to investigate the limits of existing methods. In our task, a model learns to seek and combine evidence – effectively performing multi-hop, alias multi-step, inference. We devise a methodology to produce datasets for this task, given a collection of query-answer pairs and thematically linked documents. Two datasets from different domains are induced, and we identify potential pitfalls and devise circumvention strategies. We evaluate two previously proposed competitive models and find that one can integrate information across documents. However, both models struggle to select relevant information; and providing documents guaranteed to be relevant greatly improves their performance. While the models outperform several strong baselines, their best accuracy reaches 54.5\% on an annotated test set, compared to human performance at 85.0\%, leaving ample room for improvement.