Proper names of organisations are a special case of collective nouns. Their meaning can be conceptualised as a collective unit or as a plurality of persons, allowing for different morphological marking of coreferent anaphoric pronouns. This paper explores the variability of references to organisation names with 1) a corpus analysis and 2) two crowd-sourced story continuation experiments. The first shows that the preference for singular vs. plural conceptualisation is dependent on the level of formality of a text. In the second, we observe a strong preference for the plural they otherwise typical of informal speech. Using edited corpus data instead of constructed sentences as stimuli reduces this preference.
