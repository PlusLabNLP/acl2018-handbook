There is often the need to perform sentiment classification in a particular domain where no labelled document is available. Although we could make use of a general-purpose off-the-shelf sentiment classifier or a pre-built one for a different domain, the effectiveness would be inferior. In this paper, we explore the possibility of building domain-specific sentiment classifiers with unlabelled documents only. Our investigation indicates that in the word embeddings learnt from the unlabelled corpus of a given domain, the distributed word representations (vectors) for opposite sentiments form distinct clusters, though those clusters are not transferable across domains. Exploiting such a clustering structure, we are able to utilise machine learning algorithms to induce a quality domain-specific sentiment lexicon from just a few typical sentiment words ("seeds"). An important finding is that simple linear model based supervised learning algorithms (such as linear SVM) can actually work better than more sophisticated semi-supervised/transductive learning algorithms which represent the state-of-the-art technique for sentiment lexicon induction. The induced lexicon could be applied directly in a lexicon-based method for sentiment classification, but a higher performance could be achieved through a two-phase bootstrapping method which uses the induced lexicon to assign positive/negative sentiment scores to unlabelled documents first, and then uses those documents found to have clear sentiment signals as pseudo-labelled examples to train a document sentiment classifier via supervised learning algorithms (such as LSTM). On several benchmark datasets for document sentiment classification, our end-to-end pipelined approach which is overall unsupervised (except for a tiny set of seed words) outperforms existing unsupervised approaches and achieves an accuracy comparable to that of fully supervised approaches.