In a corruption of John Searle's famous AI thought experiment, the Chinese Room (Searle, 1980), we twist its original intent by enabling humans to translate text, e.g. from Uyghur to English, even if they don't have any prior knowledge of the source language. Our enabling tool, which we call the Chinese Room, is equipped with the same resources made available to a machine translation engine. We find that our superior language model and world knowledge allows us to create perfectly fluent and nearly adequate translations, with human expertise required only for the target language. The Chinese Room tool can be used to rapidly create small corpora of parallel data when bilingual translators are not readily available, in particular for low-resource languages.
