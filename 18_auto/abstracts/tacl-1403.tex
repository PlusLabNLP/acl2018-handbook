We present a computational analysis of cognate effects on the spontaneous linguistic productions of advanced non-native speakers. Introducing a large corpus of highly competent non-native English speakers, and using a set of carefully selected lexical items, we show that the lexical choices of non-natives are affected by cognates in their native language. This effect is so powerful that we are able to reconstruct the phylogenetic language tree of the Indo-European language family solely from the frequencies of specific lexical items in the English of authors with various native languages. We quantitatively analyze non-native lexical choice, highlighting cognate facilitation as one of the important phenomena shaping the language of non-native speakers.