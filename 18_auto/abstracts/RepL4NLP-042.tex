In this study, we compare token representations constructed from visual features (i.e., pixels) with standard lookup-based embeddings. Our goal is to gain insight about the challenges of encoding a text representation from low-level features, e.g. from characters or pixels. We focus on Chinese, which---as a logographic language---has properties that make a representation via visual features challenging and interesting. To train and evaluate different models for the token representation, we chose the task of character-based neural machine translation (NMT) from Chinese to English. We found that a token representation computed only from visual features can achieve competitive results to lookup embeddings. However, we also show different strengths and weaknesses in the models' performance in a part-of-speech tagging task and also a semantic similarity task. In summary, we show that it is possible to achieve a \textit{text representation} only from pixels. We hope that this is a useful stepping stone for future studies that exclusively rely on visual input, or aim at exploiting visual features of written language.
