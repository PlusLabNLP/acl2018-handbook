Shi, Huang, and Lee (2017a) obtained state-of-the-art results for English and Chinese dependency parsing by combining dynamic-programming implementations of transition-based dependency parsers with a minimal set of bidirectional LSTM features. However, their results were limited to projective parsing. In this paper, we extend their approach to support non-projectivity by providing the first practical implementation of the MH\_4 algorithm, an $O(n^4)$ mildly nonprojective dynamic-programming parser with very high coverage on non-projective treebanks. To make MH\_4 compatible with minimal transition-based feature sets, we introduce a transition-based interpretation of it in which parser items are mapped to sequences of transitions. We thus obtain the first implementation of global decoding for non-projective transition-based parsing, and demonstrate empirically that it is effective than its projective counterpart in parsing a number of highly non-projective languages.
