Recurrent neural network grammars (RNNGs) are generative models of (tree , string ) pairs that rely on neural networks to evaluate derivational choices. Parsing with them using beam search yields a variety of incremental complexity metrics such as word surprisal and parser action count. When used as regressors against human electrophysiological responses to naturalistic text, they derive two amplitude effects: an early peak and a P600-like later peak. By contrast, a non-syntactic neural language model yields no reliable effects. Model comparisons attribute the early peak to syntactic composition within the RNNG. This pattern of results recommends the RNNG+beam search combination as a mechanistic model of the syntactic processing that occurs during normal human language comprehension.
