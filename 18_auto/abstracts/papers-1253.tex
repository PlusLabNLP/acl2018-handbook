Question Answering (QA), as a research field, has primarily focused on either knowledge bases (KBs) or free text as a source of knowledge. These two sources have historically shaped the kinds of questions that are asked over these sources, and the methods developed to answer them. In this work, we look towards a practical use-case of QA over user-instructed knowledge that uniquely combines elements of both structured QA over knowledge bases, and unstructured QA over narrative, introducing the task of multi-relational QA over personal narrative. As a first step towards this goal, we make three key contributions: (i) we generate and release TextWorldsQA, a set of five diverse datasets, where each dataset contains dynamic narrative that describes entities and relations in a simulated world, paired with variably compositional questions over that knowledge, (ii) we perform a thorough evaluation and analysis of several state-of-the-art QA models and their variants at this task, and (iii) we release a lightweight Python-based framework we call TextWorlds for easily generating arbitrary additional worlds and narrative, with the goal of allowing the community to create and share a growing collection of diverse worlds as a test-bed for this task.
