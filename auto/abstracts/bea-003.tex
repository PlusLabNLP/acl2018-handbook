Multilingual corpora are difficult to compile and a classroom setting adds pedagogy to the mix of factors which make this data so rich and problematic to classify. In this paper, we set out methodological considerations of using automated speech recognition to build a corpus of teacher speech in an Indonesian language classroom. Our preliminary results (64\% word error rate) suggest these tools have the potential to speed data collection in this context. We provide practical examples of our data structure, details of our piloted computer-assisted processes, and fine-grained error analysis. Our study is informed and directed by genuine research questions and discussion in both the education and computational linguistics fields. We highlight some of the benefits and risks of using these emerging technologies to analyze the complex work of language teachers and in education more generally.
