In this study, we aim to test our hypothesis that confidence scores of sentiment values of tweets aid in classification of sentiment. We used several feature sets consisting of lexical features, emoticons, features based on sentiment scores and combination of lexical and sentiment features. Since our dataset includes confidence scores of real numbers in [0-1] range, we employ regression analysis on each class of sentiments. We determine the class label of a tweet by looking at the maximum of the confidence scores assigned to it by these regressors. We test the results against classification results obtained by converting the confidence scores into discrete labels. Thus, the strength of sentiment is ignored. Our expectation was that taking the strength of sentiment into consideration would improve the classification results. Contrary to our expectations, our results indicate that using classification on discrete class labels and ignoring sentiment strength perform similar to using regression on continuous confidence scores.
