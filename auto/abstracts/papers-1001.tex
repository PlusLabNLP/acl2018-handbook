Prior work has explored directly regularizing the output distributions of probabilistic models to alleviate peaky (i.e. over-confident) predictions, a common sign of overfitting. This class of techniques, of which label smoothing is one, has a connection to entropy regularization. Despite the consistent success of label smoothing across architectures and data sets in language generation tasks, two problems remain open: (1) there is little understanding of the underlying effects entropy regularizers have on models, and (2) the full space of entropy regularization techniques is largely unexplored. We introduce a parametric family of entropy regularizers, which includes label smoothing as a special case, and use it to gain a better understanding of the relationship between the entropy of a model and its performance on language generation tasks. We also find that variance in model performance can be explained largely by the resulting entropy of the model. Lastly, we find that label smoothing provably does not allow for sparsity in an output distribution, an undesirable property for language generation models, and therefore advise the use of other entropy regularization methods in its place.
