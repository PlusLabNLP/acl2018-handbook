An interesting and frequent type of multi-word expression (MWE) is the headless MWE, for which there are no true internal syntactic dominance relations; examples include many named entities (``Wells Fargo'') and dates (``July 5, 2020'') as well as certain productive constructions (``blow for blow'', ``day after day''). Despite their special status and prevalence, current dependency-annotation schemes require treating such flat structures as if they had internal syntactic heads, and most current parsers handle them in the same fashion as headed constructions. Meanwhile, outside the context of parsing, taggers are typically used for identifying MWEs, but taggers might benefit from structural information. We empirically compare these two common strategies—parsing and tagging—for predicting flat MWEs. Additionally, we propose an efficient joint decoding algorithm that combines scores from both strategies. Experimental results on the MWE-Aware English Dependency Corpus and on six non-English dependency treebanks with frequent flat structures show that: (1) tagging is more accurate than parsing for identifying flat-structure MWEs, (2) our joint decoder reconciles the two different views and, for non-BERT features, leads to higher accuracies, and (3) most of the gains result from feature sharing between the parsers and taggers.
