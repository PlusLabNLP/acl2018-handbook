Autism spectrum disorders are developmental disorders characterised as deficits in social and communication skills, and they affect both verbal and non-verbal communication. Previous works measured differences in children with and without autism spectrum disorders in terms of linguistic and acoustic features, although they do not mention automatic identification using integration of these features. In this paper, we perform an exploratory study of several language and speech features of both single utterances and full narratives. We find that there are characteristic differences between children with autism spectrum disorders and typical development with respect to word categories, prosody, and voice quality, and that these differences can be used in automatic classifiers. We also examine the differences between American and Japanese children and find significant differences with regards to pauses before new turns and linguistic cues.
