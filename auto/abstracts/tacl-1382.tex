We quantify the linguistic complexity of different languages’ morphological systems. We verify that there is an empirical trade-off between paradigm size and irregularity: a language’s inflectional paradigms may be either large in size or highly irregular, but never both. Our methodology quantifies paradigm irregularity as the entropy of the surface realization of a paradigm—how hard it is to jointly predict all the surface forms of a paradigm—which we estimate by a variational approximation. Our measurements are taken on large morphological paradigms from 31 typologically diverse languages.