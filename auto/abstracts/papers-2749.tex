Traditional named entity recognition models use gazetteers (lists of entities) as features to improve performance. Although modern neural network models do not require such hand-crafted features for strong performance, recent work has demonstrated their utility for named entity recognition on English data. However, designing such features for low-resource languages is challenging, because exhaustive entity gazetteers do not exist in these languages. To address this problem, we propose a method of ``soft gazetteers'' that incorporates ubiquitously available information from English knowledge bases, such as Wikipedia, into neural named entity recognition models through cross-lingual entity linking. Our experiments on four low-resource languages show an average improvement of 4 points in F1 score.
