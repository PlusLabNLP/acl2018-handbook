Inspired by experimental psychological findings suggesting that function words play a special role in word learning, we make a simple modification to an Adaptor Grammar based Bayesian word segmentation model to allow it to learn sequences of monosyllabic ``function words'' at the beginnings and endings of collocations of (possibly multi-syllabic) words.  This modification improves unsupervised word segmentation on the standard Bernstein-Ratner (1987) corpus of child-directed English by more than 4\% token f-score compared to a model identical except that it does not special-case ``function words'', setting a new state-of-the-art of 92.4\% token f-score. Our function word model assumes that function words appear at the left periphery, and while this is true of languages such as English, it is not true universally.  We show that a learner can use Bayesian model selection to determine the location of function words in their language, even though the input to the model only consists of unsegmented sequences of phones.  Thus our computational models support the hypothesis that function words play a special role in word learning.
