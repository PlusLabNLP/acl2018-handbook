The effort required for a human annotator to detect sentiment is not uniform for all texts, irrespective of his/her expertise. We aim to predict a score that quantifies this effort, using linguistic properties of the text. Our proposed metric is called Sentiment Annotation Complexity (SAC). As for training data, since any direct judgment of complexity by a human annotator is fraught with subjectivity, we rely on cognitive evidence from eye-tracking. The sentences in our dataset are labeled with SAC scores derived from eye-fixation duration. Using linguistic features and annotated SACs, we train a regressor that predicts the SAC with a best mean error rate of 22.02\% for five-fold cross-validation. We also study the correlation between a human annotator's perception of complexity and a machine's confidence in polarity determination. The merit of our work lies in (a) deciding the sentiment annotation cost in, for example, a crowdsourcing setting,(b) choosing the right classifier for sentiment prediction.
