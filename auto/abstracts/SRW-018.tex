Media bias can strongly impact the public perception of topics reported in the news. A difficult to detect, yet powerful form of slanted news coverage is called bias by word choice and labeling (WCL). WCL bias can occur, for example, when journalists refer to the same semantic concept by using different terms that frame the concept differently and consequently may lead to different assessments by readers, such as the terms ``freedom fighters'' and ``terrorists,'' or ``gun rights'' and ``gun control.'' In this research project, I aim to devise methods that identify instances of WCL bias and estimate the frames they induce, e.g., not only is ``terrorists'' of negative polarity but also ascribes to aggression and fear. To achieve this, I plan to research methods using natural language processing and deep learning while employing models and using analysis concepts from the social sciences, where researchers have studied media bias for decades. The first results indicate the effectiveness of this interdisciplinary research approach. My vision is to devise a system that helps news readers to become aware of the differences in media coverage caused by bias.
