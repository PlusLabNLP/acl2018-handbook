We survey 146 papers analyzing ``bias'' in NLP systems, finding that their motivations are often vague, inconsistent, and lacking in normative reasoning, despite the fact that analyzing ``bias'' is an inherently normative process. We further find that these papers' proposed quantitative techniques for measuring or mitigating ``bias'' are poorly matched to their motivations and do not engage with the relevant literature outside of NLP. Based on these findings, we describe the beginnings of a path forward by proposing three recommendations that should guide work analyzing ``bias'' in NLP systems. These recommendations rest on a greater recognition of the relationships between language and social hierarchies, encouraging researchers and practitioners to articulate their conceptualizations of ``bias''---i.e., what kinds of system behaviors are harmful, in what ways, to whom, and why, as well as the normative reasoning underlying these statements---and to center work around the lived experiences of members of communities affected by NLP systems, while interrogating and reimagining the power relations between technologists and such communities.
