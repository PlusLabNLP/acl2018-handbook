Effective dialogue involves grounding, the process of establishing mutual knowledge that is essential for communication between people. Modern dialogue systems are not explicitly trained to build common ground, and therefore overlook this important aspect of communication. Improvisational theater (improv) intrinsically contains a high proportion of dialogue focused on building common ground, and makes use of the yes-and principle, a strong grounding speech act, to establish coherence and an actionable objective reality. We collect a corpus of more than 26,000 yes-and turns, transcribing them from improv dialogues and extracting them from larger, but more sparsely populated movie script dialogue corpora, via a bootstrapped classifier. We fine-tune chit-chat dialogue systems with our corpus to encourage more grounded, relevant conversation and confirm these findings with human evaluations.
