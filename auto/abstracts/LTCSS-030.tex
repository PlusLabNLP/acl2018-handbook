Party brands are central to theories of Congressional action.  While previous work assumes that a party's brand---its long run reputation---is a direct consequence of the content of legislation, in this presentation I show how partisans from both parties use public statements to craft their own party's reputation and to undermine their opponents party.   The incentive to craft and destroy brands varies across legislators, creating systematic distortions in who contributes to partisan branding efforts, what is said about a party's brand, and when partisan criticism becomes salient. To demonstrate the construction of party brands I use new collections of newsletters from Congressional offices, along with press releases, floor speeches, and media broadcasts.  Across the diverse sources, I show that ideologically extreme legislators are the most likely to explain their party's work in Washington and the most likely to criticize the opposing party---particularly when their is an opposing party president.  Extreme legislators also engage in more vitriolic criticism of the opposing party, particularly when opposing presidents are unpopular.  The result is that parties in rhetoric appear even more combative and polarized in public debate outside Congress than inside Congress.
