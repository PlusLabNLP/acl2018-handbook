User generated content is bringing new aspects of processing data on the web. Due to the advancement of World Wide Web technology, users are not only consumer of web contents but also they are producers of contents in the form of text, audio, video and picture. This study focuses on the analysis of textual contents with subjective information (referring to sentiment analysis). Most of conventional approaches of sentiment analysis do not effectively capture negation in languages where there are limited computational linguistic resources (e.g. Amharic). For this research, we proposed Amharic negation handling framework for Amharic sentiment classification. The proposed framework combines the lexicon based sentiment classification approach and character ngram based machine learning algorithms. Finally, the performance of framework is evaluated using the annotated Amharic news comments. The system is performing the best of all models and the baselines with accuracy of 98.0. The result is compared with the baselines (without negation handling and word level ngram model).
