We consider the distinction between intended and perceived sarcasm in the context of textual sarcasm detection. The former occurs when an utterance is sarcastic from the perspective of its author, while the latter occurs when the utterance is interpreted as sarcastic by the audience. We show the limitations of previous labelling methods in capturing intended sarcasm and introduce the iSarcasm dataset of tweets labeled for sarcasm directly by their authors. Examining the state-of-the-art sarcasm detection models on our dataset showed low performance compared to previously studied datasets, which indicates that these datasets might be biased or obvious and sarcasm could be a phenomenon under-studied computationally thus far. By providing the iSarcasm dataset, we aim to encourage future NLP research to develop methods for detecting sarcasm in text as intended by the authors of the text, not as labeled under assumptions that we demonstrate to be sub-optimal.
