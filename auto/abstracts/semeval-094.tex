This paper describes the system used by the ValenTo team in the Task 11, Sentiment Analysis of Figurative Language in Twitter, at SemEval 2015. Our system used a regression model and additional external resources to assign polarity values. A distinctive feature of our approach is that we used not only word-sentiment lexicons providing polarity annotations, but also novel resources for dealing with emotions and psycholinguistic information. These are important aspects to tackle in figurative language such as irony and sarcasm, which were represented in the dataset. The system also exploited novel and standard structural features of tweets. Considering the different kinds of figurative language in the dataset our submission obtained good results in recognizing sentiment polarity in both ironic and sarcastic tweets.
