The Transformer model has been shown to outperform other neural seq2seq models in several character-level tasks. It is unclear, however, if the Transformer would benefit as much as other seq2seq models from data augmentation strategies in the low-resource setting. In this paper we explore strategies for data augmentation in the g2p task together with the Transformer model.  Our results show that a relatively simple alignment-based strategy of identifying consistent input-output subsequences in grapheme-phoneme data coupled together with a subsequent splicing together of such pieces to generate hallucinated data works well in the low-resource setting, often delivering substantial performance improvement over a standard Transformer model.
