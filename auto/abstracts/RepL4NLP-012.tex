In this paper, we provide an alternate perspective on word representations, by reinterpreting the dimensions of the vector space of a word embedding as a collection of features. In this reinterpretation, every component of the word vector is normalized against all the word vectors in the vocabulary. This idea now allows us to view each vector as an \$n\$-tuple (akin to a fuzzy set), where \$n\$ is the dimensionality of the word representation and each element represents the probability of the word possessing a feature. Indeed, this representation enables the use fuzzy set theoretic operations, such as union, intersection and difference. Unlike previous attempts, we show that this representation of words provides a notion of similarity which is inherently asymmetric and hence closer to human similarity judgements. We compare the performance of this representation with various benchmarks, and explore some of the unique properties including function word detection, detection of polysemous words, and some insight into the interpretability provided by set theoretic operations.
