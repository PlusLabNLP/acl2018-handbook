In this paper, we have proposed a technique for generating complex reading comprehension questions from a discourse that are more useful than factual ones derived from assertions. Our system produces a set of general-level questions using coherence relations and a set of well-defined syntactic transformations on the input text. Generated questions evaluate comprehension abilities like a comprehensive analysis of the text and its structure, correct identification of the author's intent, a thorough evaluation of stated arguments; and a deduction of the high-level semantic relations that hold between text spans. Experiments performed on the RST-DT corpus allow us to conclude that our system possesses a strong aptitude for generating intricate questions. These questions are capable of effectively assessing a student's interpretation of the text.
