In this paper we introduce the task of fact checking, i.e. the assessment of the truthfulness of a claim. The task is commonly performed manually by journalists verifying the claims made by public figures. Furthermore, ordinary citizens need to assess the truthfulness of the increasing volume of statements they consume. Thus, developing fact checking systems is likely to be of use to various members of society. We first define the task and detail the construction of a publicly available dataset using statements fact-checked by journalists available online. Then, we discuss baseline approaches for the task and the challenges that need to be addressed. Finally, we discuss how fact checking relates to mainstream natural language processing tasks and can stimulate further research.
