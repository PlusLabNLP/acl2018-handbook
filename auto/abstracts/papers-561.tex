Paraphrasing natural language sentences is a multifaceted process: it might involve replacing individual words or short phrases, local rearrangement of content, or high-level restructuring like topicalization or passivization. Past approaches struggle to cover this space of paraphrase possibilities in an interpretable manner. Our work, inspired by pre-ordering literature in machine translation, uses syntactic transformations to softly ``reorder'' the source sentence and guide our neural paraphrasing model. First, given an input sentence, we derive a set of feasible syntactic rearrangements using an encoder-decoder model. This model operates over a partially lexical, partially syntactic view of the sentence and can reorder big chunks. Next, we use each proposed rearrangement to produce a sequence of position embeddings, which encourages our final encoder-decoder paraphrase model to attend to the source words in a particular order. Our evaluation, both automatic and human, shows that the proposed system retains the quality of the baseline approaches while giving a substantial increase in the diversity of the generated paraphrases.
