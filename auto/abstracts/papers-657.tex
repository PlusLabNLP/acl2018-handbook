Given the fast development of analysis techniques for NLP and speech processing systems, few systematic studies have been conducted to compare the strengths and weaknesses of each method.  As a step in this direction we study the case of representations of phonology in neural network models of spoken language. We use two commonly applied analytical techniques, diagnostic classifiers and representational similarity analysis, to quantify to what extent neural activation patterns encode phonemes and phoneme sequences. We manipulate two factors that can affect the outcome of analysis. First, we investigate the role of learning by comparing neural activations extracted from trained versus randomly-initialized models. Second, we examine the temporal scope of the activations by probing both local activations corresponding to a few milliseconds of the speech signal, and global activations pooled over the whole utterance. We conclude that reporting analysis results with randomly initialized models is crucial, and that global-scope methods tend to yield more consistent and interpretable results and we recommend their use as a complement to local-scope diagnostic methods.
