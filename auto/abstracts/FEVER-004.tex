In the field of factoid question answering (QA), it is known that the state-of-the-art technology has achieved an accuracy comparable to that of humans in a certain benchmark challenge.  On the other hand, in the area of non-factoid QA, there is still a limited number of datasets for training QA models, i.e., machine comprehension models.  Considering such a situation within the field of the non-factoid QA, this paper aims to develop a dataset for training Japanese how-to tip QA models.  This paper applies one of the state-of-the-art machine comprehension models to the Japanese how-to tip QA dataset. The trained how-to tip QA model is also compared with a factoid QA model trained with a Japanese factoid QA dataset. Evaluation results revealed that the how-to tip machine comprehension performance was almost comparative with that of the factoid machine comprehension even with the training data size reduced to around 4\% of the factoid machine comprehension.  Thus, the how-to tip machine comprehension task requires much less training data compared with the factoid machine comprehension task.
