Dialogue state tracking (DST) is at the heart of task-oriented dialogue systems. However, the scarcity of labeled data is an obstacle to building accurate and robust state tracking systems that work across a variety of domains. Existing approaches generally require some dialogue data with state information and their ability to generalize to unknown domains is limited. In this paper, we propose using machine reading comprehension (RC) in state tracking from two perspectives: model architectures and datasets. We divide the slot types in dialogue state into categorical or extractive to borrow the advantages from both multiple-choice and span-based reading comprehension models. Our method achieves near the current state-of-the-art in joint goal accuracy on MultiWOZ 2.1 given full training data. More importantly, by leveraging machine reading comprehension datasets, our method outperforms the existing approaches by many a large margin in few-shot scenarios when the availability of in-domain data is limited. Lastly, even without any state tracking data, i.e., zero-shot scenario, our proposed approach achieves greater than 90\% average slot accuracy in 12 out of 30 slots in MultiWOZ 2.1.
