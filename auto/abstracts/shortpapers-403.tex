We present a study of aspects of discourse structure --- specifically discourse devices used to organize information in a sentence --- that significantly impact the quality of machine translation. Our analysis is based on manual evaluations of translations of news from Chinese and Arabic to English. We find that there is a particularly strong mismatch in the notion of what constitutes a sentence in Chinese and English, which occurs often and is associated with significant degradation in translation quality. Also related to lower translation quality is the need to employ multiple explicit discourse connectives (because, but, etc.), as well as the presence of ambiguous discourse connectives in the English translation. Furthermore, the mismatches between discourse expressions across languages significantly impact translation quality.
