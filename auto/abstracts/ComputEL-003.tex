The ``D'' in ``DEL'' stands for ``documenting'' --- a code word for linguists that means the collection of linguistic data in audio and written form. The DEL (Documenting Endangered Languages) program run by the NSF and NEH is thus centered around building and archiving data resources for endangered languages. This paper is an argument for extending the ‘D' to include ``describing'' languages in terms of lexical, semantic, morphological and grammatical knowledge. We present an overview of descriptive computational tools aimed at endangered languages along with a longer summary of two particular computer programs: Linguist's Assistant and Boas. These two programs, respectively, represent research in the areas of: A) computational systems capable of representing lexical, morphological and grammatical structures and using the resulting computational models for translation in a minority language context, and B) tools for efficiently and accurately acquiring linguistic knowledge. A hoped-for side effect of this paper is to promote cooperation between these areas of research in order to provide a total solution to describing endangered languages.
