Medical conversation is a central part of medical care. Yet, the current state and quality of medical conversation is far from perfect. Therefore, a substantial amount of research has been done to obtain a better understanding of medical conversation and to address its practical challenges and dilemmas. In line with this stream of research, we have developed a multi-layer structure annotation scheme to analyze medical conversation, and are using the scheme to construct a corpus of naturally occurring medical conversation in Chinese pediatric primary care setting. Some of the preliminary findings are reported regarding 1) how a medical conversation starts,  2) where communication problems tend to occur, and  3) how physicians close a conversation. Challenges and opportunities for research on medical conversation with NLP techniques will be discussed.
