While various approaches to domain adaptation exist, the majority of them requires knowledge of the target domain, and additional data, preferably labeled. For a language like English, it is often feasible to match most of those conditions, but in low-resource languages, it presents a problem. We explore the situation when neither data nor other information about the target domain is available. We use two samples of Danish, a low-resource language, from the consumer review domain (film vs. company reviews) in a sentiment analysis task. We observe dramatic performance drops when moving from one domain to the other. We then introduce a simple offline method that makes models more robust towards unseen domains, and observe relative improvements of more than 50\%.
