In order to simplify a sentence, human editors perform multiple rewriting transformations: they split it into several shorter sentences,  paraphrase words (i.e. replacing complex words or phrases by simpler synonyms), reorder components, and/or delete information deemed unnecessary. Despite these varied range of possible text alterations, current models for automatic sentence simplification are evaluated using datasets that are focused on a single transformation, such as lexical paraphrasing or splitting. This makes it impossible to understand the ability of simplification models in more  realistic settings. To alleviate this limitation, this paper introduces ASSET, a new dataset for assessing sentence simplification in English. ASSET is a crowdsourced multi-reference corpus where each simplification was produced by executing several rewriting transformations. Through quantitative and qualitative experiments, we show that simplifications in ASSET are better at capturing characteristics of simplicity when compared to other standard evaluation datasets for the task. Furthermore, we motivate the need for developing better methods for automatic evaluation using ASSET, since we show that current popular metrics may not be suitable when multiple simplification transformations are performed.
