Pretrained masked language models (MLMs) require finetuning for most NLP tasks. Instead, we evaluate MLMs out of the box via their pseudo-log-likelihood scores (PLLs), which are computed by masking tokens one by one. We show that PLLs outperform scores from autoregressive language models like GPT-2 in a variety of tasks. By rescoring ASR and NMT hypotheses, RoBERTa reduces an end-to-end LibriSpeech model's WER by 30\% relative and adds up to +1.7 BLEU on state-of-the-art baselines for low-resource translation pairs, with further gains from domain adaptation. We attribute this success to PLL's unsupervised expression of linguistic acceptability without a left-to-right bias, greatly improving on scores from GPT-2 (+10 points on island effects, NPI licensing in BLiMP). One can finetune MLMs to give scores without masking, enabling computation in a single inference pass. In all, PLLs and their associated pseudo-perplexities (PPPLs) enable plug-and-play use of the growing number of pretrained MLMs; e.g., we use a single cross-lingual model to rescore translations in multiple languages. We release our library for language model scoring at https://github.com/awslabs/mlm-scoring. This is a submission to the cross-submission track, presented at ACL 2020.
