A broad-coverage corpus such as the Human Language Project envisioned by Abney and Bird (2010) would be a powerful resource for the study of endangered languages. Existing corpora are limited in the range of languages covered, in standardisation, or in machine-readability. In this paper we present SeedLing, a seed corpus for the Human Language Project. We first survey existing efforts to compile cross-linguistic resources, then describe our own approach. To build the foundation text for a Universal Corpus, we crawl and clean texts from several web sources that contain data from a large number of languages, and convert them into a standardised form consistent with the guidelines of Abney and Bird (2011). The resulting corpus is more easily-accessible and machine-readable than any of the underlying data sources, and, with data from 1451 languages covering 105 language families, represents a significant base corpus for researchers to draw on and add to in the future. To demonstrate the utility of SeedLing for cross-lingual computational research, we use our data in the test application of detecting similar languages.
