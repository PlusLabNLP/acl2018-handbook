For phylogenetic inference, linguistic typology is a promising alternative to lexical evidence  because it allows us to compare an arbitrary pair of languages. A challenging problem with typology-based phylogenetic inference is that the changes of typological features over time are less intuitive than those of lexical features. In this paper, we work on reconstructing typologically natural ancestors To do this, we leverage dependencies among typological features. We first represent each language by continuous latent components that capture feature dependencies. We then combine them with a typology evaluator that distinguishes typologically natural languages from other possible combinations of features. We perform phylogenetic inference in the continuous space and use the evaluator to ensure the typological naturalness of inferred ancestors. We show that the proposed method reconstructs known language families more accurately than baseline methods. Lastly, assuming the monogenesis hypothesis, we attempt to reconstruct a common ancestor of the world's languages.
