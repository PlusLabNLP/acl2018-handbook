Most NLP models today treat language as universal, even though socio- and psycholingustic research shows that the communicated message is influenced by the characteristics of the speaker as well as the target audience. This paper surveys the landscape of personalization in natural language processing and related fields, and offers a path forward to mitigate the decades of deviation of the NLP tools from sociolingustic findings, allowing to flexibly process the ``natural'' language of each user rather than enforcing a uniform NLP treatment. It outlines a possible direction to incorporate these aspects into neural NLP models by means of socially contextual personalization, and proposes to shift the focus of our evaluation strategies accordingly.
