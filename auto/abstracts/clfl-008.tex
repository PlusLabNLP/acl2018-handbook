We explore the feasibility of applying machine translation (MT) to the translation of literary texts. To that end, we measure the translata- bility of literary texts by analysing parallel corpora and measuring the degree of freedom of the translations and the narrowness of the domain. We then explore the use of domain adaptation to translate a novel between two re- lated languages, Spanish and Catalan. This is the first time that specific MT systems are built to translate novels. Our best system out- performs a strong baseline by 4.61 absolute points (9.38\% relative) in terms of BLEU and is corroborated by other automatic evaluation metrics. We provide evidence that MT can be useful to assist with the translation of nov- els between closely-related languages, namely (i) the translations produced by our best sys- tem are equal to the ones produced by a pro- fessional human translator in almost 20\% of cases with an additional 10\% requiring at most 5 character edits, and (ii) a complementary hu- man evaluation shows that over 60\% of the translations are perceived to be of the same (or even higher) quality by native speakers.
