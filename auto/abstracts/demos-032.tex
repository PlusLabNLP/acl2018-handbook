We present Tabouid, a word-guessing game automatically generated from Wikipedia. Tabouid contains 10,000 (virtual) cards in English, and as many in French, covering not only words and linguistic expressions but also a variety of topics including artists, historical events or scientific concepts. Each card corresponds to a Wikipedia article, and conversely, any article could be turned into a card. A range of relatively simple NLP and machine-learning techniques are effectively integrated into a two-stage process. First, a large subset of Wikipedia articles are scored - this score estimates the difficulty, or alternatively, the playability of the page. Then, the best articles are turned into cards by selecting, for each of them, a list of banned words based on its content. We believe that the game we present is more than mere entertainment and that, furthermore, this paper has pedagogical potential.
