Gender bias in biomedical research can have an adverse impact on the health of real people. For example, there is evidence that heart disease-related funded research generally focuses on men. Health disparities can form between men and at-risk groups of women (i.e., elderly and low-income) if there is not an equal number of heart disease-related studies for both genders. In this paper, we study temporal bias in biomedical research articles by measuring gender differences in word embeddings. Specifically, we address multiple questions, including, How has gender bias changed over time in biomedical research, and what health-related concepts are the most biased? Overall, we find that traditional gender stereotypes have reduced over time. However, we also find that the embeddings of many medical conditions are as biased today as they were 60 years ago (e.g., concepts related to drug addiction and body dysmorphia).
