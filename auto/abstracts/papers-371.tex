Due to its great importance in deep natural language understanding and various down-stream applications, text-level parsing of discourse rhetorical structure (DRS) has been drawing more and more attention in recent years. However, all the previous studies on text-level discourse parsing adopt bottom-up approaches, which much limit the DRS determination on local information and fail to well benefit from global information of the overall discourse. In this paper, we justify from both computational and perceptive points-of-view that the top-down architecture is more suitable for text-level DRS parsing. On the basis, we propose a top-down neural architecture toward text-level DRS parsing. In particular, we cast discourse parsing as a recursive split point ranking task, where a split point is classified to different levels according to its rank and the elementary discourse units (EDUs) associated with it are arranged accordingly. In this way, we can determine the complete DRS as a hierarchical tree structure via an encoder-decoder with an internal stack. Experimentation on both the English RST-DT corpus and the Chinese CDTB corpus shows the great effectiveness of our proposed top-down approach towards text-level DRS parsing.
