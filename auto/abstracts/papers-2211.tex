Authorship attribution aims to identify the author of a text based on the stylometric analysis. Authorship obfuscation, on the other hand, aims to protect against authorship attribution by modifying a text's style. In this paper, we evaluate the stealthiness of state-of-the-art authorship obfuscation methods under an adversarial threat model. An obfuscator is stealthy to the extent an adversary finds it challenging to detect whether or not a text modified by the obfuscator is obfuscated --- a decision that is key to the adversary interested in authorship attribution. We show that the existing authorship obfuscation methods are not stealthy as their obfuscated texts can be identified with an average F1 score of 0.87. The reason for the lack of stealthiness is that these obfuscators degrade text smoothness, as ascertained by neural language models, in a detectable manner. Our results highlight the need to develop stealthy authorship obfuscation methods that can better protect the identity of an author seeking anonymity.
