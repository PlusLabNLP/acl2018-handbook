We present a method that learns word embedding for Twitter sentiment classification in this paper. Most existing algorithms for learning continuous word representations typically model only the syntactic context of words but ignore the sentiment of text. This is problematic for sentiment analysis as they usually map words with similar syntactic context but opposite sentiment polarity, such as good and bad, to neighboring word vectors. We address this issue by learning sentiment-specific word embedding (SSWE), which encodes sentiment information in the continuous representation of words. Specifically, we develop three neural network architectures to effectively incorporate the supervision from sentiment polarity of text (e.g. sentences or tweets) in their loss functions. To obtain large scale training corpora, we learn the sentiment-specific word embedding from massive distant-supervised tweets collected by positive and negative emoticons. Experiments on applying SSWE to a benchmark Twitter sentiment classification dataset in SemEval 2013 show that (1) the SSWE feature performs comparably with hand-crafted features in the top-performed system; (2) the performance is further improved by concatenating SSWE with existing feature set.
