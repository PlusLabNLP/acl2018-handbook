In this paper, we first open on important issues regarding the Penn Korean Universal Treebank (PKT-UD) and address these issues by revising the entire corpus manually with the aim of producing cleaner UD annotations that are more faithful to Korean grammar. For compatibility to the rest of UD corpora, we follow the UDv2 guidelines, and extensively revise the part-of-speech tags and the dependency relations to reflect morphological features and flexible word- order aspects in Korean. The original and the revised versions of PKT-UD are experimented with transformer-based parsing models using biaffine attention. The parsing model trained on the revised corpus shows a significant improvement of 3.0\% in labeled attachment score over the model trained on the previous corpus. Our error analysis demonstrates that this revision allows the parsing model to learn relations more robustly, reducing several critical errors that used to be made by the previous model.
