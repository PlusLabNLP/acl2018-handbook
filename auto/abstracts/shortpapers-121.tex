The strength with which a statement is made can have a significant impact on the audience. For example, international relations can be strained by how the media in one country describes an event in another; and papers can be rejected because they overstate or understate their findings. It is thus important to understand the effects of statement strength. A first step is to be able to distinguish between strong and weak statements. However, even this problem is understudied, partly due to a lack of data. Since strength is inherently relative, revisions of texts that make claims are a natural source of data on strength differences. In this paper, we introduce a corpus of sentence-level revisions from academic writing. We also describe insights gained from our annotation efforts for this task.
