There is growing interest and excitement around the connection between modern natural language processing (NLP) techniques and the scientific study of human language: how humans learn it and use it, and how it evolved, as studied in linguistics and cognitive science. The connection is twofold: first, modern NLP techniques might be useful for cognitive modeling. Second, the scientific study of language might produce insights that lead to improved NLP systems with more human-like linguistic intelligence. With recent developments in machine learning and large datasets, there are now unprecedented opportunities for significant advances in this field, and more generally opportunities for fruitful cross-fertilization between NLP, cognitive science, and linguistics. We propose a tutorial to introduce the areas of language acquisition, human language processing, and language evolution and emergence, from a perspective that will be comprehensible to computer scientists, giving them an entry point into these fields. By making this area more accessible to NLP researchers, we not only provide them with a chance to have an impact on science, but we also open a channel of communication between NLP, cognitive science, and linguistics, which can enable ideas to flow back into NLP.
