Usage of presuppositions in social media and news discourse can be a powerful way to influence the readers as they usually tend to not examine the truth value of the hidden or indirectly expressed information. Fairclough and Wodak (1997) discuss presupposition at a discourse level where some implicit claims are taken for granted in the explicit meaning of a text or utterance. From the Gricean perspective, the presuppositions of a sentence determine the class of contexts in which the sentence could be felicitously uttered.  This paper aims to correlate the type of knowledge presupposed in a news article to the bias present in it. We propose a set of guidelines to identify various kinds of presuppositions in news articles and present a dataset consisting of 1050 articles which are annotated for bias (positive, negative or neutral) and the magnitude of presupposition. We introduce a supervised classification approach for detecting bias in political news which significantly outperforms the existing systems.
