Due to their origin in computer graphics, graphics processing units (GPUs) are highly optimized for dense problems, where the exact same operation is applied repeatedly to all data points. Natural language processing algorithms, on the other hand, are traditionally constructed in ways that exploit structural sparsity. Recently, Canny et al. (2013) presented an approach to GPU parsing that sacrifices traditional sparsity in exchange for raw computational power, obtaining a system that can compute Viterbi parses for a high-quality grammar at about 164 sentences per second on a mid-range GPU. In this work, we reintroduce sparsity to GPU parsing by adapting a coarse-to-fine pruning approach to the constraints of a GPU. The resulting system is capable of computing over 404 Viterbi parses per second—more than a 2x speedup—on the same hardware. Moreover, our approach allows us to efficiently implement less GPU-friendly minimum Bayes risk inference, improving throughput for this more accurate algorithm from only 32 sentences per second unpruned to over 190 sentences per second using pruning—nearly a 6x speedup.
