Question Answering (QA) is in increasing demand as the amount of information available online and the desire for quick access to this content grows. A common approach to QA has been to fine-tune a pretrained language model on a task-specific labeled dataset. This paradigm, however, relies on scarce, and costly to obtain, large-scale human-labeled data. We propose an unsupervised approach to training QA models with generated pseudo-training data. We show that generating questions for QA training by applying a simple template on a related, retrieved sentence rather than the original context sentence improves downstream QA performance by allowing the model to learn more complex context-question relationships. Training a QA model on this data gives a relative improvement over a previous unsupervised model in F1 score on the SQuAD dataset by about 14\%, and 20\% when the answer is a named entity, achieving state-of-the-art performance on SQuAD for unsupervised QA.
