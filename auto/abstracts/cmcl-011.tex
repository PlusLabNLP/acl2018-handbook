Models of language acquisition are typically evaluated against a ``gold standard'' meant to represent adult linguistic knowledge, such as orthographic words for the task of speech segmentation. Yet adult knowledge is rarely the target knowledge for the stage of acquisition being modeled, making the gold standard an imperfect evaluation metric. To supplement the gold standard evaluation metric, we propose an alternative utility-based metric that measures whether the acquired knowledge facilitates future learning. We take the task of speech segmentation as a case study, assessing previously proposed models of segmentation on their ability to generate output that (i) enables creation of language-specific segmentation cues that rely on stress patterns, and (ii) assists the subsequent acquisition task of learning word meanings. We find that behavior that maximizes gold standard performance does not necessarily maximize the utility of the acquired knowledge, highlighting the benefit of multiple evaluation metrics.
