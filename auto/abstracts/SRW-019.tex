Aphasia is a speech and language disorder which results from brain damage, often characterized by word retrieval deficit (anomia) resulting in naming errors (paraphasia). Automatic paraphasia detection has many benefits for both treatment and diagnosis of Aphasia and its type. But supervised learning methods cant be properly utilized as there is a lack of aphasic speech data. In this paper, we describe our novel unsupervised method which can be implemented without the need for labeled paraphasia data. Our evaluations show that our method outperforms previous work based on supervised learning and transfer learning approaches for English. We demonstrate the utility of our method as an essential first step in developing augmentative and alternative communication (AAC) devices for patients suffering from aphasia in any language.
