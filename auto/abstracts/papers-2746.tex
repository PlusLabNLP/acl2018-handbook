With the great success of pre-trained language models, the pretrain-finetune paradigm now becomes the undoubtedly dominant solution for natural language understanding (NLU) tasks. At the fine-tune stage, target task data is usually introduced in a completely random order and treated equally. However, examples in NLU tasks can vary greatly in difficulty, and similar to human learning procedure, language models can benefit from an easy-to-difficult curriculum. Based on this idea, we propose our Curriculum Learning approach. By reviewing the trainset in a crossed way, we are able to distinguish easy examples from difficult ones, and arrange a curriculum for language models. Without any manual model architecture design or use of external data, our Curriculum Learning approach obtains significant and universal performance improvements on a wide range of NLU tasks.
