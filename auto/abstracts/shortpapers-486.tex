Pedagogical materials frequently contain deixis to communicative artifacts such as textual structures (e.g., sections and lists), discourse entities, and illustrations. By relating such artifacts to the prose, deixis plays an essential role in structuring the flow of information in informative writing. However, existing language technologies have largely overlooked this mechanism. We examine properties of deixis to communicative artifacts using a corpus rich in determiner-established instances of the phenomenon (e.g., ``this section'', ``these equations'', ``those reasons'') from Wikibooks, a collection of learning texts. We use this corpus in combination with WordNet to determine a set of word senses that are characteristic of the phenomenon, showing its diversity and validating intuitions about its qualities. The results motivate further research to extract the connections encoded by such deixis, with the goals of enhancing tools to present pedagogical e-texts to readers and, more broadly, improving language technologies that rely on deictic phenomena.
