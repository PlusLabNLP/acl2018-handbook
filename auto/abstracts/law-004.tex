The present paper describes an attempt to create an interoperable scheme for the annotation of textual phenomena across languages and genres including non-canonical ones. The resulting scheme will be tested for a contrastive analysis of English, German and Czech, for which we expect the greatest differences in the degree of variation between non-canonical and canonical language. Such a kind of analysis requires annotated multilingual resources which are costly. Therefore, we make use of annotations already available in the resources for the three languages under analysis. As the annotations in these corpora are based on different conceptual and methodological backgrounds, we need an interoperable scheme that covers existing categories and at the same time allows a comparison of the resources. In this paper, we describe how this interoperable scheme was created and which problematic cases we had to consider.
