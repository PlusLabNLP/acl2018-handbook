The increased focus on misinformation has spurred development of data and systems for detecting the veracity of a claim as well as retrieving authoritative evidence. The Fact Extraction and VERification (FEVER) dataset provides such a resource for evaluating endto- end fact-checking, requiring retrieval of evidence from Wikipedia to validate a veracity prediction. We show that current systems for FEVER are vulnerable to three categories of realistic challenges for fact-checking --- multiple propositions, temporal reasoning, and ambiguity and lexical variation --- and introduce a resource with these types of claims. Then we present a system designed to be resilient to these ``attacks'' using multiple pointer networks for document selection and jointly modeling a sequence of evidence sentences and veracity relation predictions. We find that in handling these attacks we obtain state-of-the-art results on FEVER, largely due to improved evidence retrieval.
