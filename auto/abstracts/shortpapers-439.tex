How do journalists mark quoted content as certain or uncertain, and how do readers interpret these signals? Predicates such as ``thinks'', ``claims'', and ``admits'' offer a range of options for framing quoted content according to the author's own perceptions of its credibility. We gather a new dataset of direct and indirect quotes from Twitter, and obtain annotations of the perceived certainty of the quoted statements. We then compare the ability of linguistic and extra-linguistic features to predict readers' assessment of the certainty of quoted content. We see that readers are indeed influenced by such framing devices --- and we find no evidence that they consider other factors, such as the source, journalist, or the content itself. In addition, we examine the impact of specific framing devices on perceptions of credibility.
