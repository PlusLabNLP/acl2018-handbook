Neural conversation models are known to generate appropriate but non-informative responses in general. A scenario where informativeness can be significantly enhanced is Conversing by Reading (CbR), where conversations take place with respect to a given external document. In previous work, the external document is utilized by (1) creating a context-aware document memory that integrates information from the document and the conversational context, and then (2) generating responses referring to the memory. In this paper, we propose to create the document memory with some anticipated responses in mind. This is achieved using a teacher-student framework. The teacher is given the external document, the context, and the ground-truth response, and learns how to build a response-aware document memory from three sources of information. The student learns to construct a response-anticipated document memory from the first two sources, and teacher's insight on memory creation. Empirical results show that our model outperforms the previous state-of-the-art for the CbR task.
