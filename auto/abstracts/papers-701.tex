The Bechdel test is a sequence of three questions designed to assess the presence of women in movies. Many believe that because women are seldom represented in film as strong leaders and thinkers, viewers associate weaker stereotypes with women. In this paper, we present a computational approach to automate the task of finding whether a movie passes or fails the Bechdel test. This allows us to study the key differences in language use and in the importance of roles of women in movies that pass the test versus the movies that fail the test. Our experiments confirm that in movies that fail the test, women are in fact portrayed as less-central and less-important characters.
