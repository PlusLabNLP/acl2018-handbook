We propose a Semi-supervIsed GeNerative Active Learning (SIGNAL) model to address the imbalance, efficiency, and text camouflage problems of Chinese text spam detection task. A ``self-diversity'' criterion is proposed for measuring the ``worthiness'' of a candidate for annotation. A semi-supervised variational autoencoder with masked attention learning approach and a character variation graph-enhanced augmentation procedure are proposed for data augmentation. The preliminary experiment demonstrates the proposed SIGNAL model is not only sensitive to spam sample selection, but also can improve the performance of a series of conventional active learning models for Chinese spam detection task. To the best of our knowledge, this is the first work to integrate active learning and semi-supervised generative learning for text spam detection.
