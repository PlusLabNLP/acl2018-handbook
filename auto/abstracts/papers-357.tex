Negation words, such as no and not, play a fundamental role in modifying sentiment of textual expressions. We will refer to a negation word as the negator and the text span within the scope of the negator as the argument. Commonly used heuristics to estimate the sentiment of negated expressions rely simply on the sentiment of argument (and not on the negator or the argument itself). We use a sentiment treebank to show that these existing heuristics are poor estimators of sentiment. We then modify these heuristics to be dependent on the negators and show that this improves prediction. Next, we evaluate a recently proposed composition model (Socher et al., 2013) that relies on both the negator and the argument. This model learns the syntax and semantics of the negator's argument with a recursive neural network. We show that this approach performs better than those mentioned above. Finally, we explicitly incorporate the prior sentiment of the argument and observe that this information can help further reduce fitting errors.
