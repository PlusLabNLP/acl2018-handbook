This paper describes an application of distributional semantics to the study of syntactic productivity in diachrony, i.e., the property of grammatical constructions to attract new lexical items over time. By providing an empirical measure of semantic similarity between words derived from lexical co-occurrences, distributional semantics not only reliably captures how the verbs in the distribution of a construction are related, but also enables the use of visualization techniques and statistical modeling to analyze the semantic development of a construction over time and identify the semantic determinants of syntactic productivity in naturally occurring data.
