Existing models for social media personal analytics assume access to thousands of messages per user, even though most users author content only sporadically over time.  Given this sparsity, we: (i) leverage content from the local neighborhood of a user; (ii) evaluate batch models as a function of size and the amount of messages in various types of neighborhoods; and (iii) estimate the amount of time and tweets required for a dynamic model to predict user preferences. We show that even when limited or no self-authored data is available, language from friend, retweet and user mention communications provide sufficient evidence for prediction.  When updating models over time based on Twitter, we find that political preference can be often be predicted using roughly 100 tweets, depending on the context of user selection, where this could mean hours, or weeks, based on the author's tweeting frequency.
