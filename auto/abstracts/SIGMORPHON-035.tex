The paper describes the University of Melbourne's submission to the SIGMORPHON 2020 Shared Task 0: Typologically Diverse Morphological Inflection. Our team submitted three systems in total, two neural and one non-neural. Our analysis of systems' performance shows positive effects of newly introduced data hallucination technique that we employed in one of neural systems, especially in low-resource scenarios. A non-neural system based on observed inflection patterns shows optimistic results even in its simple implementation (>75\% accuracy for 50\% of languages). With possible improvement within the same modeling principle, accuracy might grow to values above 90\%.
