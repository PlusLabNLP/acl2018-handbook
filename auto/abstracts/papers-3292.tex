While national politics often receive the spotlight, the overwhelming majority of legislation proposed, discussed, and enacted is done at the state level. Despite this fact, there is little awareness of the dynamics that lead to adopting these policies. In this paper, we take the first step towards a better understanding of these processes and the underlying dynamics that shape them, using data-driven methods. We build a new large-scale dataset, from multiple data sources, connecting state bills and legislator information, geographical information about their districts, and donations and donors' information. We suggest a novel task, predicting the legislative body's vote breakdown for a given bill, according to different criteria of interest, such as gender, rural-urban and ideological splits.  Finally, we suggest a shared relational embedding model, representing the interactions between the text of the bill and the legislative context in which it is presented. Our experiments show that providing this context helps improve the prediction over strong text-based models.
