Update summarization is a form of multi-document summarization where a document set must be summarized in the context of other documents assumed to be known. Efficient update summarization must focus on identifying new information and avoiding repetition of known information. In Query-focused summarization, the task is to produce a summary as an answer to a given query.  We introduce a new task, Query-Chain Summarization, which combines aspects of the two previous tasks: starting from a given document set, increasingly specific queries are consid-ered, and a new summary is produced at each step. This process models exploratory search: a user explores a new topic by submitting a sequence of queries, inspecting a summary of the result set and phrasing a new query at each step. We present a novel dataset comprising 22 query-chains sessions of length 3 with 3 matching human summaries each in the consumer-health domain. Our analysis demonstrates that summaries produced in the context of such exploratory process are different from informative summaries. We present an algorithm for Query-Chain Summarization based on a new LDA topic model variant. Evaluation indicates the algorithm improves on strong baselines.
