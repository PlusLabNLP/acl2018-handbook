While pretrained models such as BERT have shown large gains across natural language understanding tasks, their performance can be improved by further training the model on a data-rich intermediate task, before fine-tuning it on a target task. However, it is still poorly understood when and why intermediate-task training is beneficial for a given target task. To investigate this, we perform a large-scale study on the pretrained RoBERTa model  with 110 intermediate-target task combinations. We further evaluate all trained models with 25 probing tasks meant to reveal the specific skills that drive transfer. We observe that intermediate tasks requiring high-level inference and reasoning abilities tend to work best. We also observe that target task performance is strongly correlated with higher-level abilities such as coreference resolution. However, we fail to observe more granular correlations between probing and target task performance, highlighting the need for further work on broad-coverage probing benchmarks. We also observe evidence that the forgetting of knowledge learned during pretraining may limit our analysis, highlighting the need for further work on transfer learning methods in these settings.
