Existing sentiment analysers are weak AI systems: they try to capture the functionality of human sentiment detection faculty, without worrying about how such faculty is realized in the hardware of the human. These analysers are agnostic of the actual cognitive processes involved. This, however, does not deliver when applications demand order of magnitude facelift in accuracy, as well as insight into characteristics of sentiment detection process. In this paper, we present a cognitive study of sentiment detection from the perspective of strong AI. We study the sentiment detection process of a set of human ''sentiment readers''. Using eye-tracking, we show that on the way to sentiment detection, humans first extract subjectivity. They focus attention on a subset of sentences before arriving at the overall sentiment. This they do either through ''anticipation'' where sentences are skipped during the first pass of reading, or through ''homing'' where a subset of the sentences are read over multiple passes, or through both. ''Homing'' behaviour is also observed at the sub-sentence level in complex sentiment phenomena like sarcasm.
