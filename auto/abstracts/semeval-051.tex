We present an approach for tackling the Sentiment Analysis problem in SemEval 2015. The approach is based on the use of a cooccurrence graph to represent existing relationships among terms in a document with the aim of using centrality measures to extract the most representative words that express the sentiment. These words are then used in a supervised learning algorithm as features to obtain the polarity of unknown documents. The best results obtained for the different datasets are: 77.76\% for positive, 100\% for negative and 68.04\% for neutral, showing that the proposed graph-based representation could be a way of extracting terms that are relevant to detect a sentiment.
