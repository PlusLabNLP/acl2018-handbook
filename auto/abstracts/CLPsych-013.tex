Title: Aided diagnosis of dementia type through computer-based analysis of spontaneous speech. OBJECTIVE: To evaluate the diagnostic accuracy of a machine-learning induced algorithm capable of classifying patients with frontotemporal lobar degeneration (FTLD), Alzheimer's Disease (AD) and healthy controls using acoustic and lexical features of spontaneous speech. METHOD: We analyzed recordings of a brief spontaneous speech sample from 48 subjects from 5 diagnostic categories: healthy controls, Alzheimer's Disease (AD), and three subtypes of Fronto-temporal Lobar Degeneration (FTLD): Semantic Dementia (SD), the behavioral variant of frontotemporal dementia (bvFTD), and progressive non-fluent aphasia (PNFA). Recordings were analyzed using a speech recognition system optimized for speaker-independent spontaneous speech.  Bag of word frequencies, pause and speech sound durations were then extracted.  The resulting feature profiles were used as input to a machine learning system that was trained to identify the diagnosis assigned to each research participant. We evaluated the diagnostic performance of this system using five-fold cross validation. RESULTS: Between groups lexical and acoustic differences in spontaneous speech features were detected in accordance with expectations from prior research literature. Machine learning algorithms were able to identify subjects' diagnostic category with accuracy mostly on par with existing diagnostic methods used by clinicians. For example, accuracy distinguishing between AD and FTLD was 88\%. CONCLUSIONS: This pilot study suggests that our clinical speech analytic approach can distinguish between controls and a number of dementia subtypes via differences in lexico-acoustic profiles of speech. Diagnostic accuracy usually on par with clinician accuracy                          despite speech sample brevity. This suggest the approach offers promise as an additional, objective and easily obtained source of screening information for clinical neurologists and neuropsychologists performing differential diagnosis.
