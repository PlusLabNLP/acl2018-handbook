Emotion-controllable response generation is an attractive and valuable task that aims to make open-domain conversations more empathetic and engaging. Existing methods mainly enhance the emotion expression by adding regularization terms to standard cross-entropy loss and thus influence the training process. However, due to the lack of further consideration of content consistency, the common problem of response generation tasks, safe response, is intensified. Besides, query emotions that can help model the relationship between query and response are simply ignored in previous models, which would further hurt the coherence. To alleviate these problems, we propose a novel framework named Curriculum Dual Learning (CDL) which extends the emotion-controllable response generation to a dual task to generate emotional responses and emotional queries alternatively. CDL utilizes two rewards focusing on emotion and content to improve the duality. Additionally, it applies curriculum learning to gradually generate high-quality responses based on the difficulties of expressing various emotions. Experimental results show that CDL significantly outperforms the baselines in terms of coherence, diversity, and relation to emotion factors.
