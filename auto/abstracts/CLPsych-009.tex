We use computational techniques to extract a large number of different features from the narrative speech of individuals with primary progressive aphasia (PPA). We examine several different types of features, including part-of-speech, complexity, context-free grammar, fluency, psycholinguistic, vocabulary richness, and acoustic, and discuss the circumstances under which they can be extracted. We consider the task of training a machine learning classifier to determine whether a participant is a control, or has the fluent or nonfluent variant of PPA.We first evaluate the individual feature sets on their classification accuracy, then perform an ablation study to determine the optimal combination of feature sets. Finally, we rank the features in four practical scenarios: given audio data only, given unsegmented transcripts only, given segmented transcripts only, and given both audio and segmented transcripts. We find that psycholinguistic features are highly discriminative in most cases, and that acoustic, context-free grammar, and part-of-speech features can also be important in some circumstances.
