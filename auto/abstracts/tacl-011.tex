This article discusses the requirements of a formal specification for the annotation of tem- poral information in clinical narratives. We discuss the implementation and extension of ISO-TimeML for annotating a corpus of clini- cal notes, known as the XXX corpus. To reflect the information task and the heavily inference- based reasoning demands in the domain, a new annotation guideline has been developed, ``the XXX Guidelines to ISO-TimeML (XXX- TimeML)''. To clarify what relations merit an- notation, we distinguish between linguistically- derived and inferentially-derived temporal or- derings in the text. We also apply a top per- forming TempEval 2013 system against this new resource to measure the difficulty of adapt- ing systems to the clinical domain. The corpus is available to the community and has been proposed for use in a SemEval 2015 task.
