Recent works have discussed the extent to which emergent languages can exhibit properties of natural languages particularly learning compositionality. In this paper, we investigate the learning biases that affect the efficacy and compositionality in multi-agent communication in addition to the communicative bandwidth. Our foremost contribution is to explore how the capacity of a neural network impacts its ability to learn a compositional language. We additionally introduce a set of evaluation metrics with which we analyze the learned languages. Our hypothesis is that there should be a specific range of model capacity and channel bandwidth that induces compositional structure in the resulting language and consequently encourages systematic generalization. While we empirically see evidence for the bottom of this range, we curiously do not find evidence for the top part of the range and believe that this is an open question for the community.
