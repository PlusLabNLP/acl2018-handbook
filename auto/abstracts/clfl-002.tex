In this paper we present a specific application of SPARSAR, a system for poetry analysis and TextToSpeech ``expressive reading''. We will focus on the graphical output organized at three macro levels, a Phonetic Relational View where phonetic and phonological features are highlighted; a Poetic Relational View that accounts for a poem rhyming and metrical structure; and a Semantic Relational View that shows semantic and pragmatic relations in the poem. We will also discuss how colours may be used appropriately to account for the overall underlying attitude expressed in the poem, whether directed to sadness or to happiness. This is done following traditional approaches which assume that the underlying feeling of a poem is strictly related to the sounds conveyed by the words besides their meaning. This will be shown using part of Shakespeare's Sonnets.
