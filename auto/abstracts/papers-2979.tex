Machine reading is an ambitious goal in NLP that subsumes a wide range of text understanding capabilities. Within this broad framework, we address the task of machine reading the time of historical events, compile datasets for the task, and develop a model for tackling it. Given a brief textual description of an event, we show that good performance can be achieved by extracting relevant sentences from Wikipedia, and applying a combination of task-specific and general-purpose feature embeddings for the classification. Furthermore, we establish a link between the historical event ordering task and the event focus time task from the information retrieval literature, showing they also provide a challenging test case for machine reading algorithms.
