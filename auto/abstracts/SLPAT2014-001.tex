Augmentative Alternative Communication (AAC) policy suffers from a lack of large scale quantitative evidence on the demographics of users and diversity of devices. The 2013 Domesday Dataset was created to aid formation of AAC policy at the national level. The dataset records  purchases of AAC technology by the UK's National Health Service between 2006 and 2012; giving information for each item on: make, model, price, year of purchase, and geographic area of purchase.  The dataset was designed to help answer open questions about the provision of AAC services in the UK; and the level of detail of the dataset is such that it can be used at the research level to provide context for researchers and to help validate (or not) assumptions about everyday AAC use. This paper examine three different ways of using the Domesday Dataset to provide verified evidence to support, or refute, assumptions, uncover important research problems, and to properly map the technological distinctiveness of a user community.
