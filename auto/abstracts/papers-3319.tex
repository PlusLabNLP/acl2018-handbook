In this work, we present a detailed analysis of how accent information is reflected in the internal representation of speech in an end-to-end automatic speech recognition (ASR) system. We use a state-of-the-art end-to-end ASR system, comprising convolutional and recurrent layers, that is trained on a large amount of US-accented English speech and evaluate the model on speech samples from seven different English accents. We examine the effects of accent on the internal  representation using three main probing techniques: a) Gradient-based explanation methods, b) Information-theoretic measures, and c) Outputs of accent and phone classifiers. We find different accents exhibiting similar trends irrespective of the probing technique used. We also find that most accent information is encoded within the first recurrent layer, which is suggestive of how one could adapt such an end-to-end model to learn representations that are invariant to accents.
