In conversation, speakers tend to echo the linguistic style of the person they are interacting with. This paper contributes to a body of work that addresses how this linguistic style coordination is affected by the social context in which the interaction occurs. In particular, we investigate the effect that an agent's social network centrality has on the coordination exhibited in replies to their utterances. We find that linguistic coordination is positively correlated with social network centrality and that this effect is greater than previous results showing a similar connection between status-based power and linguistic coordination. We conjecture that the social value of coordination may reside in the wish to conform to the linguistic norms of a community.
