In this paper, we propose the use of Message Sequence Charts (MSC) as a representation for visualizing narrative text in Hindi. An MSC is a formal representation allowing the depiction of actors and interactions among these actors in a scenario, apart from supporting a rich framework for formal inference. We propose an approach to extract MSC actors and interactions from a Hindi narrative. As a part of the approach, we enrich an existing event annotation scheme where we provide guidelines for annotation of the mood of events (realis vs irrealis) and guidelines for annotation of event arguments. We report performance on multiple evaluation criteria by experimenting with Hindi narratives from Indian History. Though Hindi is the fourth most-spoken first language in the world, from the NLP perspective it has comparatively lesser resources than English. Moreover, there is relatively less work in the context of event processing in Hindi. Hence, we believe that this work is among the initial works for Hindi event processing.
