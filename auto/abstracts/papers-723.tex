The Natural Language Understanding (NLU) component in task oriented dialog systems processes a user's request and converts it into structured information that can be consumed by downstream components such as the Dialog State Tracker (DST). This information is typically represented as a semantic frame that captures the intent and slot-labels provided by the user. We first show that such a shallow representation is insufficient for complex dialog scenarios, because it does not capture the recursive nature inherent in many domains. We propose a recursive, hierarchical frame-based representation and show how to learn it from data. We formulate the frame generation task as a template-based tree decoding task, where the decoder recursively generates a template and then fills slot values into the template. We extend local tree-based loss functions with terms that provide global supervision and show how to optimize them end-to-end. We achieve a small improvement on the widely used ATIS dataset and a much larger improvement on a more complex dataset we describe here.
