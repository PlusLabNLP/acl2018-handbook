In this paper, we describe the findings of the SIGMORPHON 2020 shared task on unsupervised morphological paradigm completion (SIGMORPHON 2020 Task 2), a novel task in the field of inflectional morphology. Participants were asked to submit systems which take raw text and a list of lemmas as input, and output all inflected forms, i.e., the entire morphological paradigm, of each lemma. In order to simulate a realistic use case, we first released data for 5 development languages. However, systems were officially evaluated on 9 surprise languages, which were only revealed a few days before the submission deadline. We provided a modular baseline system, which is a pipeline of 4 components. 3 teams submitted a total of 7 systems, but, surprisingly, none of the submitted systems was able to improve over the baseline on average over all 9 test languages. Only on 3 languages did a submitted system obtain the best results. This shows that unsupervised morphological paradigm completion is still largely unsolved. We present an analysis here, so that this shared task will ground further research on the topic.
