Translationese is  a phenomenon  present  in human translations, simultaneous interpreting, and  even  machine  translations.  Some  translationese  features  tend  to  appear in simultaneous interpreting with higher frequency than in human text translation, but the reasons for this  are  unclear.   This  study analyzes  translationese  patterns  in  translation,  interpreting, and  machine  translation  outputs in order  to explore possible reasons.  In our analysis we -- (i)  detail  two  non-invasive  ways  of detecting translationese and (ii) compare translationese across  human  and  machine  translations  from text and speech.  We find that machine translation shows traces of translationese, but does not  reproduce the  patterns  found  in  human translation, offering support to the hypothesis that such patterns are due to the model (human vs machine) rather than to the data (written vs spoken).
