The Tigrinya language is agglutinative and has a large number of inflected and derived forms of words. Therefore a Tigrinya large vocabulary continuous speech recognition system often has a large number of different units and a high out-of-vocabulary (OOV) rate if a word is used as a recognition unit of a language model (LM) and lexicon. Therefore a morpheme-based approach has often been used  and a morpheme is used as the recognition unit to reduce the high OOV rate. This paper presents an automatic speech recognition experiment conducted to see the effect of OOV words on the performance speech recognition system for Tigrinya. We tried to solve the OOV problem by using morphemes as lexicon and language model units. It has been found that the morpheme-based recognition system is better lexical and language modeling units than words. An absolute improvement (in word recognition accuracy) of 3.45 token and 8.36 types has been obtained as a result of using a morph-based vocabulary.
