Argumentation mining obviously involves finding support relations between statements, but many interesting instances of argumentation also contain counter-considerations, which the author mentions in order to preempt possible objections by the readers. A counter-consideration in monologue text thus involves a switch of perspective toward an imaginary „opponent``. We present a classification approach to classifying counter-considerations and apply it to two different corpora: a selection of very short argumentative texts produced in a text generation experiment, and a set of newspaper commentaries. As expected, the latter pose more difficulties, which we investigate in a brief error anaylsis.
