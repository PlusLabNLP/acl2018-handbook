In this paper, we show that neural machine translation (NMT) systems trained on large back-translated data overfit some of the characteristics of machine-translated texts. Such NMT systems better translate human-produced translations, i.e., translationese, but may largely worsen the translation quality of original texts. Our analysis reveals that adding a simple tag to back-translations prevents this quality degradation and improves on average the overall translation quality by helping the NMT system to distinguish back-translated data from original parallel data during training. We also show that, in contrast to high-resource configurations, NMT systems trained in low-resource settings are much less vulnerable to overfit back-translations. We conclude that the back-translations in the training data should always be tagged especially when the origin of the text to be translated is unknown.
