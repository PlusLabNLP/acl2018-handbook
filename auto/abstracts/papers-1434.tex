In argumentation, people state premises to reason towards a conclusion. The conclusion conveys a stance towards some target, such as a concept or statement. Often, the conclusion remains implicit, though, since it is self-evident in a discussion or left out for rhetorical reasons. However, the conclusion is key to understanding an argument and, hence, to any application that processes argumentation. We thus study the question to what extent an argument's conclusion can be reconstructed from its premises. In particular, we argue here that a decisive step is to infer a conclusion's target, and we hypothesize that this target is related to the premises' targets. We develop two complementary target inference approaches: one ranks premise targets and selects the top-ranked target as the conclusion target, the other finds a new conclusion target in a learned embedding space using a triplet neural network. Our evaluation on corpora from two domains indicates that a hybrid of both approaches is best, outperforming several strong baselines. According to human annotators, we infer a reasonably adequate conclusion target in 89\% of the cases.
