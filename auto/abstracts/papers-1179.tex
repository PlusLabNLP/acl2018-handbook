Unsupervised machine translation (MT) has recently achieved impressive results with monolingual corpora only. However, it is still challenging to associate source-target sentences in the latent space. As people speak different languages biologically share similar visual systems, the potential of achieving better alignment through visual content is promising yet under-explored in unsupervised multimodal MT (MMT). In this paper, we investigate how to utilize visual content for disambiguation and promoting latent space alignment in unsupervised MMT. Our model employs multimodal back-translation and features pseudo visual pivoting in which we learn a shared multilingual visual-semantic embedding space and incorporate visually-pivoted captioning as additional weak supervision. The experimental results on the widely used Multi30K dataset show that the proposed model significantly improves over the state-of-the-art methods and generalizes well when images are not available at the testing time.
