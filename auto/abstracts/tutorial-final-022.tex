The FrameNet lexical database (Fillmore and Baker 2010) covers roughly 13,000 lexical units (word senses) for the core Engish lexicon, associating them with roughly 1,200 fully defined semantic frames; these frames and their roles cover the majority of event types in everyday, non-specialist text, and they are documented with 200,000 manually annotated examples. Although much of the NLP community has heard of FrameNet and has a rough picture of what it is, to date, only a few NLP researchers have fully explored the wealth of information that FrameNet offers and how it differs from other resources such as PropBank. In this tutorial, we cover recent developments like the new NLTK Python API that make it easy to get started with FrameNet. This tutorial will teach attendees what they need to know to start using the FrameNet lexical database as part of an NLP system. We will cover the basics of Frame Semantics, explain how the database has been created and introduce the Python API and the state of the art in automatic frame semantic role labeling systems, and FrameNet collaboration with commercial partners. As time permits, we will also discuss new research on frame semantic approaches to negation and conditionals and the how frame semantic roles can aid the interpretation of metaphors.
