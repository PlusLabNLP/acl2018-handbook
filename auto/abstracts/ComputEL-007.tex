LingSync and the Online Linguistic Database (OLD) are new models for the collection and management of data in endangered language settings. The LingSync and OLD projects seek to close a feedback loop between field linguists, language communities, software developers, and computational linguists by creating web services and user interfaces which facilitate collaborative and inclusive language documentation. This paper presents the architectures of these tools and the resources generated thus far. We also briefly discuss some of the features of the systems which are particularly helpful to endangered languages fieldwork and which should also be of interest to computational linguists, these being a service that automates the identification of utterances within audio/video, another that automates the alignment of audio recordings and transcriptions, and a number of services that automate the morphological parsing task. The paper discusses the requirements of software used for endangered language documentation, and presents novel data which demonstrates that users are actively seeking alternatives despite existing software.
