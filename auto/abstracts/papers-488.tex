Random walks over large knowledge bases like WordNet have been successfully used in word similarity, relatedness and disambiguation tasks. Unfortunately, those algorithms are relatively slow for large repositories, with significant memory footprints. In this paper we present a novel algorithm which encodes the structure of a knowledge base in a continuous vector space, combining random walks and neural net language models in order to produce novel word representations. Evaluation in word relatedness and similarity datasets yields equal or better results than those of a random walk algorithm, using a dense representation (300 dimensions instead of 117K). Furthermore, the word representations are complementary to those of the random walk algorithm and to corpus-based continuous representations, improving the state-of-the-art in the similarity dataset. Our technique opens up exciting opportunities to combine distributional and knowledge-based word representations.
