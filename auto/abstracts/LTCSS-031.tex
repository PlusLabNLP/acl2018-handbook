Louis Armstrong (is said to have) said, ``I don't need words --- it's all in the phrasing''.                    As someone who does natural-language processing for a living, I'm a big fan of words; but lately, my collaborators and I have been studying aspects of phrasing (in the linguistic, rather than musical sense) that go beyond just the selection of one particular word over another.  I'll describe some of these projects in this talk. The issues we'll consider include: Does the way in which something is worded in and of itself have an effect on whether it is remembered or attracts attention, beyond its content or context? Can we characterize how different sides in a debate frame their arguments, in a way that goes beyond specific lexical choice (e.g., ``pro-choice'' vs. ``pro-life'')?  The settings we'll explore range from movie quotes that achieve cultural prominence; to posts on Facebook, Wikipedia, Twitter, and the arXiv; to framing in public discourse on the inclusion of genetically-modified organisms in food. Joint work with Lars Backstrom, Justin Cheng, Eunsol Choi, Cristian Danescu-Niculescu-Mizil, Jon Kleinberg, Bo Pang, Jennifer Spindel, and Chenhao Tan.
