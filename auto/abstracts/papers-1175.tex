Language exhibits hierarchical structure, but recent work using a subject-verb agreement diagnostic argued that state-of-the-art language models, LSTMs, fail to learn long-range syntax sensitive dependencies. Using the same diagnostic, we show that, in fact, LSTMs do succeed in learning such dependencies—provided they have enough capacity. We then explore whether models that have access to explicit syntactic information learn agreement more effectively, and how the way in which this structural information is incorporated into the model impacts performance. We find that the mere presence of syntactic information does not improve accuracy, but when model architecture is determined by syntax, number agreement is improved. Further, we find that the choice of how syntactic structure is built affects how well number agreement is learned: top-down construction outperforms left-corner and bottom-up variants in capturing non-local structural dependencies.
