This paper describes a local effort to bridge the gap between computational and documentary linguistics by teaching students and young researchers in computational linguistics about doing research and developing systems for low-resource languages. We describe four student software projects developed within one semester. The projects range from a front-end for building small-vocabulary speech recognition systems, to a broad-coverage (more than 1000 languages) language identification system, to language-specific systems: a lemmatizer for the Mayan language Uspanteko and named entity recognition systems for both Slovak and Persian. Teaching efforts such as these are an excellent way to develop not only tools for low-resource languages, but also computational linguists well-equipped to work on endangered and low-resource languages.
