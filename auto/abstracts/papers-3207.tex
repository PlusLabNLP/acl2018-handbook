Neural networks are surprisingly good at interpolating and perform remarkably well when the training set examples resemble those in the test set. However, they are often unable to extrapolate patterns beyond the seen data, even when the abstractions required for such patterns are simple. In this paper, we first review the notion of extrapolation, why it is important and how one could hope to tackle it. We then focus on a specific type of extrapolation which is especially useful for natural language processing: generalization to sequences that are longer than the training ones. We hypothesize that models with a separate content- and location-based attention are more likely to extrapolate than those with common attention mechanisms. We empirically support our claim for recurrent seq2seq models with our proposed attention on variants of the Lookup Table task. This sheds light on some striking failures of neural models for sequences and on possible methods to approaching such issues.
