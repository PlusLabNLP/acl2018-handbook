Automatic metrics are fundamental for the development and evaluation of machine translation systems. Judging whether, and to what extent, automatic metrics concur with the gold standard of human evaluation is not a straightforward problem. We show that current methods for judging metrics are highly sensitive to the translations used for assessment, particularly the presence of outliers, which often leads to falsely confident conclusions about a metric's efficacy. Finally, we turn to pairwise system ranking, developing a method for thresholding performance improvement under an automatic metric against human judgements, which allows quantification of type I versus type II errors incurred, i.e., insignificant human differences in system quality that are accepted, and significant human differences that are rejected. Together, these findings suggest improvements to the protocols for metric evaluation and system performance evaluation in machine translation.
