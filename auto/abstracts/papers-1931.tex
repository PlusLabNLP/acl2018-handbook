Language models pretrained on text from a wide variety of sources form the foundation of today's NLP. In light of the success of these broad-coverage models, we investigate whether it is still helpful to tailor a pretrained model to the domain of a target task. We present a study across four domains (biomedical and computer science publications, news, and reviews) and eight classification tasks, showing that a second phase of pretraining in-domain (domain-adaptive pretraining) leads to performance gains, under both high- and low-resource settings. Moreover, adapting to the task's unlabeled data (task-adaptive pretraining) improves performance even after domain-adaptive pretraining. Finally, we show that adapting to a task corpus augmented using simple data selection strategies is an effective alternative, especially when resources for domain-adaptive pretraining might be unavailable. Overall, we consistently find that multi-phase adaptive pretraining offers large gains in task performance.
