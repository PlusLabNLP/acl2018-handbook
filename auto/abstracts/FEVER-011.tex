The alarming spread of fake news in social media, together with the impossibility of scaling manual fact verification, motivated the development of natural language processing techniques to automatically verify the veracity of claims. Most approaches perform a claim-evidence classification without providing any insights about why the claim is trustworthy or not. We propose, instead, a model-agnostic framework that consists of two modules: (1) a span extractor, which identifies the crucial information connecting claim and evidence; and (2) a classifier that combines claim, evidence, and the extracted spans to predict the veracity of the claim. We show that the spans are informative for the classifier, improving performance and robustness. Tested on several state-of-the-art models over the Fever dataset, the enhanced classifiers consistently achieve higher accuracy while also showing reduced sensitivity to artifacts in the claims.
