Modern automated essay scoring systems rely on identifying linguistically-relevant features to estimate essay quality. This paper attempts to bridge work in psycholinguistics and natural language processing by proposing sentence processing complexity as a feature for automated essay scoring, in the context of English as a Foreign Language (EFL). To quantify processing complexity we used a psycholinguistic model called surprisal theory. First, we investigated whether essays' average surprisal values decrease with EFL training. Preliminary results seem to support this idea. Second, we investigated whether surprisal can be effective as a predictor of essay quality. The results indicate an inverse correlation between surprisal and essay scores. Overall, the results are promising and warrant further investigation on the usability of surprisal for essay scoring.
