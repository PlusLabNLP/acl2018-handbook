The Kamusi Project, a multilingual online dictionary website, has as one of its goals to document the lexicons of endangered and less-resourced languages (LRLs). Kamusi.org provides a unified platform and repository for this kind of data that is both simple to use and free to researchers and the public. Since Kamusi has a separate entry for each homophone or polyseme, it can be used to produce sophisticated multilingual dictionaries. We have recently been confronting issues inherent in contact language-based lexicography, especially the elicitation of culturally-specific semantic terms, which cannot be obtained through fieldwork purely reliant on a contact language. To address this, we have designed a system of ``balloons.'' Based on a variety of factors, balloons raise the likelihood of revealing terms and fields that have particular relevance within a culture, rather than perpetuating linguistic bias toward the concerns and artifacts of more powerful groups. Kamusi has also developed a smartphone application which can be used for crowdsourcing contributions and validation.  It will also be invaluable in gathering oral data from speakers of endangered languages for the production of monolingual talking dictionaries. The first of these projects is planned for the Arrernte language in central Australia.
