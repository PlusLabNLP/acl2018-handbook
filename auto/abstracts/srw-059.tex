The aim of this thesis proposal is to perform bilingual lexicon extraction for cases in which small parallel corpora are available and it is not easy to obtain monolingual corpus for at least one of the languages. Moreover, the languages are typologically distant and there is no bilingual seed lexicon available. We focus on the language pair Spanish-Nahuatl, we propose to work with morpheme based representations in order to reduce the sparseness and to facilitate the task of finding lexical correspondences between a highly agglutinative language and a fusional one. We take into account contextual information but instead of using a precompiled seed dictionary, we use the distribution and dispersion of the positions of the morphological units as cues to compare the contextual vectors and obtaining the translation candidates.
