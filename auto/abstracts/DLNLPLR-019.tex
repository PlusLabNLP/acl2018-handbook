Neural part-of-speech (POS) taggers are known to not perform well with little training data. As a step towards overcoming this problem, we present an architecture for learning more robust neural POS taggers by jointly training a hierarchical, recurrent model and a recurrent character-based sequence-to-sequence network supervised using an auxiliary objective. This way, we introduce stronger character-level supervision into the model, which enables better generalization to unseen words and provides regularization, making our encoding less prone to overfitting. We experiment with three auxiliary tasks: lemmatization, character-based word autoencoding, and character-based random string autoencoding. Experiments with minimal amounts of labeled data on 34 languages show that our new architecture outperforms a single-task baseline and, sur- prisingly, that, on average, raw text autoencoding can be as beneficial for low-resource POS tagging as using lemma information. Our neural POS tagger closes the gap to a state-of-the-art POS tagger (MarMoT) for low-resource scenarios by 43\%, even outperforming it on languages with templatic morphology, e.g., Arabic, Hebrew, and Turkish, by some margin.
