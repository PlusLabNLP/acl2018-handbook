Recurrent neural networks (RNNs) reached striking performance in many natural language processing tasks. This has renewed interest in whether these generic sequence processing devices are inducing genuine linguistic knowledge. Nearly all current analytical studies, however, initialize the RNNs with a vocabulary of known words, and feed them tokenized input during training. We present a multi-lingual study of the linguistic knowledge encoded in RNNs trained as character-level language models, on input data with word boundaries removed. These networks face a tougher and more cognitively realistic task, having to discover and store any useful linguistic unit from scratch, based on input statistics. The results show that our "near tabula rasa" RNNs are mostly able to solve morphological, syntactic and semantic tasks that intuitively presuppose word-level knowledge, and indeed they learned to track "soft" word boundaries. Our study opens the door to speculations about the necessity of an explicit word lexicon in language learning and usage.