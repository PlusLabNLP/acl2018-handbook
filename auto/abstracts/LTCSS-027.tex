Framing---portraying an issue from one perspective to the necessary exclusion of alternative perspectives---is a central concept in political communication. It is also a powerful political tool, as evidenced through experiments and single-issue studies beyond the lab. Yet compared to its significance, we know very little about framing as a generalizable phenomenon. Do framing dynamics, such as the evolution of one frame into another, play out the same way for all issues? Under what conditions does framing influence public opinion and policy? Understanding the general patterns of framing dynamics and effects is thus hugely important. It is also a serious challenge, thanks to the volume of text data, the dynamic nature of language, and variance in applicable frames across issues (e.g., the `innocence' frame of the death penalty debate is irrelevant for discussing smoking bans). To address this challenge, I describe a collaborative project with Justin Gross, Philip Resnik, and Noah Smith. We advance a unified policy frames codebook, in which issue-specific frames (e.g., innocence) are nested within high-level categories of frames (e.g., fairness) that cross cut issues. Through manual annotation bolstered by supervised learning, we can track the relative use of different frame cues within a given issue over time and in an apples-to-apples way across issues. Preliminary findings suggest our work may help unlock the black box of framing, pointing to generalizable conditions under which we should expect to see different types of framing dynamics and framing effects.
