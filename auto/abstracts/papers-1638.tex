The process of obtaining high quality labeled data for natural language understanding tasks is often slow, error-prone, complicated and expensive. With the vast usage of neural networks, this issue becomes more notorious since these networks require a large amount of labeled data to produce satisfactory results. We propose a methodology to blend high quality but scarce strong labeled data with noisy but abundant weak labeled data during the training of neural networks. Experiments in the context of topic-dependent evidence detection with two forms of weak labeled data show the advantages of the blending scheme. In addition, we provide a manually annotated data set for the task of topic-dependent evidence detection. We believe that blending weak and strong labeled data is a general notion that may be applicable to many language understanding tasks, and can especially assist researchers who wish to train a network but have a small amount of high quality labeled data for their task of interest.
