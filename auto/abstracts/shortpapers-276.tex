Recent work classifying citations in scientific literature has shown that it is possible to improve classification results with extensive feature engineering.  While this result confirms that citation classification is feasible, there are two drawbacks to this approach: (i) it requires a large annotated corpus for supervised classification, which in the case of scientific literature is quite expensive; and (ii) feature engineering that is too specific to one area of scientific literature may not be portable to other domains, even within scientific literature.  In this paper we address these two drawbacks.                    First, we frame citation classification as a domain adaptation task and leverage the abundant labeled data available in other domains.                    Then, to avoid over-engineering specific citation features for a particular scientific domain, we explore a deep learning neural network approach that has shown to generalize well across domains using unigram and bigram features.  We achieve better citation classification results with this cross-domain approach than using in-domain classification.
