Idioms and metaphors are characteristic to all areas of human activity and to all types of discourse. Their processing is a rapidly growing area in NLP, since they have become a big challenge for NLP systems. Their omnipresence in language has been established in a number of corpus studies and the role they play in human reasoning has also been confirmed in psychological experiments. This makes idioms and metaphors an important research area for computational and cognitive linguistics, and their automatic identification and interpretation indispensable for any semantics-oriented NLP application. This tutorial aims to provide attendees with a clear notion of the linguistic characteristics of idioms and metaphors, computational models of idioms and metaphors using state-of-the-art NLP techniques, their relevance for the intersection of deep learning and natural language processing, what methods and resources are available to support their use, and what more could be done in the future. Our target audience are researchers and practitioners in machine learning, parsing (syntactic and semantic) and language technology, not necessarily experts in idioms and metaphors, who are interested in tasks that involve or could benefit from considering idioms and metaphors as a pervasive phenomenon in human language and communication.
