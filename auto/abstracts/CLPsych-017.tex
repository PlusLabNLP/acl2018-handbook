Children with autism spectrum disorder often exhibit idiosyncratic patterns of behaviors and interests. In this paper, we focus on measuring the presence of idiosyncratic interests at the linguistic level in children with autism using distributional semantic models. We model the semantic space of children's narratives by calculating pairwise word overlap, and we compare the overlap found within and across diagnostic groups. We find that the words used by children with typical development tend to be used by other children with typical development, while the words used by children with autism overlap less with those used by children with typical development and even less with those used by other children with autism. These findings suggest that children with autism are veering not only away from the topic of the target narrative but also in idiosyncratic semantic directions potentially defined by their individual topics of interest.
