While the effect of various lexical, syntactic, semantic and stylistic features have been addressed in persuasive language from a computational point of view, the persuasive effect of phonetics has received little attention. By modeling a notion of euphony and analyzing four datasets comprising persuasive and non-persuasive sentences in different domains (political speeches, movie quotes, slogans and tweets), we explore the impact of sounds on different forms of persuasiveness. We conduct a series of analyses and prediction experiments within and across datasets. Our results highlight the positive role of phonetic devices on persuasion.
