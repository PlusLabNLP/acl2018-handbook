Recent humor classification shared tasks have struggled with two issues: either the data comprises a highly constrained genre of humor which does not broadly represent humor, or the data is so indiscriminate that the inter-annotator agreement on its humor content is drastically low. These tasks typically average over all annotators' judgments, in spite of the fact that humor is a highly subjective phenomenon. We argue that demographic factors influence whether a text is perceived as humorous or not. We propose the addition of demographic information about the humor annotators in order to bin ratings more sensibly. We also suggest the addition of an 'offensive' label to distinguish between different generations, in terms of humor. This would allow for more nuanced shared tasks and could lead to better performance on downstream tasks, such as content moderation.
