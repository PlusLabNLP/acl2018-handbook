Computational social science has been emerging over the last several years as a hotbed of interesting work, taking advantage of, to quote Lazer et al. (Science, v.323), ``digital traces that can be compiled into comprehensive pictures of both individual and group behavior, with the potential to transform our understanding of our lives, organizations, and societies.'' Within that larger setting, I'm interested in how language is used to influence people, with an emphasis on computational modeling of agendas (who is most effectively directing attention, and toward what topics?), framing or ``spin'' (what underlying perspective does this language seek to encourage?), and sentiment (how does someone feel, as evidenced in the language they use)?  These questions are particularly salient in political discourse.   In this talk, I'll present recent work looking at political debates and other conversations using Bayesian models to capture relevant aspects of the conversational dynamics, as well as new methods for collecting people's reactions to speeches, debates, and other public conversations on a large scale. This talk includes work done in collaboration with Jordan Boyd-Graber, Viet-An Nguyen, Deborah Cai, Amber Boydstun, Rebecca Glazier, Matthew Pietryka, Tim Jurka, and Kris Miler.
