Recent work has shown that neural rerankers can improve results for dependency parsing over the top k trees produced by a base parser. However, all neural rerankers so far have been evaluated on English and Chinese only, both languages with a configurational word order and poor morphology. In the paper, we re-assess the potential of successful neural reranking models from the literature on English and on two morphologically rich(er) languages, German and Czech. In addition, we introduce a new variation of a discriminative reranker based on graph convolutional networks (GCNs). We show that the GCN not only outperforms previous models on English but is the only model that is  able to improve results over the baselines on German and Czech. We explain the differences in reranking performance based on an analysis of a) the gold tree ratio and b) the variety in the k-best lists.
