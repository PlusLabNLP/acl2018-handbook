The task of graph-to-text generation aims at producing sentences that preserve the meaning of input graphs. As a crucial defect, the current state-of-the-art models may mess up or even drop the core structural information of input graphs when generating outputs. We propose to tackle this problem by leveraging richer training signals that can guide our model for preserving input information. In particular, we introduce two types of autoencoding losses, each individually focusing on different aspects (a.k.a. views) of input graphs. The losses are then back-propagated to better calibrate our model via multi-task training. Experiments on two benchmarks for graph-to-text generation show the effectiveness of our approach over a state-of-the-art baseline.
