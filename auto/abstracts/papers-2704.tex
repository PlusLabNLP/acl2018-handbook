As part of growing NLP capabilities, coupled with an awareness of the ethical dimensions of research, questions have been raised about whether particular datasets and tasks should be deemed off-limits for NLP research. We examine this question with respect to a paper on automatic legal sentencing from EMNLP 2019 which was a source of some debate, in asking whether the paper should have been allowed to be published, who should have been charged with making such a decision, and on what basis. We focus in particular on the role of data statements in ethically assessing research, but also discuss the topic of dual use, and examine the outcomes of similar debates in other scientific disciplines.
