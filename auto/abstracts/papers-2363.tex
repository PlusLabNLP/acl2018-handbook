In this paper, we observe that semi-structured tabulated text is ubiquitous; understanding them requires not only comprehending the meaning of text fragments, but also implicit relationships between them. We argue that such data can prove as a testing ground for understanding how we reason about information. To study this, we introduce a new dataset called INFOTABS, comprising of human-written textual hypotheses based on premises that are tables extracted from Wikipedia info-boxes. Our analysis shows that the semi-structured, multi-domain and heterogeneous nature of the premises admits complex, multi-faceted reasoning. Experiments reveal that, while human annotators agree on the relationships between a table-hypothesis pair, several standard modeling strategies are unsuccessful at the task, suggesting that reasoning about tables can pose a difficult modeling challenge.
