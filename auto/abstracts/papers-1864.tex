In a context of offensive content mediation on social media now regulated by European laws,  it is important not only to be able to automatically detect sexist content but also to identify if a message with a sexist content is really sexist or is a story of sexism experienced by a woman. We propose: (1) a new characterization of sexist content inspired by speech acts theory and discourse  analysis  studies, (2) the  first  French  dataset  annotated for sexism detection, and (3) a set of deep learning experiments  trained on top of a combination of several tweet's vectorial representations (word embeddings, linguistic features, and various generalization strategies). Our results are encouraging and constitute a first step towards offensive content moderation.
