Research on automatically geolocating social media users has conventionally been based on the text content of posts from a given user or the social network of the user, with very little crossover between the two, and no benchmarking of the two approaches over comparable datasets. We bring the two threads of research together in first proposing a text-based method based on adaptive grids, followed by a hybrid network- and text-based method. Evaluating over three Twitter datasets, we show that the empirical difference between text- and network-based methods is not great, and that hybridisation of the two is superior to the component methods, especially in contexts where the user graph is not well connected. We achieve state-of-the-art results on all three datasets.
