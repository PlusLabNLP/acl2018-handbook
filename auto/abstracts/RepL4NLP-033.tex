In this paper, we investigate the effects of using subword information in representa- tion learning. We argue that using syntactic subword units effects the quality of the word representations positively. We introduce a morpheme-based model and compare it against to word-based, character-based, and character n-gram level models. Our model takes a list of candidate segmentations of a word and learns the representation of the word based on different segmentations that are weighted by an attention mechanism. We performed experiments on Turkish as a morphologically rich language and English with a comparably poorer morphology. The results show that morpheme-based models are better at learning word representations of morphologically complex languages compared to character-based and character n-gram level models since the morphemes help to incorporate more syntactic knowledge in learning, that makes morpheme-based models better at syntactic tasks.
