The task of automatic misogyny identification and categorization has not received as much attention as other natural language tasks have, even though it is crucial for identifying hate speech in social Internet interactions. In this work, we address this sentence classification task from a representation learning perspective, using both a bidirectional LSTM and BERT optimized with the following metric learning loss functions: contrastive loss, triplet loss, center loss, congenerous cosine loss and additive angular margin loss. We set new state-of-the-art for the task with our fine-tuned BERT, whose sentence embeddings can be compared with a simple cosine distance, and we release all our code as open source for easy reproducibility. Moreover, we find that almost every loss function performs equally well in this setting, matching the regular cross entropy loss.
