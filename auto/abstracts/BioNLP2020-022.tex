When comparing entities extracted by a medical entity recognition system with gold standard annotations over a test set, two types of mismatches might occur, label mismatch or span mismatch. Here we focus on span mismatch and show that its severity can vary from a serious error to a fully acceptable entity extraction due to the subjectivity of span annotations. For a domain-specific BERT-based NER system, we showed that 25\% of the errors have the same labels and overlapping span with gold standard entities. We collected expert judgement which shows more than 90\% of these mismatches are accepted or partially accepted by the user. Using the training set of the NER system, we built a fast and lightweight entity classifier to approximate the user experience of such mismatches through accepting or rejecting them. The decisions made by this classifier are used to calculate a learning-based F-score which is shown to be a better approximation of a forgiving user's experience than the relaxed F-score. We demonstrated the results of applying the proposed evaluation metric for a variety of deep learning medical entity recognition models trained with two datasets.
