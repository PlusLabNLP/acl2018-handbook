Discourse relation classification is an important component for automatic discourse parsing and natural language understanding. The performance bottleneck of a discourse parser comes from implicit discourse relations, whose discourse connectives are not overtly present. Explicit discourse connectives can potentially be exploited to collect more training data to collect more data and boost the performance. However, using them indiscriminately has been shown to hurt the performance because not all discourse connectives can be dropped arbitrarily. Based on this insight, we investigate the interaction between discourse connectives and the discourse relations and propose the criteria for selecting the discourse connectives that can be dropped independently of the context without changing the interpretation of the discourse. Extra training data collected only by the freely omissible connectives improve the performance of the system without additional features.
