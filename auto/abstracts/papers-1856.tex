This position paper describes and critiques the Pretraining-Agnostic Identically Distributed (PAID) evaluation paradigm, which has become a central tool for measuring progress in natural language understanding. This paradigm consists of three stages: (1) pre-training of a word prediction model on a corpus of arbitrary size; (2) fine-tuning (transfer learning) on a training set representing a classification task; (3) evaluation on a test set drawn from the same distribution as that training set. This paradigm favors simple, low-bias architectures, which, first, can be scaled to process vast amounts of data, and second, can capture the fine-grained statistical properties of a particular data set, regardless of whether those properties are likely to generalize to examples of the task outside the data set. This contrasts with humans, who learn language from several orders of magnitude less data than the systems favored by this evaluation paradigm, and generalize to new tasks in a consistent way. We advocate for supplementing or replacing PAID with paradigms that reward architectures that generalize as quickly and robustly as humans.
