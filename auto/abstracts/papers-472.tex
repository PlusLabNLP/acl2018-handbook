It has been exactly a decade since the first establishment of  SPMRL, a  research initiative unifying  multiple research efforts to  address the peculiar challenges of Statistical Parsing for Morphologically-Rich Languages (MRLs). Here we reflect on parsing MRLs in that decade,  highlight the solutions and lessons  learned for the architectural, modeling and lexical challenges  in the pre-neural era, and argue that similar challenges  re-emerge  in  neural architectures for MRLs. We then aim to offer a climax, suggesting that  incorporating symbolic ideas proposed in  SPMRL terms into nowadays neural architectures has the potential to push NLP for MRLs to a new level. We  sketch  a strategies for designing Neural Models for MRLs (NMRL), and showcase preliminary  support for these strategies via  investigating the task of multi-tagging in  Hebrew, a morphologically-rich,  high-fusion, language.
