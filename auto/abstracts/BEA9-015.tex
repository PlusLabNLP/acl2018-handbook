The paper offers an effective way of teacher-student computer-based collaboration in translation class. We show how a quantitative-qualitative method of analysis supported by word alignment technology can be applied to student translations for use in the classroom. The combined use of natural-language processing and manual techniques enables students to ‘co-emerge' during highly motivated collaborative sessions. Within the advocated approach, students are pro-active seekers for a better translation (grade) in a teacher-centered computer-based peer-assisted translation class.
