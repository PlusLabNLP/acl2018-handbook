In this paper, we present a fine-grained opinion mining dataset called SentiTel. SentiTel is human annotated for targeted aspect-based sentiment analysis (TABSA). SentiTel contains Twitter reviews about three major Ugandan telecoms posted in the period between February 2019 and September 2019. The dataset contains reviews that are in English or have a codemix of English and Luganda. SentiTel contains 6\,320 reviews that are annotated with the target telecom, aspect and sentiment towards the aspect of the target telecom. The reviews contain at least one target telecom. We present two models on the TABSA task; random forest (RF) which is the baseline model and the BERT based model. The best result is presented by the BERT model with an AUC of 0.950 and 0.965 on the aspect category detection task and sentiment classification task respectively. The results show that a great performance can be obtained on a downstream task by fine-tuning the pre-trained BERT model. Finally, the results also confirm that fine-grained information can be extracted from the short and unstructured text from Twitter with limited cues.
