Clinical coding is currently a labour-intensive, error-prone, but a critical administrative process whereby hospital patient episodes are manually assigned codes by qualified staff from large, standardised taxonomic hierarchies of codes. Automating clinical coding has a long history in NLP research and has recently seen novel developments setting new benchmark results. A popular dataset used in this task is MIMIC-III, a large database of clinical free text notes and their associated codes amongst other data. We argue for the reconsideration of the validity MIMIC-III's assigned codes, as MIMIC-III has not undergone secondary validation. This work presents an open-source, reproducible experimental methodology for assessing the validity of EHR discharge summaries. We exemplify the methodology with MIMIC-III discharge summaries and show the most frequently assigned codes in MIMIC-III are undercoded up to 35\%.
