As natural language processing methods are increasingly deployed in real-world scenarios such as healthcare, legal systems, and social science, it becomes necessary to recognize the role they potentially play in shaping social biases and stereotypes. Previous work has revealed the presence of social biases in widely used word embeddings involving gender, race, religion, and other social constructs. While some methods were proposed to debias these word-level embeddings, there is a need to perform debiasing at the sentence-level given the recent shift towards new contextualized sentence representations such as ELMo and BERT. In this paper, we investigate the presence of social biases in sentence-level representations and propose a new method, Sent-Debias, to reduce these biases. We show that Sent-Debias is effective in removing biases, and at the same time, preserves performance on sentence-level downstream tasks such as sentiment analysis, linguistic acceptability, and natural language understanding. We hope that our work will inspire future research on characterizing and removing social biases from widely adopted sentence representations for fairer NLP.
