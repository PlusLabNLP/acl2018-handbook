This paper addresses the problem of automatic recognition of emotions in conversational text datasets for the EmotionX challenge. Emotion is a human characteristic expressed through several modalities (e.g., auditory, visual, tactile). Trying to detect emotions only from the text becomes a difficult task even for humans. This paper evaluates several neural architectures based on Attention Models, which allow extracting relevant parts of the context within a conversation to identify the emotion associated with each utterance. Empirical results in the validation datasets demonstrate the effectiveness of the approach compared to the reference models for some instances, and other cases show better results with simpler models.
