We investigate the behavior of maps learned by machine translation methods. The maps translate words by projecting between word embedding spaces of different languages. We locally approximate these maps using linear maps, and find that they vary across the word embedding space. This demonstrates that the underlying maps are non-linear. Importantly, we show that the locally linear maps vary by an amount that is tightly correlated with the distance between the neighborhoods on which they are trained. Our results can be used to test non-linear methods, and to drive the design of more accurate maps for word translation.
