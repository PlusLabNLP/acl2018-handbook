This article contributes to the ongoing discussion in the computational linguistics community regarding instances that are difficult to annotate reliably. Is it worthwhile to identify those? What information can be inferred from them regarding the nature of the task? What should be done with them when building supervised machine learning systems? We address these questions in the context of a subjective semantic task. In this setting, we show that the presence of such instances in training data misleads a machine learner into misclassifying clear-cut cases.  We also show that considering machine learning outcomes with and without the difficult cases, it is possible to identify specific weaknesses of the problem representation.
