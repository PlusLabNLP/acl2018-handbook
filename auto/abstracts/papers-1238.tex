Keyphrase generation (KG) aims to summarize the main ideas of a document into a set of keyphrases. A new setting is recently introduced into this problem, in which, given a document, the model needs to predict a set of keyphrases and simultaneously determine the appropriate number of keyphrases to produce. Previous work in this setting employs a sequential decoding process to generate keyphrases. However, such a decoding method ignores the intrinsic hierarchical compositionality existing in the keyphrase set of a document. Moreover, previous work tends to generate duplicated keyphrases, which wastes time and computing resources. To overcome these limitations, we propose an exclusive hierarchical decoding framework that includes a hierarchical decoding process and either a soft or a hard exclusion mechanism. The hierarchical decoding process is to explicitly model the hierarchical compositionality of a keyphrase set. Both the soft and the hard exclusion mechanisms keep track of previously-predicted keyphrases within a window size to enhance the diversity of the generated keyphrases. Extensive experiments on multiple KG benchmark datasets demonstrate the effectiveness of our method to generate less duplicated and more accurate keyphrases.
