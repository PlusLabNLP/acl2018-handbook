This paper  describes  the  development  of  a verbal   morphological   parser   for   an   under-resourced  Papuan  language, Nen. Nen  verbal morphology is particularly complex, with a transitive verb taking up to 1,740 unique features. The  structural  properties exhibited  by Nen verbs raises interesting choices for analysis. Here we compare two possible methods of  analysis: ‘Chunking' and decomposition. ‘Chunking' refers to the concept of collating morphological segments into one, whereas the decomposition model follows a more classical linguistic approach. Both models are built using the Finite-State Transducer toolkit foma. The resultant architecture shows differences in size and structural clarity.  While the ‘Chunking' model is under half the size of the full de-composed counterpart, the decomposition displays higher structural order. In this paper, we describe the challenges encountered when modelling a language exhibiting distributed exponence and present the first morphological analyser for Nen, with an overall accuracy of 80.3\%.
