Stance detection is an important task, which aims to classify the attitude of an opinionated text towards a given target. Remarkable success has been achieved when sufficient labeled training data is available. However, annotating sufficient data is labor-intensive, which establishes significant barriers for generalizing the stance classifier to the data with new targets.  In this paper, we proposed a Semantic-Emotion Knowledge Transferring (SEKT) model for cross-target stance detection, which uses the external knowledge (semantic and emotion lexicons) as a bridge to enable knowledge transfer across different targets.  Specifically, a semantic-emotion heterogeneous graph is constructed from external semantic and emotion lexicons, which is then fed into a graph convolutional network to learn multi-hop semantic connections between words and emotion tags. Then, the learned semantic-emotion graph representation, which serves as prior knowledge bridging the gap between the source and target domains, is fully integrated into the bidirectional long short-term memory (BiLSTM) stance classifier by adding a novel knowledge-aware memory unit to the BiLSTM cell. Extensive experiments on a large real-world dataset demonstrate the superiority of SEKT against the state-of-the-art baseline methods.
