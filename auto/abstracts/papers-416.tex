We use single-agent and multi-agent Reinforcement Learning (RL) for learning dialogue policies in a resource allocation negotiation scenario. Two agents learn concurrently by interacting with each other without any need for simulated users (SUs) to train against or corpora to learn from. In particular, we compare the Q-learning, Policy Hill-Climbing (PHC) and Win or Learn Fast Policy Hill-Climbing (PHC-WoLF) algorithms, varying the scenario complexity (state space size), the number of training episodes, the learning rate, and the exploration rate. Our results show that generally Q-learning fails to converge whereas PHC and PHC-WoLF always converge and perform similarly. We also show that very high gradually decreasing exploration rates are required for convergence. We conclude that multi-agent RL of dialogue policies is a promising alternative to using single-agent RL and SUs or learning directly from corpora.
