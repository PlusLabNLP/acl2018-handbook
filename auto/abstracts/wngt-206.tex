We present the task of Simultaneous Translation and Paraphrasing for Language Education (STAPLE). Given a prompt in one language, the goal is to generate a diverse set of correct translations that language learners are likely to produce. This is motivated by the need to create and maintain large, high-quality sets of acceptable translations for exercises in a language-learning application, and synthesizes work spanning machine translation, MT evaluation, automatic paraphrasing, and language education technology. We developed a novel corpus with unique properties for five languages (Hungarian, Japanese, Korean, Portuguese, and Vietnamese), and report on the results of a shared task challenge which attracted 20 teams to solve the task. In our meta-analysis, we focus on three aspects of the resulting systems: external training corpus selection, model architecture and training decisions, and decoding and filtering strategies. We find that strong systems start with a large amount of generic training data, and then fine-tune with in-domain data, sampled according to our provided learner response frequencies.
