We describe two tasks—named entity recognition (Task 1) and template slot filling (Task 2)—for clinical texts. The tasks leverage annotations from the ShARe corpus, which consists of clinical notes with annotated mentions disorders, along with their normalization to a medical terminology and eight additional attributes. The purpose of these tasks was to identify advances in clinical named entity recognition and establish the state of the art in disorder template slot filling. Task 2 consisted of two subtasks: template slot filling given gold-standard disorder spans (Task 2a) and end-to-end disorder span identification together with template slot filling (Task 2b). For Task 1 (disorder span detection and normalization), 16 teams participated. The best system yielded a strict F1-score of 75.7, with a precision of 78.3 and recall of 73.2. For Task 2a (template slot filling given gold- standard disorder spans), six teams participated. The best system yielded a combined overall weighted accuracy for slot filling of 88.6. For Task 2b (disorder recognition and template slot filling), nine teams participated. The best system yielded a combined relaxed F (for span detection) and overall weighted accuracy of 80.8.
