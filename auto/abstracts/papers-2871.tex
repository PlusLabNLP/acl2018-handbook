Unsupervised constituency parsing aims to learn a constituency parser from a training corpus without parse tree annotations. While many methods have been proposed to tackle the problem, including statistical and neural methods, their experimental results are often not directly comparable due to discrepancies in datasets, data preprocessing, lexicalization, and evaluation metrics. In this paper, we first examine experimental settings used in previous work and propose to standardize the settings for better comparability between methods. We then empirically compare several existing methods, including decade-old and newly proposed ones, under the standardized settings on English and Japanese, two languages with different branching tendencies. We find that recent models do not show a clear advantage over decade-old models in our experiments. We hope our work can provide new insights into existing methods and facilitate future empirical evaluation of unsupervised constituency parsing.
