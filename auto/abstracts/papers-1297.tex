Question answering and conversational systems are often baffled and need help clarifying certain ambiguities. However, limitations of existing datasets hinder the development of large-scale models capable of generating and utilising clarification questions. In order to overcome these limitations, we devise a novel bootstrapping framework (based on self-supervision) that assists in the creation of a diverse, large-scale dataset of clarification questions based on post-comment tuples extracted from stackexchange. The framework utilises a neural network based architecture for classifying clarification questions. It is a two-step method where the first aims to increase the precision of the classifier and second aims to increase its recall. We quantitatively demonstrate the utility of the newly created dataset by applying it to the downstream task of question-answering. The final dataset, ClarQ, consists of {\textasciitilde}2M examples distributed across 173 domains of stackexchange. We release this dataset in order to foster research into the field of clarification question generation  with the larger goal of enhancing dialog and question answering systems.
