We report on a comparative style analysis of hyperpartisan (extremely one-sided) news and fake news. A corpus of 1,627 articles from 9 political publishers, three each from the mainstream, the hyperpartisan left, and the hyperpartisan right, have been fact-checked by professional journalists at BuzzFeed: 97\% of the 299 fake news articles identified are also hyperpartisan. We show how a style analysis can distinguish hyperpartisan news from the mainstream (F1 = 0.78), and satire from both (F1 = 0.81). But stylometry is no silver bullet as style-based fake news detection does not work (F1 = 0.46). We further reveal that left-wing and right-wing news share significantly more stylistic similarities than either does with the mainstream. This result is robust: it has been confirmed by three different modeling approaches, one of which employs Unmasking in a novel way. Applications of our results include partisanship detection and pre-screening for semi-automatic fake news detection.
