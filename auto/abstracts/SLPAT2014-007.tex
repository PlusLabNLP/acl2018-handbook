To help individuals with Alzheimer's disease live at home for longer, we are developing a mobile robotic platform, called ED, intended to be used as a personal caregiver to help with the performance of activities of daily living. In a series of experiments, we study speech-based interactions between each of 10 older adults with Alzheimer's disease and ED as the former makes tea in a simulated home environment. Analysis reveals that speech recognition remains a challenge for this recording environment, with word-level accuracies between 5.8\% and 19.2\% during household tasks with individuals with Alzheimer's disease. This work provides a baseline assessment for the types of technical and communicative challenges that will need to be overcome in human-robot interaction for this population.
