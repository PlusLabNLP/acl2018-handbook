More and more product information, including advertisements and user reviews, are presented to Internet users nowadays. Some of the information is false, misleading or overstated, which can cause seriousness and needs to be identified. Authorities, advertisers, website owners and consumers all have the needs to detect such statements. In this paper, we propose a False Advertisements Recognition system called FAdR by using one-class and binary classification models. Illegal advertising lists made public by a government and product descriptions from a shopping website are obtained for training and testing. The results show that the binary SVM models can achieve the highest performance when unigrams with the weighting of log relative frequency ratios are used as features. Comparatively, the benefit of the one-class classification models is the adjustable rejection rate parameter, which can be changed to suit different applications. Verb phrases more likely to introduce overstated information are obtained by mining the datasets. These phrases help find problematic wordings in the advertising texts.
