Current models of dialogue mainly focus on utterances within a topically coherent discourse segment, rather than new-topic utterances (NTUs), which begin a new topic not correlating with the content of prior discourse. As a result, these models may sufficiently account for discourse context of task-oriented but not social conversations. We conduct a pilot annotation study of NTUs as a first step towards a model capable of rationalizing conversational coherence in social talk. We start with the naturally occurring social dialogues in the Disco-SPICE corpus, annotated with discourse relations in the Penn Discourse Treebank and Cognitive approach to Coherence Relations frameworks. We first annotate content-based coherence relations that are not available in Disco-SPICE, and then heuristically identify NTUs, which lack a coherence relation to prior discourse. Based on the interaction between NTUs and their discourse context, we construct a classification for NTUs that actually convey certain non-topical coherence in social talk. This classification introduces new sequence-based social intents that traditional taxonomies of speech acts do not capture. The new findings advocates the development of a Bayesian game-theoretic model for social talk.
