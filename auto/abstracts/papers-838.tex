Fact checking is a challenging task because verifying the truthfulness of a claim requires reasoning about multiple retrievable evidence. In this work, we present a method suitable for reasoning about the semantic-level structure of evidence. Unlike most previous works, which typically represent evidence sentences with either string concatenation or fusing the features of isolated evidence sentences, our approach operates on rich semantic structures of evidence obtained by semantic role labeling. We propose two mechanisms to exploit the structure of evidence while leveraging the advances of pre-trained models like BERT, GPT or XLNet. Specifically, using XLNet as the backbone, we first utilize the graph structure to re-define the relative distances of words, with the intuition that semantically related words should have short distances. Then, we adopt graph convolutional network and graph attention network to propagate and aggregate information from neighboring nodes on the graph. We evaluate our system on FEVER, a benchmark dataset for fact checking, and find that rich structural information is helpful and both our graph-based mechanisms improve the accuracy. Our model is the state-of-the-art system in terms of both official evaluation metrics, namely claim verification accuracy and FEVER score.
