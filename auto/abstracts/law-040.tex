As researchers developing robust NLP for a wide range of text types, we are often confronted with the prejudice that annotation of non-canonical language (whatever that means) is somehow more arbitrary than annotation of canonical language. To investigate this, we present a small annotation study where annotators were asked, with minimal guidelines, to identify main predicates and arguments in sentences across five different domains, ranging from newswire to Twitter. Our study indicates that (at least such) annotation of non-canonical language is not harder. However, we also observe that agreements in social media domains correlate less with model confidence, suggesting that maybe annotators disagree for different reasons when annotating social media data.
