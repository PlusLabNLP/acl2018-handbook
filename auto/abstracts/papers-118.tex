Analyses of filler-gap dependencies usually involve complex syntactic rules or heuristics; however recent results suggest that filler-gap comprehension begins earlier than seemingly simpler constructions such as ditransitives or passives. Therefore, this work models filler-gap acquisition as a byproduct of learning word orderings (e.g. SVO vs OSV), which must be done at a very young age anyway in order to extract meaning from language. Specifically, this model, trained on part-of-speech tags, represents the preferred locations of semantic roles relative to a verb as Gaussian mixtures over real numbers. This approach learns role assignment in filler-gap constructions in a manner consistent with current developmental findings and is extremely robust to initialization variance. Additionally, this model is shown to be able to account for a characteristic error made by learners during this period ('A and B gorped' interpreted as 'A gorped B').
