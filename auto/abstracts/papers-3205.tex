Current advances in machine translation (MT) increase the need for translators to switch from traditional translation to post-editing (PE) of machine-translated text, a process that saves time and reduces errors. This affects the design of translation interfaces, as the task changes from mainly generating text to correcting errors within otherwise helpful translation proposals. Since this paradigm shift offers potential for modalities other than mouse and keyboard, we present MMPE, the first prototype to combine traditional input modes with pen, touch, and speech modalities for PE of MT. The results of an evaluation with professional translators suggest that pen and touch interaction are suitable for deletion and reordering tasks, while they are of limited use for longer insertions. On the other hand, speech and multi-modal combinations of select \\& speech are considered suitable for replacements and insertions but offer less potential for deletion and reordering. Overall, participants were enthusiastic about the new modalities and saw them as good extensions to mouse \\& keyboard, but not as a complete substitute.
