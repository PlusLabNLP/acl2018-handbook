Crowdsourcing makes it possible to create translations at much lower cost than hiring professional translators. However, it is still expensive to obtain the millions of translations that are needed to train statistical machine translation systems. We propose two mechanisms to reduce the cost of crowdsourcing while maintaining high translation quality. First, we develop a method to reduce redundant translations. We train a linear model to evaluate the translation quality on a sentence-by-sentence basis, and fit a threshold between acceptable and unacceptable translations. Unlike past work, which always paid for a fixed number of translations for each source sentence and then chose the best from them, we can stop earlier and pay less when we receive a translation that is good enough. Second, we introduce a method to reduce the pool of translators by quickly identifying bad translators after they have translated only a few sentences. This also allows us to rank translators, so that we re-hire only good translators to reduce cost.
