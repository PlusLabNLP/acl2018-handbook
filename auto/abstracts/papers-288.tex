In modern practice, labeling a dataset often involves aggregating annotator judgments obtained from crowdsourcing. State-of-the-art aggregation is performed via inference on probabilistic models, some of which are data-aware, meaning that they leverage features of the data (e.g., words in a document) in addition to annotator judgments. Previous work largely prefers discriminatively trained conditional models. This paper demonstrates that a data-aware crowdsourcing model incorporating a generative multinomial data model enjoys a strong competitive advantage over its discriminative log-linear counterpart in the typical crowdsourcing setting. That is, the generative approach is better except when the annotators are highly accurate in which case simple majority vote is often sufficient. Additionally, we present a novel mean-field variational inference algorithm for the generative model that significantly improves on the previously reported state-of-the-art for that model. We validate our conclusions on six text classification datasets with both human-generated and synthetic annotations.
