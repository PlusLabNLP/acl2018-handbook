Human speakers have an extensive toolkit of ways to express themselves. In this paper, we engage with an idea largely absent from discussions of meaning in natural language understanding—namely, that the way something is expressed reflects different ways of conceptualizing or construing the information being conveyed. We first define this phenomenon more precisely, drawing on considerable prior work in theoretical cognitive semantics and psycholinguistics. We then survey some dimensions of construed meaning and show how insights from construal could inform theoretical and practical work in NLP.
