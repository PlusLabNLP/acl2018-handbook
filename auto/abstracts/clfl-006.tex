Current machine translation (MT) techniques are continuously improving. In specific areas, post-editing (PE) can enable the production of high-quality translations relatively quickly. But is it feasible to translate a literary work (fiction, short story, etc) using such an MT+PE pipeline? This paper offers an initial response to this question. An essay by the American writer Richard Powers, currently not available in French, is automatically translated and post-edited and then revised by non-professional translators. In addition to presenting experimental evaluation results of the MT+PE pipeline (MT system used, automatic evaluation), we also discuss the quality of the translation output from the perspective of a panel of readers (who read the translated short story in French, and answered a survey afterwards). Finally, some remarks of the official French translator of R. Powers, requested on this occasion, are given at the end of this article.
