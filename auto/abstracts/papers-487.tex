Crowdsourcing is a viable mechanism for creating training data for machine translation. It provides a low cost, fast turnaround way of processing large volumes of data. However, when compared to professional translation, naive collection of translations from non-professionals yields low-quality results. Careful quality control is necessary for crowdsourcing to work well. In this paper, we examine the challenges of a two-step collaboration process with translation and post-editing by non-professionals. We develop graph-based ranking models that automatically select the best output from multiple redundant versions of translations and edits, and improves translation quality closer to professionals.
