Many recent Short Answer Scoring (SAS) systems have employed Quadratic Weighted Kappa (QWK) as the evaluation measure of their systems. However, we hypothesize that QWK is unsatisfactory for the evaluation of the SAS systems when we consider measuring their effectiveness in actual usage. We introduce a new task formulation of SAS that matches the actual usage. In our formulation, the SAS systems should extract as many scoring predictions that are not critical scoring errors (CSEs). We conduct the experiments in our new task formulation and demonstrate that a typical SAS system can predict scores with zero CSE for approximately 50\% of test data at maximum by filtering out low-reliablility predictions on the basis of a certain confidence estimation. This result directly indicates the possibility of reducing half the scoring cost of human raters, which is more preferable for the evaluation of SAS systems.
