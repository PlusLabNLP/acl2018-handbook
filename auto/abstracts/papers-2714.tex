Non-autoregressive (NAR) models generate all the tokens of a sequence in parallel, resulting in faster generation speed compared to their autoregressive (AR) counterparts but at the cost of lower accuracy. Different techniques including knowledge distillation and source-target alignment have been proposed to bridge the gap between AR and NAR models in various tasks such as neural machine translation (NMT), automatic speech recognition (ASR), and text to speech (TTS). With the help of those techniques, NAR models can catch up with the accuracy of AR models in some tasks but not in some others. In this work, we conduct a study to understand the difficulty of NAR sequence generation and try to answer: (1) Why NAR models can catch up with AR models in some tasks but not all? (2) Why techniques like knowledge distillation and source-target alignment can help NAR models. Since the main difference between AR and NAR models is that NAR models do not use dependency among target tokens while AR models do, intuitively the difficulty of NAR sequence generation heavily depends on the strongness of dependency among target tokens. To quantify such dependency, we propose an analysis model called CoMMA to characterize the difficulty of different NAR sequence generation tasks. We have several interesting findings: 1) Among the NMT, ASR and TTS tasks, ASR has the most target-token dependency while TTS has the least. 2) Knowledge distillation reduces the target-token dependency in target sequence and thus improves the accuracy of NAR models. 3) Source-target alignment constraint encourages dependency of a target token on source tokens and thus eases the training of NAR models.
