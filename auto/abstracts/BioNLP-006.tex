Medical coding is a process of classifying health records according to standard code sets representing procedures and diagnoses. It is an integral part of health care in the U.S., and the high costs it incurs have prompted adoption of natural language processing techniques for automatic generation of these codes from the clinical narrative contained in electronic health records. The need for effective auto-coding methods becomes even greater with the impending adoption of ICD-10, a code inventory of greater complexity than the currently used code sets. This paper presents a system that predicts ICD-10 procedure codes from the clinical narrative using several levels of abstraction. First, partial hierarchical classification is used to identify potentially relevant concepts and codes. Then, for each of these concepts we estimate the confidence that it appears in a procedure code for that document. Finally, confidence values for the candidate codes are estimated using features derived from concept confidence scores. The concept models can be trained on data with ICD-9 codes to supplement sparse ICD-10 training resources. Evaluation on held-out data shows promising results.
