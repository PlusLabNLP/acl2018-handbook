Suicide is a leading cause of death in the United States. One of the major challenges to suicide prevention is that those who may be most at risk cannot be relied upon to report their conditions to clinicians. This paper takes an initial step toward the automatic detection of suicidal risk factors through social media activity, with no reliance on self-reporting.  We consider the performance of annotators with various degrees of expertise in suicide prevention at annotating microblog data for the purpose of training text-based models for detecting suicide risk behaviors. Consistent with crowdsourcing literature, we found that novice-novice annotator pairs underperform expert annotators and outperform automatic lexical analysis tools, such as Linguistic Inquiry and Word Count.
