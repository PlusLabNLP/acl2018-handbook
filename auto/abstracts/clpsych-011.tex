Restrictive and repetitive behavior (RRB) is a core symptom of autism spectrum disorder (ASD) and are manifest in language. Based on this, we expect children with autism to talk about fewer topics, and more repeatedly, during their conversations. We thus hypothesize a higher semantic overlap ratio between dialogue turns in children with ASD compared to those with typical development (TD). Participants of this study include children ages 4-8, 44 with TD and 25 with ASD without language impairment. We apply several semantic similarity metrics to the children's dialogue turns in semi-structured conversations with examiners. We find that children with ASD have significantly more semantically overlapping turns than children with TD, across different turn intervals. These results support our hypothesis, and could provide a convenient and robust ASD-specific behavioral marker.
