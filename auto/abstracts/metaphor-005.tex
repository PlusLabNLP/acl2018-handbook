Metaphor is a fundamentally antagonistic way of viewing and describing the world. Metaphors ask us to see what is not there, so as to remake the world to our own liking and to suit our own lexicons. But if metaphors clash with the world as it is, they can also clash with each other. Each metaphor represents a stance from which to view a topic, and though some stances are mutually compatible, many more are naturally opposed to each other. So while we cringe at a clumsily mixed metaphor, there is real value to be had from a deliberate opposition of conceptual metaphors. Such contrasts reveal the limits of a particular worldview, and allow us to extract humorous insight from each op-position. We present here an automatic approach to the framing of antagonistic metaphors, embodied in a metaphor-generating Twitterbot named \@MetaphorMagnet.
