We propose the task of unsupervised morphological paradigm completion. Given only raw text and a lemma list, the task consists of generating the morphological paradigms, i.e., all inflected forms, of the lemmas. From a natural language processing (NLP) perspective, this is a challenging unsupervised task, and high-performing systems have the potential to improve tools for low-resource languages or to assist linguistic annotators. From a cognitive science perspective, this can shed light on how children acquire morphological knowledge. We further introduce a system for the task, which generates morphological paradigms via the following steps: (i) EDIT TREE retrieval, (ii) additional lemma retrieval, (iii) paradigm size discovery, and (iv) inflection generation. We perform an evaluation on 14 typologically diverse languages. Our system outperforms trivial baselines with ease and, for some languages, even obtains a higher accuracy than minimally supervised systems.
