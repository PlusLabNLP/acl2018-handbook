Natural disasters (e.g., hurricanes) affect millions of people each year, causing widespread destruction in their wake. People have recently taken to social media websites (e.g., Twitter) to share their sentiments and feelings with the larger community. Consequently, these platforms have become instrumental in understanding and perceiving emotions at scale. In this paper, we introduce HurricaneEmo, an emotion dataset of 15,000 English tweets spanning three hurricanes: Harvey, Irma, and Maria. We present a comprehensive study of fine-grained emotions and propose classification tasks to discriminate between coarse-grained emotion groups. Our best BERT model, even after task-guided pre-training which leverages unlabeled Twitter data, achieves only 68\% accuracy (averaged across all groups). HurricaneEmo serves not only as a challenging benchmark for models but also as a valuable resource for analyzing emotions in disaster-centric domains.
