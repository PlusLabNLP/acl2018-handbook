Legal Artificial Intelligence (LegalAI) focuses on applying the technology of artificial intelligence, especially natural language processing, to benefit tasks in the legal domain. In recent years, LegalAI has drawn increasing attention rapidly from both AI researchers and legal professionals, as LegalAI is beneficial to the legal system for liberating legal professionals from a maze of paperwork. Legal professionals often think about how to solve tasks from rule-based and symbol-based methods, while NLP researchers concentrate more on data-driven and embedding methods. In this paper, we introduce the history, the current state, and the future directions of research in LegalAI. We illustrate the tasks from the perspectives of legal professionals and NLP researchers and show several representative applications in LegalAI. We conduct experiments and provide an in-depth analysis of the advantages and disadvantages of existing works to explore possible future directions. You can find the implementation of our work from https://github.com/thunlp/CLAIM.
