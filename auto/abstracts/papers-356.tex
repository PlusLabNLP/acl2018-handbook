Consider a person trying to spread an important message on a social network. He/she can spend hours trying to craft the message. Does it actually matter? While there has been extensive prior work looking into predicting popularity of social-media content, the effect of wording per se has rarely been studied since it is of- ten confounded with the popularity of the author and the topic. To control for these confounding factors, we take advantage of the surprising fact that there are many pairs of tweets containing the same url and written by the same user but employing different wording. Given such pairs, we ask: which version attracts more retweets? This turns out to be a more difficult task than predicting popular topics. Still, humans can answer this question better than chance (but far from perfectly), and the computational methods we develop can do better than an average human as well as a strong competing method trained on non-controlled data.
