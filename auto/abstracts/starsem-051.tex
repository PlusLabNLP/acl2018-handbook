As they grow in size, OWL ontologies tend to comprise intuitively incompatible statements, even when they remain logically consistent. This is true in particular of lightweight ontologies, especially the ones which aggregate knowledge from different sources. The article investigates how distributional semantics can help detect and repair violation of common sense in consistent ontologies, based on the identification of consequences which are unlikely to hold if the rest of the ontology does. A score evaluating the plausibility for a consequence to hold with regard to distributional evidence is defined, as well as several methods in order to decide which statements should be preferably amended or discarded. A conclusive evaluation is also provided, which consists in extending an input ontology with randomly generated statements, before trying to discard them automatically.
