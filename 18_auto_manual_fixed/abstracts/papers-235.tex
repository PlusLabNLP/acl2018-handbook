Most of the neural sequence-to-sequence (seq2seq) models for grammatical error correction (GEC) have two limitations: (1) a seq2seq model may not be well generalized with only limited error-corrected data; (2) a seq2seq model may fail to completely correct a sentence with multiple errors through normal seq2seq inference. We attempt to address these limitations by proposing a fluency boost learning and inference mechanism. Fluency boosting learning generates fluency-boost sentence pairs during training, enabling the error correction model to learn how to improve a sentence's fluency from more instances, while fluency boosting inference allows the model to correct a sentence incrementally with multiple inference steps until the sentence's fluency stops increasing. Experiments show our approaches improve the performance of seq2seq models for GEC, achieving state-of-the-art results on both CoNLL-2014 and JFLEG benchmark datasets.
