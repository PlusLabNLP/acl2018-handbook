The purpose of text geolocation is to associate geographic information contained in a document with a set (or sets) of coordinates, either implicitly by using linguistic features and/or explicitly by using geographic metadata combined with heuristics. We introduce a geocoder (location mention disambiguator) that achieves state-of-the-art (SOTA) results on three diverse datasets by exploiting the implicit lexical clues. Moreover, we propose a new method for systematic encoding of geographic metadata to generate two distinct views of the same text. To that end, we introduce the Map Vector (MapVec), a sparse representation obtained by plotting prior geographic probabilities, derived from population figures, on a World Map. We then integrate the implicit (language) and explicit (map) features to significantly improve a range of metrics. We also introduce an open-source dataset for geoparsing of news events covering global disease outbreaks and epidemics to help future evaluation in geoparsing.
