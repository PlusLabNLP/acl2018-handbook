\documentclass[11pt]{article}
\usepackage{acl2014}
\usepackage{times}
\usepackage{url}
\usepackage{latexsym}
%\usepackage{natbib}
\usepackage{graphicx}
\usepackage{amsmath}
\usepackage{amsthm}
\usepackage{float}
\usepackage{algpseudocode}
\usepackage{algorithm}

%\usepackage{slashbox } 
%remove space around itemize
\usepackage{enumitem}
%\setlist{nolistsep}
%\special{papersize=210mm,297mm}
%space between table caption and text
\usepackage{caption}
\usepackage{color}
% \captionsetup[table]{skip=3pt}
%\captionsetup{belowskip=-10pt,aboveskip=4pt}s
%space around equations
\expandafter\def\expandafter\normalsize\expandafter{%
    \normalsize
   \setlength\abovedisplayskip{0pt}
    \setlength\belowdisplayskip{5pt}
    \setlength\abovedisplayshortskip{0pt}
    \setlength\belowdisplayshortskip{5pt}
}
%space around sections
\usepackage[compact]{titlesec}  
%\titlespacing{\section}{0pt}{0pt}{7pt}
%\titlespacing{\paragraph}{0pt}{0pt}{0pt}
%\titlespacing{\subparagraph}{0pt}{0pt}{0pt}

\title{Infusion of Labeled Data into Distant Supervision for Relation Extraction}


\author{Maria Pershina \textsuperscript{+}  Bonan Min\textsuperscript{\^{}}  
\hskip -0.05cm\thanks{\hskip 0.2cm Most of the work was done when this author was at New York University}
 \hskip 0.2cm Wei Xu \textsuperscript{\#} Ralph Grishman \textsuperscript{+}\\
  \textsuperscript{+}New York University, New York, NY \\
  {\tt \{pershina, grishman\}@cs.nyu.edu} \\
  \textsuperscript{\^{}}Raytheon BBN Technologies, Cambridge, MA\\
  {\tt bmin@bbn.com} \\
  \textsuperscript{\#}University of Pennsylvania, Philadelphia, PA\ \\
  {\tt xwe@cis.upenn.edu} 
 }
 
 
  
\date{}

\begin{document}
\maketitle
\begin{abstract}
 
Distant supervision usually utilizes only unlabeled data and existing knowledge bases to learn relation extraction models. 
However, in some cases a small amount of human labeled data is available. In this paper, we demonstrate how a state-of-the-art
multi-instance multi-label model can be modified to make use of these reliable sentence-level labels in addition
to the relation-level distant supervision from a database. Experiments show that our approach achieves a statistically significant increase of 13.5\% in 
F-score and 37\% in area under the precision recall curve.

\end{abstract}

\section{Introduction}

Relation extraction is the task of tagging semantic relations between pairs of entities from free text. Recently, distant supervision has 
emerged as an important technique for relation extraction and has attracted increasing attention because of its effective use of readily 
available databases \cite{mintz09,bunescu07,snyder07,wu07}.
It automatically labels its own training data by heuristically aligning a knowledge base of facts
with an unlabeled corpus. The intuition is that any sentence which mentions a pair of entities ($e_1$ and $e_2$) that 
participate in a relation, $r$, is likely to express the fact $r(e_1,\!e_2)$ and thus forms a positive training example of $r$. 

One of most crucial problems in distant supervision is the inherent errors in the automatically generated training data \cite{roth2013survey}. Table \ref{example_motivation} illustrates this problem
with a toy example.
Sophisticated multi-instance learning algorithms \cite{riedel10,hoffmann11,surdeanu12} have been proposed to 
address the issue by loosening the distant supervision assumption. These approaches consider all mentions of the same pair $(e_1,\!e_2)$ and assume
that $at$-$least$-$one$ mention actually expresses the relation.
On top of that, researchers further improved performance by explicitly adding preprocessing 
steps \cite{takamatsu12,xu13} or additional layers inside the model \cite{ritter13,min13} to reduce the effect of training noise. 

\captionsetup{belowskip=-14pt,aboveskip=-5pt}
\begin{table}[ht]
\small
\begin{center}
\begin{tabular}{ p{.8in}|p{1.9in} }
\hline
True Positive & ... to get information out of captured al-Qaida \textit{leader} {\textbf{Abu Zubaydah}}.\\
\hline
False Positive & {\textbf{...Abu Zubaydah}} and former Taliban \textit{leader} Jalaluddin Haqqani ... \\
\hline
False Negative & \textbf{...Abu Zubaydah} is one of Osama bin Laden's \textit{senior operational planners}...\\
\hline
\end{tabular}
\end{center}
\caption{Classic errors in the training data generated by a toy knowledge base of only one entry 
$\mathsf{personTitle}$(Abu Zubaydah, \textit{leader}).}
\label{example_motivation}
\end{table}

However, the potential of these previously proposed approaches is limited by the inevitable gap between the relation-level knowledge and the instance-level 
extraction task. In this paper, we present the first effective approach, $\mathsf{{\bf G}uided\ DS}$ (distant supervision), to incorporate labeled data into distant 
supervision for extracting relations from sentences. In contrast to simply taking the union of the hand-labeled data and the corpus labeled by 
distant supervision as in the previous work by Zhang et al.~\shortcite{zhang2012big}, we generalize the labeled data through feature selection and 
model this additional information directly in the latent variable approaches. Aside from previous semi-supervised work that employs labeled and unlabeled data \cite[and others]{yarowsky1995unsupervised,blum98,collins1999unsupervised,nigam2001using}, this is a learning scheme that combines unlabeled text and two training sources whose quantity and quality are radically different \cite{liang09}.

%this is a new learning scheme that combines unlabeled text and two training sources whose quantity and quality are radically different. 

\captionsetup{belowskip=-12pt,aboveskip=4pt}
\setlength{\intextsep}{10pt}
\setlength{\textfloatsep}{0pt}
\begin{table*}[ht]
\begin{tabular}{| l |c | }
\hline Guideline $g=\{g_i| i=1, 2, 3\}$: 						&  Relation $r(g)$       \\
  $\mathsf{types\ of\ entities}$, $dependency\ path$, span word (optional)    	&   \\
 \hline
 $\mathsf{person\_person}$,    $nsubj \rightarrow\leftarrow dobj$, married   			 &   $\mathsf{personSpouse}$ \\
$\mathsf{person\_organization}$,    $nsubj \rightarrow\leftarrow prep\_of$, became     	 &   $\mathsf{personMemberOf}$ \\
$\mathsf{organization\_organization}$,   $nsubj \rightarrow\leftarrow prep\_of$, company    	 &    $\mathsf{organizationSubsidiaries}$ \\
$\mathsf{person\_person}$,    $poss \rightarrow\leftarrow appos$, sister				&   $\mathsf{personSiblings}$\\
$\mathsf{person\_person}$,    $poss \rightarrow\leftarrow appos$, father				&   $\mathsf{personParents}$ \\
$\mathsf{person\_title}$,    $\leftarrow nn$   									&   $\mathsf{personTitle}$ \\
$\mathsf{organization\_person}$,    $prep\_of\rightarrow appos\rightarrow$     	&   $\mathsf{organizationTopMembersEmployees}$ \\
$\mathsf{person\_cause}$,    $nsubj \rightarrow\leftarrow prep\_of$     				&   $\mathsf{personCauseOfDeath}$ \\
$\mathsf{person\_number}$,    $\leftarrow appos$     							&   $\mathsf{personAge}$ \\
$\mathsf{person\_date}$,    $nsubjpass \rightarrow\leftarrow prep\_on\leftarrow num$    	 &  $\mathsf{personDateOfBirth}$ \\

\hline
\end{tabular}
\caption{\label{font-table} Some examples from the final set  ${\bf G}$ of extracted guidelines.}
\label{set}
\end{table*}

To demonstrate the effectiveness of our proposed approach, we extend $\mathsf{MIML}$ \cite{surdeanu12}, a state-of-the-art distant supervision model 
and show a significant improvement of 13.5\% in F-score on the relation extraction benchmark TAC-KBP \cite{ji11} dataset. While prior work employed tens 
of thousands of human labeled examples \cite{zhang2012big} and only got a 6.5\% increase in F-score over a logistic regression baseline, our approach 
uses much less labeled data (about 1/8) but achieves much higher improvement on performance over stronger baselines.



\section{The Challenge}

Simply taking the union of the hand-labeled data and the corpus labeled by distant supervision is not effective since hand-labeled
data will be swamped by a larger amount of distantly labeled data. An effective approach must recognize that the hand-labeled data
is more reliable than the automatically labeled data and so must take precedence
in cases of conflict. Conflicts cannot be limited to those cases where all the
features in two examples are the same; this would almost never occur, because of the dozens of features used by a typical relation
extractor \cite{zhou05}.
Instead we propose to perform feature selection to generalize human labeled data into \textit{training guidelines}, and integrate them into latent variable model.

 \subsection{Guidelines}
 \label{guidel}
 
The sparse nature of feature space dilutes the discriminative capability of useful features.
Given the small amount of hand-labeled data, it is important to
identify a small set of features that are general enough while being capable of
predicting quite accurately the type of relation that may hold
between two entities. 

We experimentally tested alternative feature sets by building
supervised Maximum Entropy (MaxEnt) models using the hand-labeled data (Table~\ref{maxent}), and selected
an effective combination of three features from the full feature set used by Surdeanu et al.,~\shortcite{surdeanu11}: \vspace{-1mm}
\begin{itemize}[leftmargin=*]\itemsep -0.3em
\item the semantic types of the two arguments (e.g. person, organization, location, date, title, ...)
 \item the sequence of dependency relations along the path
 connecting the heads of the two arguments in the dependency tree.
 \item a word in the sentence between the two arguments
\end{itemize}

%\vspace{-3mm}

These three features are strong indicators of the type
of relation between two entities.  In some cases the
semantic types of the arguments alone narrows the
possibilities to one or two relation types.
For example, entity types such as $\mathsf{person}$ and $\mathsf{title}$
often implies the relation $\mathsf{personTitle}$.
% sacrificed to save space
%, and types $\mathsf{person}$ and $\mathsf{date}$ likely lead to either
%$\mathsf{personDateOfBirth}$ or $\mathsf{personDateOfDeath}$.
Some lexical items
are clear indicators of particular relations, such as ``brother'' and ``sister'' for a sibling relationship
%sacrificed to save space
%, and more interestingly,
%``longtime'' and ``complications'' for the location of death.
 
 
 \begin{table} [bp]
 \vspace{8mm}
 \small
\begin{tabular}{ p{.8in}|p{.5in}| p{.5in}| p{.5in} }
\hline 
Model 					&  Precision        	   & Recall & F-score  \\ \hline
$\mathsf{MaxEnt}^{\rm all}$ 		& 18.6  	   &   	    6.3     		& 9.4  	\\ 		
  \hline
$\mathsf{MaxEnt}^{\rm two}$		&	 24.13 &	10.75 	  	&	14.87	\\ 	
  \hline
$\mathsf{MaxEnt}^{\rm three}$	&	 40.27 &	12.40 	  	&	18.97	\\ 	
  \hline
\end{tabular}
\caption{\label{font-table}Performance of a MaxEnt, trained on hand-labeled 
data using all features \cite{surdeanu11} vs using a subset of two (types of entities, dependency path), or three (adding a span word) features, and evaluated on the test set. }
\label{maxent}
\end{table}

We extract guidelines from hand-labeled data. Each guideline $g$=\{$g_i| i $=1,2,3\} consists of a pair of
semantic types, a dependency path, and optionally
a span word and is associated with a particular relation $r(g)$. We keep only those guidelines which make the correct
prediction for $all$ and at least $k$=3 examples in the training
corpus (threshold 3 was obtained by running experiments on the development dataset). Table~\ref{set} shows some examples in the final set ${\bf G}$ of extracted guidelines. 


\section{Guided DS}
\label{sec:gds}

Our goal is to jointly model human-labeled ground truth and structured data from a knowledge base in distant supervision. To do this, we extend the MIML model \cite{surdeanu12} by adding a new layer as shown in Figure \ref{gds}.

The input to the model consists of (1) distantly supervised data, represented as a list of {\it n} bags\footnote{A bag is a set of mentions sharing same entity pair.} 
 with a vector ${\bf y_i} $ of binary gold-standard  labels, 
either $Positive (P)$ or $Negative (N)$  for each relation $r\!\!\in\!\!R$; (2) generalized human-labeled ground truth, represented as a set {\bf G}
of feature conjunctions  $g$=\{$g_i| i $=1,2,3\} associated with a unique relation $r(g)$.
Given a bag of sentences, ${\bf x_i} $, which mention an $i$th entity pair ($e_1$,\! $e_2$), 
our goal is to correctly predict which relation is mentioned in each sentence, or $N\!R$ if none of the relations under consideration are mentioned.
The vector ${\bf z_i} $ contains the latent mention-level classifications for the $i$th entity pair. 
 We introduce a set of latent variables ${\bf h_i} $ which model human ground truth for each mention in the {\it i}th bag 
  and take precedence over the current model assignment ${\bf z_i}$.
 
 \begin{figure}[h!]
 \captionsetup{belowskip=-8pt,aboveskip=3pt}
\centerline{\includegraphics[width=4.8cm,height=5.0cm]{plateDS.pdf}}
\caption{Plate diagram of $\mathsf{{\bf G}uided\ DS}$ }
\label{gds}
\end{figure}

   Let $i, j$ be the index in the bag and the mention level, respectively.
 We model mention-level extraction $p(z_{ij}| {\bf x}_{ij}; {\bf w}_z)$, 
 human relabeling $h_{ij}({\bf x}_{ij}, z_{ij})$ and multi-label aggregation $p(y_i^r | {\bf h}_i; {\bf w_y})$.
 We define:
 
 \vspace{-2mm}
 \begin{itemize}[leftmargin=*]\itemsep -0.2em 
  \item $y_i^r\!\in\! \{P,N\}: r$ holds for the $i$th bag or not.
  \item ${\bf x}_{ij}$ is the feature representation of the $j$th relation mention in the $i$th bag. We use the same set of features as in Surdeanu et al. (2012).
 \item $z_{ij}\!\!\in\!\! R\cup N\!R$: a latent variable that denotes the relation of the $j$th mention in the $i$th bag \item $h_{ij}\!\in\! R\cup N\!R$: a latent variable that denotes the refined relation of the mention ${\bf x}_{ij}$
 \end{itemize}
 We define relabeled relations $h_{ij}$ as following:
 \vspace{1mm}
 $$h_{ij} (x_{ij}, z_{ij})\!\!=\!\! 
 \left\{ \begin{array}{l}
\!\!\!\!r(g),\!\ {\rm if\ } \exists!\ \! g\!\in\!{\bf G} {\rm\ s.t.\ } \!g\!=\!\!\{g_k\!\}\!\!\subseteq\!\!  \{{\bf x}_{ij}\!\} \   \\ 
\!\!\!z_{ij}\ \ , {\rm\ \ otherwise}
\end{array}\right.$$
 \vspace{-2mm}

Thus, relation $r(g)$ is assigned to $h_{ij}$ iff there exists a unique guideline $g\!\!\in\!\!{\bf G}$, 
such that the feature vector ${\bf x}_{ij}$ contains all constituents of $g$, i.e.
 entity types, a dependency path and maybe a span word, if $g$ has one. 
We use mention relation $z_{ij}$ inferred by the model 
only in case no such a guideline exists or there is more than one matching guideline.
We also define:

 \vspace{-2mm}
\begin{itemize}[leftmargin=*]\itemsep -0.2em
\item ${\bf w}_z$ is the weight vector for the multi-class relation mention-level classifier\footnote{All classifiers are implemented using L2-regularized logistic regression with Stanford CoreNLP package.}
\item ${\bf w}_y^r$ is the weight vector for the {\it r}th binary top-level aggregation classifier (from mention labels to bag-level prediction).
We use ${\bf w}_y$ to represent  ${\bf w}_y^1, {\bf w}_y^2,\dots, {\bf w}_y^{|R|}$.
\end{itemize}
 \vspace{-2mm}
Our approach is aimed at improving the mention-level classifier, while keeping the multi-instance multi-label
framework to allow for joint modeling.


\noindent
\begin{table*}[t!] 
\begin{tabular}{ l  || *{7}{c |} c }  \hline
\hskip 4.5cm Iteration &1 &2 &3 &4 &5 &6 &7 & 8
\\ \hline
(a) Corrected relations:	&   2052    & 718    & 648   & 596 & 505 & 545  & 557   & 535 \\ \hline
(b) Retrieved relations: 	&10219 & 860    & 676   & 670  & 621  & 599 & 594   & 592 \\ 	\hline
{ Total relabelings }       &      12271  &  1578& 1324 & 1264& 1226 & 1144 & 1153  &  1127 \\
\hline
\end{tabular}
\caption{\label{font-table} Number of relabelings for each training iteration of $\mathsf{{\bf G}uided\ DS}$:  
(a) relabelings due to corrected relations, e.g. $\mathsf{personChildren}\!\!\rightarrow\!\!\mathsf{personSiblings}$
(b) relabelings due to retrieved relations, e.g. $\mathsf{notRelated}(N\!R)\!\rightarrow\!\mathsf{personTitle}$}
\label{relabel}
\end{table*}

%\titlespacing{\section}{0pt}{2pt}{7pt}
%\vspace{-3mm}
\section{Training}

We use a hard expectation maximization algorithm to train the model. 
Our objective function is to maximize log-likelihood of the data:
\begin{align*}
&LL({\bf w_y, w_z}) = \sum_{i=1}^n \log p({\bf y_i|x_i, w_y, w_z, G})\\
&=  \sum_{i=1}^n\log \sum_{\bf h_i}  p({\bf y_i, h_i | x_i, w_y, w_z, G})\\
&= \sum_{i=1}^n\log \sum_{\bf h_i}
\prod_{j=1}^{|{\bf h_i}|}  p(h_{ij} |{\bf x}_{ij},\! {\bf w_z}, \!{\bf G})\!\!\!\!\!\!
       \prod_{r\in P_i\cup N_i}\!\! \!\!\!p(y_i^r  |{\bf h}_i, \!{\bf w}_{\bf y}^r)
\end{align*}
where the last equality is due to conditional independence.
Because of the non-convexity of $LL({\bf w_y, w_z})$ we approximate and maximize the joint log-probability $p({\bf y_i,\! h_i | x_i,\! w_y,\!\! w_z,\! G}\!)$ for each entity pair in the database:

\begin{align*}
&\log p({\bf y_i,\! h_i | x_i,\! w_y,\!\! w_z,\! G}\!)  \\
&=\sum_{j=1}^{|{\bf h_i}|}\!   \log   p(h_{ij} |{\bf x}_{ij}, \! {\bf w_z, \!G})\! +\!\!\!\!\!\!\!
       \sum_{r\in P_i\cup N_i}\!\!\!\! \!\!\log p(y_i^r  |{\bf h}_i,\! {\bf w}_{\bf y}^r). 
 \end{align*}
 

\begin{algorithm}[h!]
  \caption{:     {\bf G}uided DS training}\label{alg}
  \begin{algorithmic}[1]
    \State Phase 1: build set {\bf G} of guidelines
    \State Phase 2: EM training
    \For { ${\rm iteration} = 1,\dots, T$}
        % \State // E-step: for each entity tuple
   	 \For  {$i = 1, \dots , n$}
	 	% \State // infer mention-level labels
	    	 \For  {$j = 1, \dots , |{\bf x_i}|$}
		 \State  $\!\!\!\!\!\!z_{ij}^* \!\!= {\rm argmax}_{z_{ij}} p(z_{ij} |{\bf x_i,\! y_i,\!w_z,\!w_y}\!)$
		 %\State \!\!\!\!\!// guided relabeling
		 \State $\!\!\!\!\!\!h_{ij}^*\!\!\!=\!\!
		 \left\{ \begin{array}{l}
			\!\!\!\!r(g),\!\ {\rm if\ } \exists!\ \! g\!\in\!{\bf G}\!:\! \{g_k\!\}\!\!\subseteq\!\!  \{{\bf x}_{ij}\!\} \   \\ 
			\!\!\!{z_{ij}}^*\ , {\rm\ otherwise}
		\end{array}\right.$
		 \State \!\!\!\!\!\!{\bf update $\bf h_i$ with } $h_{ij}^*$
	 	\EndFor 		
    	\EndFor 
	\State \!${\bf w}_{\bf z}^*\!\! 
			=\!\!{\rm argmax}_{\bf w}\!\sum_{i=1}^n\!\sum_{j=1}^{|{\bf x_i}|}\! \log p(h_{ij} | {\bf x}_{ij}, \!\!{\bf w}\!)$
	\For {$r\in R$}
	 \State \!\!\!\!${\bf w_y^{r*}}\!\! =\!{\rm argmax}_{\bf w}\!\!\!\!\!\!\!\!\!\!\!\sum\limits_{1\le i\le n\ s.t.\ r\in P_i\cup N_i}
		                                                                                     \!\!\!\!\!\!\!\!\!\!\!   \log p(y_i^r | {\bf h_i}, \!\!{\bf w}\!)$
	\EndFor
    \EndFor 
    \State {\bf return} $\bf w_z, w_y$
  \end{algorithmic}
  \label{algDS}
\end{algorithm}

The pseudocode is presented as algorithm 1.

The following approximation is used for inference at step 6:
\begin{align*}p(z_{ij} | &{\bf x_i,\! y_i,\!w_z,\!w_y}\!) \propto p({\bf y_i}, z_{ij} |{\bf x_i, w_y, w_z})\\
			     &\approx p(z_{ij} |x_{ij}, {\bf w_z}) p({\bf y_i | h_i', w_y})\\
			     & = p(z_{ij} | x_{ij}, {\bf w_z}) \prod_{r\in P_i\cup N_i} p(y_i^r | {\bf h_i', w_y^r}),
\end{align*}


where ${\bf h_i'}$ contains previously inferred and maybe further relabeled mention labels for group $i$ (steps 5-10), with the exception of component $j$ whose label is replaced by $z_{ij}$. 
% sacrificed this sentence to save the space
%Thus the E-step comprises the inference of a mention-level labels $\bf z_i$, based on current parameters of the model and previously inferred labels for group $i$, and then guided relabeling $\bf h_i$ via guidelines in a set {\bf G}. 
In the M-step (lines 12-15) we optimize model parameters $\bf w_z, w_y$, given the current assignment 
of mention-level labels $\bf h_i$.

Experiments show that $\mathsf{{\bf G}uided\ DS}$
efficiently learns new model, 
resulting in a drastically decreasing 
number of needed relabelings for further iterations (Table~\ref{relabel}).
At the inference step we first classify all mentions: 
$$z_{ij}^*={\rm argmax}_{z\in R\cup N\!R} \ \ p(z | x_{ij}, {\bf w_z})$$
Then final relation labels for $i$th entity tuple are obtained via the top-level classifiers:
$$y_i^{r*} = {\rm argmax}_{y\in \{P,N\} } \ \ p(y |{\bf z_i^*, w_y^r})$$


\section{Experiments}
\label{exp}

\subsection{Data}
We use the KBP \cite{ji11} 
dataset\footnote{Available from Linguistic Data Consortium (LDC) at http://projects.ldc.upenn.edu/kbp/data.} 
which is preprocessed by Surdeanu et al.~\shortcite{surdeanu11} using the Stanford parser\footnote{http://nlp.stanford.edu/software/lex-parser.shtml}~\cite{klein03}. 
This dataset is generated by mapping Wikipedia 
 infoboxes into a large unlabeled corpus that consists of 1.5M documents from KBP source corpus and a complete snapshot of Wikipedia. 

The KBP 2010 and 2011 data includes 200 query named entities with the relations they are involved in. 
We used 40 queries as development set and the rest 160 queries (3334 entity pairs that express a relation) as the test set.
The official KBP evaluation is performed by pooling the system responses and manually reviewing each response,
 producing a hand-checked assessment data. We used KBP  2012 assessment data to generate guidelines since
queries from different years do not overlap. It contains about 2500
labeled sentences of 41 relations, which
is less than 0.09\% of the size of the distantly labeled dataset of 2M sentences.
The final set {\bf G} consists of 99 guidelines (section~\ref{guidel}).

 \captionsetup{belowskip=-15pt,aboveskip=3pt}
\begin{figure*}[ht!]
\includegraphics[width=16.0cm,height=7.0cm]{picP.pdf}
\caption{Performance of $\mathsf{{\bf G}uided\ DS}$ on KBP task compared to a) baselines: 
$\mathsf{MaxEnt}$, $\mathsf{DS}$+$\mathsf{upsampling}$, $\mathsf{Semi}$-$\mathsf{MIML}$~\cite{min13}
b) state-of-art models: $\mathsf{Mintz}$++~\cite{mintz09}, $\mathsf{MultiR}$~\cite{hoffmann11}, $\mathsf{MIML}$~\cite{surdeanu12}}
\label{plots}
\end{figure*}

\subsection{Models}


We implement $\mathsf{{\bf G}uided\ DS}$ on top of the $\mathsf{MIML}$ \cite{surdeanu12} 
code base\footnote{Available at http://nlp.stanford.edu/software/mimlre.shtml.}. 
Training $\mathsf{MIML}$ on a simple fusion of distantly-labeled and human-labeled datasets does not improve the maximum F-score since
this hand-labeled data is swamped by a much larger amount of distant-supervised data of much lower quality. Upsampling the labeled data did not improve the performance either.
% since model does not make any distinction between human-labeled and distantly-labeled data.
%Mixing distantly-labeled data with upsampled labeled data did not improve the performance either. 
We experimented with different upsampling ratios and report best results using ratio 1:1 in Figure \ref{plots}.
%Upsampling (replicating) the labeled data to the same size as distantly labeled data did not improve
%the performance either (Figure \ref{plots}).

Our baselines: %We compare $\mathsf{{\bf G}uided\ DS}$ with three baselines and three state-of-the-art models:
1) $\mathsf{MaxEnt}$ is a supervised maximum entropy baseline trained on a human-labeled data;
2) $\mathsf{DS}$+$\mathsf{upsampling}$ is an upsampling experiment, where $\mathsf{MIML}$ was trained on a 
mix of a distantly-labeled and human-labeled data; 
3) $\mathsf{Semi}$-$\mathsf{MIML}$ is a recent semi-supervised extension.
We also compare $\mathsf{{\bf G}uided\ DS}$ with three state-of-the-art models:
1) $\mathsf{MultiR}$ and 2) $\mathsf{MIML}$ are two distant supervision models that support multi-instance learning
and overlapping relations;
3) $\mathsf{Mintz}$++ is a single-instance learning algorithm for distant supervision.
The difference between $\mathsf{{\bf G}uided\ DS}$ and all other systems is significant 
with $p$-value less than 0.05 according to a paired $t$-test assuming a normal distribution.

\subsection{Results}

We scored our model against all 41 relations and thus replicated the actual KBP evaluation. 
Figure \ref{plots} shows that our model consistently outperforms all six algorithms at almost all recall levels and 
improves the maximum $F$-score by more than 13.5$\%$ relative to $\mathsf{MIML}$ (from 28.35$\%$
to 32.19$\%$) as well as increases the area under precision-recall curve by more than 37\% (from 11.74 to 16.1).
Also, $\mathsf{{\bf G}uided\ DS}$ improves the overall recall by more than 9\% absolute 
(from 30.9\% to 39.93\%) at a comparable level of precision 
(24.35\% for $\mathsf{MIML}$ vs 23.64\% for  $\mathsf{{\bf G}uided\ DS}$), while increases the running time of $\mathsf{MIML}$ by only 3\%.
Thus, our approach outperforms state-of-the-art model for relation extraction using much less labeled data
that was used by Zhang et al.,~\shortcite{zhang2012big} to outperform logistic regression baseline.
Performance of $\mathsf{{\bf G}uided\ DS}$ also compares favorably with best scored hand-coded systems for a similar task such as Sun et al.,~\shortcite{sun11} 
system for KBP 2011, which reports an F-score of 25.7\%. 


\section{Conclusions and Future Work}

We show that relation extractors trained with distant supervision can benefit significantly from a small number of human labeled examples. We propose a strategy to generate and select guidelines so that they are more generalized 
forms of labeled instances. We show how to incorporate these guidelines into an existing state-of-art model for relation extraction. Our approach significantly improves performance in practice and thus opens up many opportunities 
for further research in RE where only a very limited amount of labeled training data is available.

%sacrificed the future work in order to save space
%Our future work will be focused on negative guidelines: combinations of features that indicate that the sentence does not express a particular relation. These can be used in a similar fashion and help a DS model with false positives. This strategy will also compensate for errors caused by the weak at-least-one assumption and thus further improve the mention-level classifier.
\vspace{5mm}
\section*{Acknowledgments}
Supported by the Intelligence Advanced Research Projects Activity (
IARPA) via Air Force Research Laboratory (AFRL) contract number FA8650-10-C-7058. The U.S. Government is authorized to reproduce and distribute reprints for Governmental purposes notwithstanding any copyright annotation thereon. The views and conclusions contained herein are those of the authors and should not be interpreted as necessarily representing the official policies or endorsements, either expressed or implied, of IARPA, AFRL, or the U.S. Government.
\newpage

\nocite{*}

\bibliographystyle{acl2014}
\bibliography{acl_guidedDS}

\end{document}


%This research was s
%upported
%in part
%by the Intel-
%ligence Advanced Research Projects Activity
%(IARPA) via Air Force Research Laboratory
%(AFRL) contract number FA8650
%-
%10
%-
%C
%-
%7058
%and
%via Department of Interior National Business
%Center (DoI/NBC) contract number
%D11PC20154
%. The U.S. Government is a
%uthor-
%ized to reproduce and distribute reprints for Gov-
%ernmental purposes notwithstanding any copy-
%right annotation thereon. The views and conclu-
%sions contained
%herein are those of the author
%and
%should not be interpreted as necessarily repre-
%senting the offi
%cial policies or endorsements, ei-
%ther expressed or implied, of IARPA, AFRL,
%DoI/NBC, or the U.S. Government
