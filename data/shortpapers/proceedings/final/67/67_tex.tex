%
% File acl2014.tex
%
% Contact: koller@ling.uni-potsdam.de, yusuke@nii.ac.jp
%%
%% Based on the style files for ACL-2013, which were, in turn,
%% Based on the style files for ACL-2012, which were, in turn,
%% based on the style files for ACL-2011, which were, in turn, 
%% based on the style files for ACL-2010, which were, in turn, 
%% based on the style files for ACL-IJCNLP-2009, which were, in turn,
%% based on the style files for EACL-2009 and IJCNLP-2008...

%% Based on the style files for EACL 2006 by 
%%e.agirre@ehu.es or Sergi.Balari@uab.es
%% and that of ACL 08 by Joakim Nivre and Noah Smith

\documentclass[11pt]{article}
\usepackage{acl2014}

\usepackage{amsmath}
\usepackage{mathptmx}
\usepackage{times}
\usepackage{url}
\usepackage{latexsym}
\usepackage[round]{natbib}
\usepackage{graphicx}
\usepackage{multirow}
\usepackage{soul}
\usepackage{array}

\DeclareMathOperator*{\argmax}{arg\,max}

%\setlength\titlebox{5cm}

% You can expand the titlebox if you need extra space
% to show all the authors. Please do not make the titlebox
% smaller than 5cm (the original size); we will check this
% in the camera-ready version and ask you to change it back.


\title{Applying a Naive Bayes Similarity Measure to\\Word Sense
  Disambiguation}

\author{Tong Wang \\
  University of Toronto\\
  {\tt tong@cs.toronto.edu} \And
  Graeme Hirst \\
  University of Toronto\\
  {\tt gh@cs.toronto.edu}}

\date{}

\begin{document}
\maketitle
\begin{abstract}
  We replace the overlap mechanism of the Lesk algorithm with a
  simple, general-purpose Naive Bayes model that measures
  \textit{many-to-many} association between two sets of random
  variables.  Even with simple probability estimates such as maximum
  likelihood, the model gains significant improvement over the Lesk
  algorithm on word sense disambiguation tasks. With additional
  lexical knowledge from WordNet, performance is further improved to
  surpass the state-of-the-art results.
\end{abstract}

\section{Introduction}
\label{sec:introduction}
To disambiguate a homonymous word in a given context,
\citet{lesk_automatic_1986} proposed a method that measured the degree
of overlap between the glosses of the target and context words.  Known
as the Lesk algorithm, this simple and intuitive method has since been
extensively cited and extended in the word sense disambiguation (WSD)
community. Nonetheless, its performance in several WSD benchmarks is
less than satisfactory
\citep{kilgarriff2000framework,vasilescu2004evaluating}. Among the
popular explanations is a key limitation of the algorithm, that
``Lesk's approach is very sensitive to the exact wording of
definitions, so the absence of a certain word can radically change the
results.''  \citep{navigli2009word}.

Compounding this problem is the fact that many Lesk variants limited
the concept of overlap to the literal interpretation of string
matching (with their own variants such as length-sensitive matching
\citep{banerjee2002adapted}, etc.), and it was not until recently that
overlap started to take on other forms such as tree-matching
\citep{Chen:2009:FUW:1620754.1620759} and vector space models
\citep{Abdalgader:2012:USW:2168748.2168750,raviv2012concept,patwardhan2006using}. To
address this limitation, a Naive Bayes model (NBM) is proposed in this
study as a novel, probabilistic treatment of overlap in gloss-based
WSD.

\section{Related Work}
\label{sec:related-work}
In the extraordinarily rich literature on WSD, we focus our review on
those closest to the topic of Lesk and NBM. In particular, we opt for
the ``simplified Lesk'' \citep{kilgarriff2000framework}, where
inventory senses are assessed by gloss-context overlap rather than
gloss-gloss overlap. This particular variant prevents proliferation of
gloss comparison on larger contexts \citep{mihalcea2004pagerank} and
is shown to outperform the original Lesk algorithm
\citep{vasilescu2004evaluating}.
%These two algorithms are to be referred to interchangeably henceforth.

To the best of our knowledge, NBMs have been employed exclusively as
classifiers in WSD --- that is, in contrast to their use as a
similarity measure in this study. \citet{gale1992method} used NB
classifier resembling an information retrieval system: a WSD instance
is regarded as a document $d$, and candidate senses are scored in
terms of ``relevance'' to $d$. When evaluated on a WSD benchmark
\citep{vasilescu2004evaluating}, the algorithm compared favourably to
Lesk variants (as expected for a supervised
method). \citet{pedersen2000simple} proposed an ensemble model with
multiple NB classifiers differing by context window size.
\citet{hristea2009recent} trained an unsupervised NB classifier using
the EM algorithm and empirically demonstrated the benefits of
WordNet-assisted \citep{fellbaum1998wne} feature selection over local
syntactic features.

Among Lesk variants, \citet{banerjee2002adapted} extended the gloss of
both inventory senses and the context words to include words in their
related synsets in WordNet. Senses were scored by the sum of overlaps
across all relation pairs, and the effect of individual relation pairs
was evaluated in a later work \citep{banerjee2003extended}. Overlap
was assessed by string matching, with the number of matching words
squared so as to assign higher scores to multi-word overlaps.

Breaking away from string matching, \citet{wilks1990providing}
measured overlap as similarity between gloss- and context-vectors,
which were aggregated word vectors encoding second order co-occurrence
information in glosses. An extension by \citet{patwardhan2006using}
differentiated context word senses and extended shorter glosses with
related glosses in WordNet. \citet{patwardhan_using_2003} measured
overlap by \textit{concept similarity}
\citep{budanitsky2006evaluating} between each inventory sense and the
context words. Gloss overlaps from their earlier work actually
out-performed all five similarity-based methods.

More recently, \citet{Chen:2009:FUW:1620754.1620759} proposed a
tree-matching algorithm that measured gloss-context overlap as the
weighted sum of dependency-induced lexical
distance. \citet{Abdalgader:2012:USW:2168748.2168750} constructed a
\textit{sentential} similarity measure \citep{li2006sentence} using
\textit{lexical} similarity measures \citep{budanitsky2006evaluating},
and overlap was measured by the cosine of their respective sentential
vectors. A related approach \citep{raviv2012concept} also used
Wikipedia-induced concepts to encoded sentential vectors. These
systems compared favourably to existing methods in WSD performance,
although by using sense frequency information, they are essentially
supervised methods.

Distributional methods have been used in many WSD systems in quite
different flavours than the current
study. \citet{kilgarriff2000framework} proposed a Lesk variant where
each gloss word is weighted by its \textit{idf} score in relation to
all glosses, and gloss-context association was incremented by these
weights rather than binary, overlap counts. \citet{miller2012using}
used distributional thesauri as a knowledge base to increase overlaps,
which were, again, assessed by string matching.

In conclusion, the majority of Lesk variants focused on extending the
gloss to increase the chance of overlapping, while the proposed NBM
aims to make better use of the limited lexical knowledge available. In
contrast to string matching, the probabilistic nature of our model
offers a ``softer'' measurement of gloss-context association,
resulting in a novel approach to unsupervised WSD with
state-of-the-art performance in more than one WSD benchmark (Section
\ref{sec:evaluation}).

\section{Model and Task Descriptions}
\label{sec:softlesk-using-nbm}
\subsection{The Naive Bayes Model}
\label{sec:prob-model}
Formally, given two sets $\mathbf{e}=\{e_i\}$ and $\mathbf{f}=\{f_j\}$
each consisting of multiple random events, the proposed model measures
the probabilistic association $p(\mathbf{f}|\mathbf{e})$ between
$\mathbf{e}$ and $\mathbf{f}$. Under the assumption of conditional
independence among the events in each set, a Naive Bayes treatment of
the measure can be formulated as:
\begin{equation}
  \label{eq:nbm}
\begin{aligned}
  p(\mathbf{f}|\mathbf{e})
  =&\prod_jp(f_j|\{e_i\})=\prod_j\frac{p(\{e_i\}|f_j)p(f_j)}{p(\{e_i\})}\\
=&\frac{\prod_j[p(f_j)\prod_ip(e_i|f_j)]}{\prod_j\prod_ip(e_i)},
\end{aligned}
\end{equation}

% the logrithm of which becomes:
% \begin{align*}
% \sum_i\log p(e_i)+\sum_i\sum_j\log p(f_j|e_i)-|\mathbf{e}|\sum_j\log p(f_j).\\
% \end{align*}

% Max.\ likelihood:\\
% \begin{align*}
% =&\sum_i\log \frac{c(e_i)}{c(\cdot)}+\sum_i\sum_j\log \frac{c(f_j,e_i)}{c(e_i)}-|\{e_i\}|\sum_j\log \frac{c(f_j)}{c(\cdot)}\\
% =&\sum_i\log c(e_i)-\sum_i\log c(\cdot)+\sum_i\sum_j\log c(f_j,e_i)\\
% &-\sum_i\sum_j\log c(e_i)-|\{e_i\}|\sum_j\log c(f_j)+|\{e_i\}|\sum_j\log c(\cdot)\\
% =&(1-|\{f_j\}|)\sum_i\log c(e_i)-|\{e_i\}|\sum_j\log c(f_j)\\
% &+\sum_i\sum_j\log c(f_j,e_i)+|\{e_i\}|(|\{f_j\}|-1)\log c(\cdot)\\
% \end{align*}
 
% To regularize for the sizes of
% $\mathbf{e}$ and $\mathbf{f}$, one can also adjust the probability by
% the following normalization terms:
% \[
%   \tilde p(\mathbf{e}|\mathbf{f}) =
%   \frac{p(\mathbf{e}|\mathbf{f})/(|\mathbf{e}|+|\mathbf{f}|)}{
%     \sum_{(\hat{\mathbf{e}},
%       \hat{\mathbf{f}})\in\mathcal{E}\times\mathcal{F}}{\frac{1}{|\hat{\mathbf{e}}|
%       + |\hat{\mathbf{f}}|}}},
% \]
% where the denominator is to ensure that $\tilde
% p(\mathbf{e}|\mathbf{f})$ is a proper distribution. For a prescribed
% pair of data set and lexical knowledge base, it is a constant and thus
% can be omitted when winning sense is set to be .

\noindent In the second expression, Bayes's rule is applied not only
to take advantage of the conditional independence among $e_i$'s, but
also to facilitate probability estimation, since $p(\{e_i\}|f_j)$ is
easier to estimate in the context of WSD, where sample spaces of
$\mathbf{e}$ and $\mathbf{f}$ become asymmetric (Section
\ref{sec:model-appl-wsd}).

\subsection{Model Application in WSD}
\label{sec:model-appl-wsd}
In the context of WSD, $\mathbf{e}$ can be regarded as an instance of
a polysemous word $w$, while $\mathbf{f}$ represents certain lexical
knowledge about the sense $s$ of $w$ manifested by
$\mathbf{e}$.\footnote{Think of the notations $\mathbf{e}$ and
  $\mathbf{f}$ mnemonically as \textit{exemplars} and
  \textit{features}, respectively.}  WSD is thus formulated as
identifying the sense $s^*$ in the sense inventory $\mathcal{S}$ of
$w$ s.t.:
\begin{equation}
  \label{eq:wsd}
s^*=\argmax_{s\in\mathcal{S}}p(\mathbf{f}|\mathbf{e})
\end{equation}

In one of their simplest forms, $e_i$'s correspond to co-occurring
words in the instance of $w$, and $f_j$'s consist of the gloss words
of sense $s$. Consequently, $p(\mathbf{f}|\mathbf{e})$ is essentially
measuring the association between context words of $w$ and definition
texts of $s$, i.e., the gloss-context association in the simplified
Lesk algorithm \citep{kilgarriff2000framework}. A major difference,
however, is that instead of using hard, overlap counts between the two
sets of words from the gloss and the context, this probabilistic
treatment can implicitly model the distributional similarity among the
elements $e_i$ and $f_j$ (and consequently between the sets
$\mathbf{e}$ and $\mathbf{f}$) over a wider range of contexts. The
result is a ``softer'' proxy of association than the binary view of
overlaps in existing Lesk variants.

The foregoing discussion offers a second motivation for applying
Bayes's rule on the second expression in Equation \eqref{eq:nbm}: it
is easier to estimate $p(e_i|f_j)$ than $p(f_j|e_i)$, since the
vocabulary for the lexical knowledge features ($f_j$) is usually more
limited than that of the contexts ($e_i$) and hence estimation of the
former suffices on a smaller amount of data than that of the latter.

\subsection{Incorporating Additional Lexical Knowledge}
\label{sec:incorp-addit-lexic}
The input of the proposed NBM is bags of words, and thus it is
straightforward to incorporate various forms of lexical knowledge (LK)
for word senses: by concatenating a tokenized knowledge source to the
existing knowledge representation $\mathbf{f}$, while the similarity
measure remains unchanged.

The availability of LK largely depends on the sense inventory used in
a WSD task. WordNet senses are often used in Senseval and SemEval
tasks, and hence senses (or synsets, and possibly their corresponding
word forms) that are semantic related to the inventory senses under
WordNet relations are easily obtainable and have been exploited by
many existing studies.

As pointed out by \citet{patwardhan_using_2003}, however, ``not all of
these relations are equally helpful.'' Relation pairs involving
hyponyms were shown to result in better F-measure when used in gloss
overlaps \citep{banerjee2003extended}. The authors attributed the
phenomenon to the the multitude of hyponyms compared to other
relations. We further hypothesize that, beyond sheer numbers, synonyms
and hyponyms offer stronger semantic specification that helps
distinguish the senses of a given ambiguous word, and thus are more
effective knowledge sources for WSD.

\def\halfTableWidth{.48\textwidth} \def\cellWidth{16mm}
\begin{table}[t]
  \centering
   \resizebox{\halfTableWidth}{!} {
  \begin{tabular}{r|p{\cellWidth}p{\cellWidth}p{\cellWidth}}
    \hline
    \hline
    Senses   &Hypernyms&Hyponyms &Synonyms\\
    \hline
    \textit{factory} & building complex, complex & brewery,
    factory, mill, ...  & works, industrial plant\\
    \hline
    \textit{life form} & organism, being & perennial, crop...  & flora, plant life\\
    \hline
    \hline
  \end{tabular}}
  \caption{Lexical knowledge for the word \textit{plant}
    under its two meanings \textit{factory} and \textit{life form}.}
\label{tab:plant_lex_knowledge}
\end{table}
Take the word \textit{plant} for example. Selected hypernyms,
hyponyms, and synonyms pertaining to its two senses \textit{factory}
and \textit{life form} are listed in Table
\ref{tab:plant_lex_knowledge}. Hypernyms can be overly general terms
(e.g., \textit{being}). Although conceptually helpful for humans in
coarse-grained WSD, this generality is likely to inflate the
hypernyms' probabilistic estimation. Hyponyms, on the other hand, help
specify their corresponding senses with information that is possibly
missing from the often overly brief glosses: the many technical terms
as hyponyms in Table \ref{tab:plant_lex_knowledge} --- though rare ---
are likely to occur in the (possibly domain-specific) contexts that
are highly typical of the corresponding senses. Particularly for the
NBM, the co-occurrence is likely to result in stronger
gloss-definition associations when similar contexts appear in a WSD
instance.

We also observe that some semantically related words appear under rare
senses (e.g., \textit{still} as an alcohol-manufacturing plant, and
\textit{annual} as a one-year-life-cycle plant; omitted from Table
\ref{tab:plant_lex_knowledge}). This is a general phenomenon in
gloss-based WSD and is beyond the scope of the current
discussion.\footnote{We do, however, refer curious readers to the work
  of \citet{raviv2012concept} for a novel treatment of a similar
  problem.} Overall, all three sources of LK may complement each other
in WSD tasks, with hyponyms particularly promising in both quantity
and quality compared to hypernyms and synonyms.\footnote{Note that LK
  expansion is a feature of our model rather than a requirement. What
  type of knowledge to include is eventually a decision made by the
  user based on the application and LK availability.}

\subsection{Probability Estimation}
\label{sec:prob-estim}
A most open-ended question is how to estimate the probabilities in
Equation \eqref{eq:nbm}. In WSD in particular, the estimation concerns
the marginal and conditional probabilities of and between word
tokens. Many options are available to this end in statistical machine
learning (MLE, MAP, etc.), information theory
\citep{church1990word,turney_miningweb_2001}, as well as the rich body
of research in lexical semantic similarity
\citealp{resnik1995using,jiang1997semantic,budanitsky2006evaluating}).

Here we choose maximum likelihood --- not only for its simplicity, but
also to demonstrate model strength with a relatively crude probability
estimation. To avoid underflow, Equation \eqref{eq:nbm} is estimated as the following log probability:
\begin{equation*} 
\hspace{-1.5mm} \begin{aligned}
    &\sum_i\log \frac{c(f_j)}{c(\cdot)}+\sum_i\sum_j\log
    \frac{c(e_i,f_j)}{c(f_j)}-|\mathbf{f}|\sum_j\log \frac{c(e_i)}{c(\cdot)}\\
    =&(1-|\mathbf{e}|)\sum_i\log c(f_j)-|\mathbf{f}|\sum_j\log c(e_i)\\
    &+\sum_i\sum_j\log c(e_i,f_j)+|\mathbf{f}|(|\mathbf{e}|-1)\log
    c(\cdot),
\end{aligned}%
\end{equation*}
where $c(x)$ is the count of word $x$, $c(\cdot)$ is the corpus size,
$c(x,y)$ is the joint count of $x$ and $y$, and $|\mathbf{v}|$ is the
dimension of vector $\mathbf{v}$.

Nonetheless, we do investigate how model performance responds to
estimation quality. Specifically in WSD, a \textit{source corpus} is
defined as the source of the majority of the WSD instances in a given
dataset, and a \textit{baseline corpus} of a smaller size and less
resemblance to the instances is used for all datasets. The assumption
is that a source corpus offers better estimates for the model than the
baseline corpus, and difference in model performance is expected when
using probability estimation of different quality.

\section{Evaluation}
\label{sec:evaluation}

\subsection{Data, Scoring, and Pre-processing}
\label{sec:benchmarking}
Various aspects of the model discussed in Section
\ref{sec:softlesk-using-nbm} are evaluated in the English lexical
sample tasks from Senseval-2 \citep{edmonds2001senseval} and
SemEval-2007 \citep{pradhan2007semeval}. Training sections are used as
development data and test sections held out for final testing. Model
performance is evaluated in terms of WSD accuracy using Equation
\eqref{eq:wsd} as the scoring function. Accuracy is defined as the
number of correct responses over the number of instances. Because it
is a rare event for the NBM to produce identical scores,\footnote{This
  has never occurred in the hundreds of thousands of runs in our
  development process.} the model always proposes a unique answer and
accuracy is thus equivalent to F-score commonly used in existing
reports.

Multiword expressions (MWEs) in the Senseval-2 sense inventory are not
explicitly marked in the contexts. Several of the top-ranking systems
implemented their own MWE detection algorithms
\citep{kilgarriff2000framework,litkowski:2002:ACL02-WSD}. Without
digressing to the details of MWE detection --- and meanwhile, to
ensure fair comparison with existing systems --- we implement two
variants of the prediction module, one completely ignorant of MWE and
defaulting to \texttt{INCORRECT} for all MWE-related answers,
% \footnote{Module\#1 essentially avoids making a
%   prediction on MWE-related senses at the cost of lower
%   recall. \textit{F-measure} will be the same regardless of the
%   implementation by ``defaulting'' or ``avoiding''.}
while the other assuming perfect MWE detection and performing regular
disambiguation algorithm on the MWE-related senses (\textit{not}
defaulting to \texttt{CORRECT}). All results reported for Senseval-2
below are harmonic means of the two outcomes.

Each inventory sense is represented by a set of LK tokens (e.g.,
definition texts, synonyms, etc.\@) from their corresponding WordNet
synset (or in the coarse-grained case, a concatenation of tokens from
all synsets in a sense group).
% where multiple senses are grouped by the sense-map (Senseval-2) or
% the sense inventory (SemEval-2007), There is no limitation on the
% data to a particular WordNet version.
The \textit{MIT-JWI} library \citep{mit_jwi} is used for accessing
WordNet. Usage examples in glosses (included by the library by
default) are removed in our experiments.\footnote{We also compared the
  two Lesk baselines (with and without usage examples) on the
  development data but did not observe significant differences as
  reported by \citet{kilgarriff2000framework}.}

% Instances missing WordNet sense annotations in SemEval-2007 are
% excluded from the experiments, as well as Senseval-2 instances
% tagged ``P'' (proper nouns) or ``U'' (unassignable,
% \citealt{edmonds2001senseval}).

% Senses are represented by LK (e.g., definition texts,
% synonyms, etc.) from their corresponding WordNet synsets. WordNet
% versions are specified by each WSD instance (1.7 for all Senseval-2
% data; 14,967, 5,563, and 1,247 instances for 2.1, 2.0, and 1.7, resp.,
% in SemEval-2007). There are also 504 instances in SemEval-2007 missing
% WordNet sense annotations and thus are excluded from the experiments,
% as well as Senseval-2 instances tagged ``P'' (proper nouns) or ``U''
% (unassignable, \citealt{edmonds2001senseval}). Both fine- and
% coarse-grained (through sense-grouping using sense-map provided with
% the data) options are explored in the Senseval-2 data.

Basic pre-processing is performed on the contexts and the glosses,
including lower-casing, stopword removal, lemmatization on both
datasets, and tokenization on the Senseval-2 instances.\footnote{The
  SemEval-2007 instances are already tokenized.} \textit{Stanford
  CoreNLP}\footnote{\url{http://nlp.stanford.edu/software/corenlp.shtml}.}
is used for lemmatization and tokenization. Identical procedures are
applied to all corpora used for probability estimation.

Binomial test is used for significance testing, and with one exception
explicitly noted in Section \ref{sec:coocc-prob-estim}, all
differences presented are statistically highly significant
($p<0.001$).
\subsection{Comparing Lexical Knowledge Sources}
\label{sec:comp-lexic-knowl}
\def\halfTableWidth{.48\textwidth} \def\cellWidth{8mm}
\begin{table*}[t]
  \centering
   \resizebox{.78\textwidth}{!} {
     \begin{tabular}{r||>{\centering\arraybackslash}p{\cellWidth}|>{\centering\arraybackslash}>{\centering\arraybackslash}p{\cellWidth}>{\centering\arraybackslash}p{\cellWidth}>{\centering\arraybackslash}p{\cellWidth}>{\centering\arraybackslash}p{\cellWidth}|>{\centering\arraybackslash}p{\cellWidth}>{\centering\arraybackslash}p{\cellWidth}|>{\centering\arraybackslash}p{\cellWidth}}
    \hline
    \hline
    Dataset &\textit{glo} &\textit{syn} &\textit{hpr}&\textit{hpo}
    & \textit{all} & 1st & 2nd & \textit{Lesk}
    \\
    \hline
    \textit{Senseval-2 Coarse} & .475 & .478 & .494 & .518 & .\textbf{523} & 
    .469 & .367  & .262 \\
    \textit{Senseval-2 Fine} & .362 & .371 & .326 & .379 & .388 &
    .\textbf{412} & .293  & .163  \\
    \textit{SemEval-2007} & .494 & .511 & .507 & .550 & .\textbf{573} & .538
    & .521 & --  \\
    \hline
    \hline
  \end{tabular}}
  \caption{Lexical knowledge sources and WSD performance (\textit{F-measure}) on the Senseval-2 (fine- and coarse-grained) and the SemEval-2007 dataset.}
\label{tab:kickass_numbers}
\end{table*}

To study the effect of different types of LK in WSD (Section
\ref{sec:incorp-addit-lexic}), for each inventory sense, we choose
synonyms (\textit{syn}), hypernyms (\textit{hpr}), and hyponyms
(\textit{hpo}) as extended LK in addition to its gloss. The WSD model
is evaluated with gloss-only (\textit{glo}), individual extended LK
sources, and the combination of all four sources (\textit{all}). The
results are listed in Table \ref{tab:kickass_numbers} together with
existing results (1st and 2nd correspond to the results of the top two
unsupervised methods in each dataset).\footnote{We excluded the
  results of \textit{UNED} \citep{fernandez2001uned} in Senseval-2
  because, by using sense frequency information that is only
  obtainable from sense-annotated corpora, it is essentially a
  supervised system.}

By using only glosses, the proposed model already shows statistically
significant improvement over the basic Lesk algorithm (92.4\% and
140.5\% relative improvement in Senseval-2 coarse- and fine-grained
tracks, respectively).\footnote{Comparisons are made against the
  simplified Lesk algorithm \citep{kilgarriff2000framework} without
  usage examples. The comparison is unavailable in SemEval\-2007 since
  we have not found existing experiments with this exact
  configuration.} Moreover, comparison between coarse- and
fine-grained tracks reveals interesting properties of different LK
sources. Previous hypotheses (Section \ref{sec:incorp-addit-lexic})
are empirically confirmed that WSD performance benefits most from
hyponyms and least from hypernyms. Specifically, highly similar,
fine-grained sense candidates apparently share more hypernyms in the
fine-grained case than in the coarse-grained case; adding to the
generality of hypernyms (both semantic and distributional), we
postulate that their probability in the NBM is uniformly inflated
among many sense candidates, and hence they decrease in
distinguishability. Synonyms might help with regard to semantic
specification, though their limited quantity also limits their
benefits. These patterns on the LK types are consistent in all three
experiments.

% bayesian treatment brought up the effectiveness of the extended LKs,
% as opposed to \citet{patwardhan_using_2003}.

When including all four LK sources, our model outperforms the
state-of-the-art systems with statistical significance in both
coarse-grained tasks. For the fine-grained track, it achieves 2nd
place after that of \citet{tugwell2001wasp}, which used a decision
list \citep{yarowsky1995unsupervised} on \textit{manually selected}
corpora evidence for each inventory sense, and thus is not subject to
loss of distinguishability in the glosses as Lesk variants are.


\subsection{Probability Estimation}
\label{sec:coocc-prob-estim}
\begin{figure}[t]
  \centering
  \includegraphics[width=.95\columnwidth]{prob_est_bars.png}
  \caption{{Model response to probability estimates of different
      quality on the SemEval-2007 dataset. Error bars indicate
      confidence intervals ($p<.001$), and the dashed line corresponds
      to the best reported result.} }
  \label{fig:model_response}
\end{figure}
 
To evaluate model response to probability estimation of different
quality (Section \ref{sec:prob-estim}), source corpora are chosen as
the majority value of the \textit{doc-source} attribute of instances
in each dataset, namely, the \textit{British National Corpus} for
Senseval-2 (94\%) and the \textit{Wall Street Journal} for
SemEval-2007 (86\%). The \textit{Brown Corpus} is shared by both
datasets as the baseline corpus. Figure \ref{fig:model_response} shows
the comparison on the SemEval-2007 dataset. Across all experiments,
higher WSD accuracy is consistently witnessed using the source corpus;
differences are statistically highly significant except for
\textit{hpo} (which is significant with $p<0.01$).

\section{Conclusions and Future Work}
\label{sec:disc-concl}
We have proposed a general-purpose Naive Bayes model for measuring
association between two sets of random events. The model replaced
string matching in the Lesk algorithm for word sense disambiguation
with a probabilistic measure of gloss-context overlap. The base model
on average more than doubled the accuracy of Lesk in Senseval-2 on
both fine- and coarse-grained tracks. With additional lexical
knowledge, the model also outperformed state of the art results with
statistical significance on two coarse-grained WSD tasks.

For future work, we plan to apply the model in other shared tasks,
including open-text WSD, so as to compare with more recent Lesk
variants. We would also like to explore how to incorporate syntactic
features and employ alternative statistical methods (e.g., parametric
models) to improve probability estimation and inference. Other NLP
problems involving compositionality in general might also benefit from
the proposed many-to-many similarity measure.

\section*{Acknowledgments}
This study is funded by the Natural Sciences and Engineering Research
Council of Canada. We thank Afsaneh Fazly, Navdeep Jaitly, and Varada
Kolhatkar for the many inspiring discussions, as well as the anonymous
reviewers for their constructive advice.

\bibliographystyle{natbib/plainnat}
{\small\bibliography{/Users/tong/Documents/Readings/ref}}

\end{document}
