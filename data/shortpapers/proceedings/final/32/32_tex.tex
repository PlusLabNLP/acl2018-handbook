%
% File acl2014.tex
%
% Contact: koller@ling.uni-potsdam.de, yusuke@nii.ac.jp
%%
%% Based on the style files for ACL-2013, which were, in turn,
%% Based on the style files for ACL-2012, which were, in turn,
%% based on the style files for ACL-2011, which were, in turn, 
%% based on the style files for ACL-2010, which were, in turn, 
%% based on the style files for ACL-IJCNLP-2009, which were, in turn,
%% based on the style files for EACL-2009 and IJCNLP-2008...

%% Based on the style files for EACL 2006 by 
%%e.agirre@ehu.es or Sergi.Balari@uab.es
%% and that of ACL 08 by Joakim Nivre and Noah Smith

\documentclass[11pt]{article}
\usepackage{acl2014}
\usepackage{times}
\usepackage{url}
\usepackage{latexsym}

\usepackage{graphicx} %marco added
\usepackage{multirow}

%\setlength\titlebox{5cm}

% You can expand the titlebox if you need extra space
% to show all the authors. Please do not make the titlebox
% smaller than 5cm (the original size); we will check this
% in the camera-ready version and ask you to change it back.


\title{DepecheMood: \\ a Lexicon for Emotion Analysis from Crowd-Annotated News}

%\author{First Author \\
%  Affiliation / Address line 1 \\
%  Affiliation / Address line 2 \\
%  Affiliation / Address line 3 \\
%  {\tt email@domain} \\\And
%  Second Author \\
%  Affiliation / Address line 1 \\
%  Affiliation / Address line 2 \\
%  Affiliation / Address line 3 \\
%  {\tt email@domain} \\}
  
  \author{Jacopo Staiano \\
University of Trento\\
Trento - Italy\\
{\tt staiano@disi.unitn.it} \\\And
Marco Guerini \\
Trento RISE\\
Trento - Italy\\
{\tt marco.guerini@trentorise.eu}
}

\date{}

\begin{document}
\maketitle
\begin{abstract}
While many lexica annotated with words polarity are available for sentiment analysis, very few tackle the harder task of emotion analysis and are
usually quite limited in coverage. In this paper, we present a novel approach for extracting -- in a totally automated way -- a high-coverage and high-precision lexicon of roughly 37 thousand terms annotated  
with emotion scores, called \texttt{DepecheMood}. 
Our approach exploits in an original way `crowd-sourced' affective annotation implicitly provided by readers of news articles from \texttt{rappler.com}. 
By providing new state-of-the-art performances in unsupervised settings for regression and classification tasks, even using a na\"{\i}ve approach, our experiments show the 
beneficial impact of harvesting
social media data for affective lexicon building.% and emotion analysis.

\end{abstract}

\section{Introduction}

Sentiment analysis has proved useful in several application scenarios,
for instance in buzz monitoring -- the marketing technique for keeping track of consumer responses to services and 
products -- where identifying positive and negative customer experiences helps to assess product and service demand, tackle crisis management, etc.%\makeatletter{\renewcommand*\@makefnmark{}
%\footnotetext{This work has been partially supported by the Trento RISE PerTe project.}
%\makeatother}


% data collected from social networks plays a key role for activities such buzz monitoring -- the marketing technique for keeping track of consumer responses to services and 
% products -- where identifying positive and negative customer experiences helps to assess product and service demand, tackle crisis management, etc.

%, -- e.g. businesses that want to find consumer opinions about their products and services -- 
On the other hand, the use of finer-grained models, accounting for the role of individual emotions, is still in its infancy. 
The simple division in 
`positive' vs. `negative' comments may not suffice, as in these examples: `\emph{I'm so miserable, I dropped my IPhone in the water and now it's not working anymore}' 
(\textsc{sadness}) vs. `\emph{I am very upset, my new IPhone keeps not working!}' (\textsc{anger}). While both texts express a negative sentiment, the latter, connected to anger, is more relevant 
for buzz monitoring. 
Thus, emotion analysis represents a natural evolution of sentiment analysis. 

Many approaches to sentiment analysis make use of lexical resources -- i.e. lists of positive and negative words -- often deployed as baselines or as features for other methods, 
usually machine learning based~\cite{liu2012survey}. In these lexica, words are associated with their prior polarity, i.e. whether such word out of context evokes something positive or 
something negative. For example, \emph{wonderful} has a positive connotation -- prior polarity -- while \emph{horrible} has a negative one. %These approaches have the advantage of 
%not needing deep semantic analysis or word sense disambiguation to assign an affective score to a word and are domain independent (they are thus less precise but more portable).} 
%In a similar way other resources contain lists of words associated with emotion labels or emotions scores.  

The quest for a high precision and high coverage lexicon, where words are associated with either sentiment or emotion scores, 
has several reasons.
First,
it is fundamental for tasks such as affective modification of existing texts, where words' polarity together with 
their score are necessary for creating multiple \emph{graded} variations of the original text~\cite{inkpen2006generating,Guerini2008,Whitehead10}. 
%Some works that cast the 
%problem of sentiment strength recognition as a multi-class classification task are presented in~\cite{wilson:AAAI-04,Paltoglou2010}.
%Others, 
%such as~\cite{neviarouskaya2011affect}, use a fine grained classification approach too, but they consider emotion categories (\textsc{anger}, \textsc{joy}, \textsc{fear}, etc.), 
%rather than emotion scores. 


Second, considering word order makes a difference in sentiment analysis. This calls 
for a role of compositionality, where the score of a sentence is computed by composing the scores of the words up in the syntactic tree.  Works worth mentioning in this 
connection are: \newcite{socher2013recursive}, which uses recursive neural networks to learn compositional rules for sentiment analysis, and \cite{neviarouskaya2009compositionality,neviarouskaya2011affect} 
which exploit hand-coded rules to compose the emotions expressed by words in a sentence. 
In this respect, compositional approaches represent a new promising trend, since all other approaches, either using semantic similarity or Bag-of-Words (BOW) based machine-learning, cannot 
handle, for example, cases of texts with same wording but different words order:
``\emph{The dangerous killer escaped one month ago, but recently he was arrested}" (\textsc{relief}, \textsc{happyness}) vs. ``\emph{The dangerous killer was arrested one month ago, but recently he 
escaped}" (\textsc{fear}). The work in ~\cite{wangbaselines} partially accounts for this problem and argues that using word bigram features allows improving over BOW based methods, where words are taken as features in isolation. This way it is possible to capture simple 
compositional 
phenomena like polarity reversing in ``\emph{killing cancer}". 
% \item 

Finally, tasks such as copywriting, where evocative names are a key element to a successful product \cite{ozbalcomputational,ozbal2012brand} require exhaustive lists of emotion 
related words. In such cases no context is given and the brand name alone, with its perceived prior polarity, is responsible for stating the area of competition and evoking 
semantic associations. For example \emph{Mitsubishi} changed the name of one of its SUVs for the Spanish market, since the original name \emph{Pajero} had a very negative prior 
polarity, as it means `wanker' in Spanish \cite{piller200310}. Evoking emotions is also fundamental for a successful name: consider names of a perfume like 
\emph{Obsession}, or technological products like MacBook \emph{air}.


In this work, we aim at automatically producing a high coverage and high precision emotion lexicon using distributional semantics, with numerical scores associated with each emotion, like it has already been done for sentiment 
analysis. To this end, we take advantage in an original way of massive crowd-sourced affective annotations associated with news articles, obtained by crawling the \texttt{rappler.com} social
news network. We also evaluate our lexicon by integrating it in unsupervised classification and 
regression settings for emotion recognition. Results indicate that the use of our resource, even if automatically acquired, is highly beneficial in affective text recognition. 

% The paper is structured as follows: in section 2 we review several sentiment and emotion lexica previously produced, along with various approaches for 
% crowd-sourcing Natural Language Processing (NLP) tasks. In section 3 we present the dataset we collected, followed by the manipulation we performed to extract our emotion lexicon 
% (section 4). Finally, in Section 5 we present some preliminary experiments to evaluate the quality of our lexicon. 

\section{Related Work}
Within the broad field of sentiment analysis, we hereby provide a short 
review of research efforts put towards building sentiment and emotion lexica, regardless of the approach in which 
such lists are then used (machine learning, rule based or deep learning). A general overview can be found in~\cite{pang2008opinion,liu2012survey,wilson:AAAI-04,Paltoglou2010}.

\medskip


\begin{table*} [!htb] 	
	\begin{center} 	 	
		{\footnotesize 		
			\begin{tabular}{l|rrrrrrrrr} 		
				\hline 
 & {\scriptsize AFRAID} & {\scriptsize AMUSED} & {\scriptsize ANGRY} & {\scriptsize ANNOYED} & {\scriptsize DONT\_CARE} & {\scriptsize HAPPY} & {\scriptsize INSPIRED} & 
{\scriptsize SAD} \\
doc\_10002 & 0.75 & 0.00 & 0.00 & 0.00 & 0.00 & 0.00 & 0.25 & 0.00 \\
doc\_10003 & 0.00 & 0.50 & 0.00 & 0.16 & 0.17 & 0.17 & 0.00 & 0.00 \\
doc\_10004 & 0.52 & 0.02 & 0.03 & 0.02 & 0.02 & 0.06 & 0.02 & 0.31 \\
%\ldots & \ldots & \ldots & \ldots & \ldots & \ldots & \ldots & \ldots & \ldots \\
doc\_10011 & 0.40 & 0.00 & 0.00 & 0.20 & 0.00 & 0.20 & 0.20 & 0.00 \\
%doc\_10014 & 0.33 & 0.00 & 0.00 & 0.06 & 0.00 & 0.22 & 0.00 & 0.39 \\
doc\_10028 & 0.00 & 0.30 & 0.08 & 0.00 & 0.00 & 0.23 & 0.31 & 0.08 \\
\hline 		
			\end{tabular} 		
		} 		 	
	\end{center}	 	
	\setlength{\belowcaptionskip}{-0.1cm} 	
	\caption{An excerpt of the Document-by-Emotion Matrix - $M_{DE}$} 	
	\label{tab:Document-by-Emotion-Matrix} 
\end{table*}  
\textbf{Sentiment Lexica}.
In recent years there has been an increasing focus on producing lists of words (lexica) with prior 
polarities, to be used in sentiment analysis. 
When building such lists, a trade-off between coverage of the resource and its precision is to be found.

% One of the most well-known resources is~\emph{SentiWordNet} (SWN)~\cite{Esuli06,baccianella2010sentiwordnet}, in which each entry is a set of \texttt{lemma\#PoS\#sense-number} sharing the same 
%meaning, called \emph{synset}. Each synset \texttt{s} is associated with the numerical scores \texttt{Pos(s)} and \texttt{Neg(s)}, ranging from 0 to 1.  These scores -- 
%automatically assigned starting from a bunch of seed terms -- represent the positive and negative valence (or posterior polarity) of the synset and are inherited by each \texttt{lemma\#PoS\#sense-number} 
%in the synset.

%------ alternativa short
One of the most well-known resources is~\emph{SentiWordNet} (SWN)~\cite{Esuli06,baccianella2010sentiwordnet}, in which each entry is associated with the numerical scores \texttt{Pos(s)} and \texttt{Neg(s)}, ranging from 0 to 1. These scores -- 
automatically assigned starting from a bunch of seed terms -- represent the positive and negative valence (or posterior polarity) of  each entry, that takes the form \texttt{lemma\#pos\#sense-number}.
%------
Starting from SWN, several prior polarities for words (\emph{SWN-prior}), in the form \texttt{lemma\#PoS}, can be computed (e.g. considering only the first-sense, averaging on all the senses, etc.). 
These approaches, detailed in \cite{guerini2013sentiment}, produce a list of 155k words, where the lower precision given by the automatic scoring  of SWN is compensated 
by the high coverage. 

Another widely used resource is \emph{ANEW} \cite{bradley1999affective}, providing
valence scores for 1k words, which were manually assigned by several annotators. This resource has a low coverage, but the precision is maximized. 
Similarly, the \emph{SO-CAL} entries \cite{taboada2011lexicon} were
% collected from a corpus and then 
manually tagged by a small number
of annotators with a multi-class label (from \texttt{very\_negative} to \texttt{very\_positive}). These ratings were further
validated through crowd-sourcing, ending up with a list of roughly 4k words. 
More recently, a resource that replicated ANEW annotation approach using crowd-sourcing, was released~\cite{warriner2013norms}, providing sentiment scores for 14k words.
Interestingly, this resource annotates the most frequent words in English, so, even if lexicon coverage is still far lower than SWN-prior,  it grants a high coverage, with human precision, of language use. 

Finally,  the \emph{General
Inquirer} lexicon \cite{stone1966general} provides a binary
classification  (\texttt{positive}/\texttt{negative}) of 4k
sentiment-bearing words, while the resource in
\cite{wilson2005recognizing} expands the General
Inquirer to 6k words. 

\textbf{Emotion Lexica}.
Compared to sentiment lexica, far less emotion lexica have been produced, and all have lower coverage. 
One of the most used resources is \emph{WordNetAffect}~\cite{strappaLREC04} which contains manually assigned affective labels to WordNet synsets 
(\textsc{anger}, \textsc{joy}, \textsc{fear}, etc.). It currently provides 900 annotated synsets and 1.6k words in the form \texttt{lemma\#PoS\#sense}, corresponding to roughly 1 thousand 
\texttt{lemma\#PoS}. 
%\textbf{We organized direct affective words and synset in WordNet-Affect. Then, we developed a selection function (named Affective-Weight) based on a semantic similarity mechanism automatically acquired in an unsupervised way from a large corpus of texts (100 millions of words), in order to individuate the indirect affective lexicon. Applied to a concept (e.g. a WordNet synset) and an emotional category, this function returns a value representing the semantic affinity with that emotion. In this way it is possible to assign a value to the concept with respect to each emotional category, and eventually select the emotion with the highest value. Applied to a set of concepts that are semantically similar, this function selects subsets characterized by some given affective constraints (e.g. referring to a particular emotional category or valence).}

\emph{AffectNet}, part of the SenticNet project \cite{cambria2012sentic}, contains 10k words (out of 23k entries) taken from ConceptNet and aligned with WordNetAffect. This 
resource extends WordNetAffect labels to concepts like `have breakfast'.
\emph{Fuzzy Affect Lexicon} \cite{subasic2001affect} contains roughly 4k \texttt{lemma\#PoS} manually annotated by one linguist using 80 emotion labels.
\emph{EmoLex} \cite{mohammad2013crowdsourcing} contains almost 10k lemmas annotated with an intensity label for each emotion using Mechanical Turk.
Finally~\emph{Affect database} is an extension of SentiFul \cite{Neviarouskaya:2007fk} and contains 2.5K words in the form \texttt{lemma\#PoS}. The latter is the only lexicon 
providing words annotated also with emotion scores rather than only with labels.

% \subsection{Crowd-sourcing and NLP}
% In recent years, there has been a growing interest in finding new 'cheap and fast' methodologies to be used in experimental research, for, but not 
% limited to, NLP tasks.  Most of these approaches rely on "crowd-sourcing" i.e. making multiple people performing a task of interest, usually on the Web, willingly or not. 
% % Such subjects may or may 
% % be not aware of taking part to the task, may or may not be paid, etc..
% 
% In particular, approaches to NLP that rely on the use of specific web tools - for crowdsourcing long and tedious tasks - have emerged. Amazon's Mechanical Turk (AMT), for example, 
% has been used for collecting annotated data \cite{snow2008cheap}, and is now a reference tool in the field.  
% % Another example is re-CAPTCHA a free CAPTCHA service that helps to 
% % digitize books, newspapers and old time radio shows. In this case subjects are not paid for their annotation effort, but are forced to complete the task if they want to access the 
% % services behind the re-CAPTCHA page. 
% AMT has been widely used also in the context of behavioral studies~\cite{mason2010conducting}, even if it poses problem of subjects 
% reliability -- on the aspect of how to handle subject reliability in coding tasks, see the method proposed in \cite{negri2010divide}. 
% 
% Finally, other approaches, where subjects are unaware of being 
% investigated,  collect data from websites during the surfing activity of the users: 
% \cite{aral2010creating} presented an approach to assess the effects of content features modification 
% on a Social Media site, while~\cite{guerini2012ecological} exploited Google AdWords to evaluate the effectiveness of 
% automatic sentiment variations in textual advertisements.
% \medskip
\section{Dataset Collection}
\label{DS}
To build our emotion lexicon we harvested all the news articles from \texttt{rappler.com}, as of June 3rd 2013: the final dataset consists of  
13.5 M words over 25.3 K documents, with an average of 530 words per document. For each document, along with the text we also harvested the information displayed by Rappler's
\emph{Mood Meter}, a small interface 
% (see Figure~\ref{mood-meter}), 
offering the readers the opportunity to click on the emotion that a given Rappler story made them feel. The 
idea behind the Mood Meter is actually ``getting people to \emph{crowdsource} the mood for the 
day"\footnote{http://nie.mn/QuD17Z}, and returning the percentage of votes for each 
emotion label for a given story. This way, hundreds of thousands votes have been collected since the launch of the service.
 In our novel approach to `crowdsourcing', as compared to other NLP tasks that rely on tools like Amazon's Mechanical Turk \cite{snow2008cheap}, the subjects are aware of the `implicit annotation task' but 
%In this case, as compared to other crowdsourcing approaches  to NLP that rely on the use of specific web tools like Amazon's Mechanical Turk \cite{snow2008cheap}, the subjects are aware of the `annotation task' but 
they are not paid. %and are freely deciding to participate to it. 
From this data, we built a document-by-emotion matrix $M_{DE}$, providing the voting percentages for each document 
in the eight affective dimensions available in Rappler. An excerpt is provided in Table \ref{tab:Document-by-Emotion-Matrix}. 


 
% \begin{figure}[ht!]
% \centering
% \includegraphics[width=50mm]{RapplerWorkingTheCrowd_mod.png}
% \caption{A simple caption}
% \label{mood-meter}
% \end{figure}

The idea of using documents annotated with emotions is not new~\cite{strapparava2008learning,mishne2005experiments,bellegarda2010emotion}, but these works had the limitation of 
providing a single emotion label per document, rather than a score for each emotion, and, moreover, the annotation was performed by the author of the document alone.
% in those cases there was only one label per 
% document with no score and it was provided by one person (the author).


Table \ref{tab:percentage-votes} reports the average percentage of votes for each emotion on the whole corpus: \textsc{happiness} has a far higher percentage of votes (at least three times). There 
are several possible explanations, out of the scope of the present paper, for this bias: (i) it is due to cultural characteristics of the audience 
%(Rappler is a Philippine based social news network);
(ii)
the bias is in the dataset itself, being formed mainly by `positive' news; (iii) it is a psychological phenomenon due to the fact that people tend to express more positive moods on 
social networks~\cite{querciamood,vittengl1998time,de2012not}. In any case, the predominance of happy mood has been found in other datasets, for instance \texttt{LiveJournal.com} posts \cite{strapparava2008learning}.
In the following section we will discuss how we handled this problem. 

  
\begin{table} [!htb] 	
	\begin{center} 	 	
		{\footnotesize 		
			\begin{tabular}{lr|lrrr} 		
				\hline 
 EMOTION & Votes$_{\mu}$ & EMOTION & Votes$_{\mu}$\\
 \hline 
AFRAID & 0.04 & DONT\_CARE & 0.05 \\
AMUSED & 0.10 & HAPPY & 0.32 \\
ANGRY & 0.10 & INSPIRED & 0.10 \\
ANNOYED & 0.06 & SAD & 0.11 \\
\hline 
			\end{tabular} 		
		} 		 	
	\end{center}	 	
	\setlength{\belowcaptionskip}{-0.1cm} 	
	\caption{Average percentages of votes.} 	
	\label{tab:percentage-votes} 
\end{table}  


\begin{table*} [!htb] 	
	\begin{center} 	 	
		{\footnotesize 		
			\begin{tabular}{l|rrrrrrrrr} 		
				\hline 
Word & AFRAID & AMUSED & ANGRY & ANNOYED & DONT\_CARE & HAPPY & INSPIRED & SAD \\
\hline
%amused\#a & 0.03 & 0.34 & 0.14 & 0.21 & 0.06 & 0.07 & 0.12 & 0.03 \\
awe\#n & 0.08 & 0.12 & 0.04 & 0.11 & 0.07 & 0.15 & \textbf{0.38} & 0.05 \\
%comedian\#n & 0.05 & 0.14 & 0.13 & 0.14 & 0.17 & 0.12 & 0.11 & 0.14 \\
%comedy\#n & 0.05 & 0.21 & 0.04 & 0.10 & 0.16 & 0.19 & 0.12 & 0.14 \\
%comic\#a & 0.04 & 0.23 & 0.03 & 0.08 & 0.21 & 0.20 & 0.16 & 0.05 \\
%comic\#n & 0.04 & 0.17 & 0.04 & 0.10 & 0.28 & 0.18 & 0.12 & 0.06 \\
comical\#a & 0.02 & \textbf{0.51} & 0.04 & 0.05 & 0.12 & 0.17 & 0.03 & 0.06 \\
%crash\#n & 0.08 & 0.10 & 0.12 & 0.10 & 0.09 & 0.07 & 0.08 & 0.36 \\
%create\#v & 0.13 & 0.14 & 0.09 & 0.12 & 0.12 & 0.13 & 0.17 & 0.09 \\
%creation\#n & 0.11 & 0.14 & 0.10 & 0.11 & 0.12 & 0.15 & 0.18 & 0.09 \\
crime\#n & 0.11 & 0.10 & \textbf{0.23} & 0.15 & 0.07 & 0.09 & 0.09 & 0.15 \\
criminal\#a & 0.12 & 0.10 & \textbf{0.25} & 0.14 & 0.10 & 0.11 & 0.07 & 0.11 \\
dead\#a & 0.17 & 0.07 & 0.17 & 0.07 & 0.07 & 0.05 & 0.05 & \textbf{0.35} \\
%death\#n & 0.16 & 0.07 & 0.16 & 0.09 & 0.09 & 0.07 & 0.09 & \textbf{0.27} \\
funny\#a & 0.04 & \textbf{0.29} & 0.04 & 0.11 & 0.16 & 0.13 & 0.15 & 0.08 \\
future\#n & 0.09 & 0.12 & 0.09 & 0.12 & 0.13 & 0.13 & \textbf{0.21} & 0.10 \\
game\#n & 0.06 & 0.15 & 0.06 & 0.08 & 0.15 & \textbf{0.23} & 0.15 & 0.12 \\
%bad\#a & 0.13 & 0.13 & 0.10 & 0.11 & 0.11 & 0.11 & 0.13 & 0.17 \\
%good\#a & 0.08 & 0.15 & 0.07 & 0.12 & 0.14 & 0.15 & 0.19 & 0.10 \\
%homicide\#n & 0.14 & 0.09 & \textbf{0.28} & 0.09 & 0.06 & 0.06 & 0.06 & \textbf{0.22} \\
kill\#v & \textbf{0.23} & 0.06 & \textbf{0.21} & 0.07 & 0.05 & 0.06 & 0.05 & \textbf{0.27} \\
%massacre\#n & 0.14 & 0.07 & \textbf{0.26} & 0.10 & 0.06 & 0.09 & 0.05 & \textbf{0.23} \\
%positive\#a & 0.13 & 0.12 & 0.07 & 0.08 & 0.11 & 0.16 & 0.21 & 0.11 \\
%rape\#n & 0.12 & 0.10 & \textbf{0.35} & 0.11 & 0.05 & 0.05 & 0.09 & 0.12 \\
rapist\#n & 0.02 & 0.07 & \textbf{0.46} & 0.07 & 0.08 & 0.16 & 0.03 & 0.12 \\
sad\#a & 0.06 & 0.12 & 0.09 & 0.14 & 0.13 & 0.07 & 0.15 & \textbf{0.24} \\
warning\#n & \textbf{0.44} & 0.06 & 0.09 & 0.09 & 0.06 & 0.06 & 0.04 & 0.16 \\
%happy\#a & 0.06 & 0.15 & 0.06 & 0.12 & 0.15 & 0.16 & 0.21 & 0.10 \\
\hline
\end{tabular} 		
		} 	
				 	
	\end{center}	 	
	\setlength{\belowcaptionskip}{-0.1cm} 	
	\caption{An excerpt of the Word-by-Emotion Matrix ($M_{WE}$) using normalized frequencies ($nf$). Emotions weighting more than 20\% in a word are highlighted for readability purposes.} 	
	\label{tab:word-emotion} 
\end{table*}  


%\section{Word-by-Emotion Matrices}
\section{Emotion Lexicon Creation}

As a next step we built a word-by-emotion matrix starting from $M_{DE}$ using an approach based on compositional semantics. 
To do so, we first lemmatized and PoS tagged all the documents (where PoS can be adj., nouns, verbs, adv.) and kept only those \texttt{lemma\#PoS} present also in WordNet, similar to SWN-prior and WordNetAffect resources, to which we want to align.
We then computed the term-by-document matrices using raw frequencies, normalized 
frequencies, and tf-idf ($M_{WD,f}$, $M_{WD,nf}$ and $M_{WD,tfidf}$ respectively), so to test which of the three weights is better. 
After that, we applied matrix multiplication between the document-by-emotion and word-by-document matrices ($M_{DE} \cdot M_{WD}$) to obtain a (raw) word-by-emotion matrix $M_{WE}$. This method allows us to `merge' words with emotions by summing the products of the weight of a word with the weight of the emotions in each document. 

Finally, we transformed $M_{WE}$ 
% in a \emph{probability} matrix 
by first applying 
%mean normalization on the columns to each sentiment score 
normalization column-wise
(so to eliminate the over representation for happiness as discussed in Section \ref{DS}) 
and then scaling the data row-wise so to sum up to one. 
An excerpt of the final Matrix $M_{WE}$ is presented in Table \ref{tab:word-emotion}, and it can be interpreted as a list of 
words with scores that represent how much 
weight a given word has in the affective dimensions we consider. So, for example, \texttt{awe\#n} has a predominant weight in \textsc{inspired} (0.38), \texttt{comical\#a} has a predominant weight in \textsc{amused} (0.51), while \texttt{kill\#v} has a predominant weight in \textsc{afraid}, \textsc{angry} and \textsc{sad} (0.23, 0.21 and 0.27 respectively). This matrix, that we call \texttt{DepecheMood}\footnote{%Reminiscent of the \emph{Depeche Mode} electronic band, 
In French, `depeche' means dispatch/news.}, represents our emotion lexicon, it contains 37k entries and is freely available for research purposes at http://git.io/MqyoIg.  

%in the form \texttt{lemma\#PoS} (where PoS can be adj., nouns, verbs, adv.), and is mapped on the corresponding WordNet entries, as done for SWN-prion and WN-affect resources.
%each emotion contribute to the word.



\section{Experiments}

%\emph{rappler in realt� restituisce una percentuale di
%agreement fra gli annotatori (eg. 90\% degli annotatori provano paura e
%10\% provano tristezza) mentre semeval restituisce una intensit� da 0 a
%100 della singola emozione come percepita dagli annotatori, non una
%agreement tra di loro se quella emozione (mood) �  presente}

To evaluate the performance we can obtain with our lexicon, we use the public dataset provided for the SemEval 2007 task on `Affective Text'~\cite{strapparava2007semeval}. The task was focused on emotion recognition in one thousand news headlines, 
both in regression and classification settings. Headlines typically consist of a few words and are often written with the 
intention to `provoke' emotions so to attract the readers' attention. An example of headline from the dataset is the following: ``\emph{Iraq car 
bombings kill 22 People, wound more than 60}". For the regression task the values provided are: \texttt{{$<$anger (0.32), disgust (0.27), fear (0.84), 
joy (0.0), sadness (0.95), surprise (0.20)$>$}} 
while for the classification task the labels provided are \texttt{\{FEAR, SADNESS\}}.

This  dataset is  of interest to us since the `compositional' problem is less prominent given the simplified 
syntax of news headlines, containing, for example, fewer adverbs (like negations or intensifiers) than normal sentences~\cite{turchi2012onts}. %linking verbs, prepositions and adverbs
Furthermore, this is to our 
knowledge the only dataset available providing numerical scores for emotions. Finally, this dataset was meant for unsupervised approaches (just a small trial sample was 
provided), so to avoid simple text categorization approaches. %The test set provided for the task, and used in our experiments, contains 1000 headlines.

As the affective dimensions present in the test set -- based on the six basic emotions model \cite{Ekman1971} -- do not exactly
 match with the ones provided by Rappler's Mood Meter, we first define a mapping between the two when possible, see
 Table~\ref{tab:mapping}. Then, we proceed to transform the test headlines to the \texttt{lemma\#PoS} format.
 
 \begin{table} [h] 	
	\begin{center} 	 	
		{\footnotesize 		
			\begin{tabular}{ll|ll} 						
			\hline 
SemEval  & Rappler  & SemEval  & Rappler \\
			\hline 
FEAR & AFRAID & \textbf{SURPRISE} & \textbf{INSPIRED} \\
%ANGER & ANGRY & DISGUST& \textbf{ANNOYED}\\
ANGER & ANGRY & - & ANNOYED\\
JOY & HAPPY & - & AMUSED\\
SADNESS & SAD & - & DON'T CARE\\
\hline


			\hline 		
			\end{tabular} 		
		} 		 	
	\end{center}	 	
	\setlength{\belowcaptionskip}{-0.1cm} 	
	\caption{Mapping of Rappler labels on Semeval2007. In bold, cases of suboptimal mapping.}
	\label{tab:mapping} 
\end{table}  

Only one test headline contained exclusively words not present in \texttt{DepecheMood}, further indicating the high-coverage nature of our resource. In Table 
\ref{tab:coverage} we report the coverage of some Sentiment and Emotion Lexica of different sizes on the same dataset. %(the average length of headlines is 7.37 words). 
Similar to Warriner et al. (2013), we 
observe that even if the number of entries of our lexicon is far lower than SWN-prior approaches, the fact that we extracted and annotated words from documents grants a high 
coverage of language use. 

\begin{table} [h] 	
	\begin{center} 	 	
		{\footnotesize 		
			\begin{tabular}{l|lrr} 						
			\hline 

%\texttt{lemma\#PoS} per headline & 5.41 \\
%$M_{WE}$ \texttt{lemma\#PoS} per headline & 4.78 \\ 
\hline			 
\multirow{3}{*}{\parbox[t]{1.6cm}{Sentiment Lexica}}& ANEW &1k entries & 0.10 \\
&Warriner et. al & 13k entries & 0.51 \\
&SWN-prior & 155k entries & \textbf{0.67} \\				  
\hline
\multirow{2}{*}{\parbox[t]{1.5cm}{Emotion Lexica}}&WNAffect &1k entries & 0.12 \\
&DepecheMood &37k entries & \textbf{0.64} \\
 \hline
			\end{tabular} 		
		} 		 	
	\end{center}	 	
	\setlength{\belowcaptionskip}{-0.1cm} 	
	\caption{Statistics on words coverage per headline.} 	
	\label{tab:coverage} 
\end{table} 

Since our primary goal is to assess the quality of  \texttt{DepecheMood} %the resource we produced, $M_{WE}$, 
we first focus on the regression task. 
We do so by using a very na\"{\i}ve approach, similar to ``WordNetAffect presence" discussed in \cite{strapparava2008learning}: for each  headline,
we simply compute a value, for any affective dimension, by averaging the corresponding affective scores --obtained from \texttt{DepecheMood}\-- of all \texttt{lemma\#PoS} present in the headline. 



% for 
% each emotion and , we computed the average score for the test headline \texttt{lemma\#PoS}  that were present in $M_{WE}$.  
% we simply take the average score , 

In Table \ref{tab:SemEval2007} we report the results obtained using the three versions of our resource (Pearson correlation), along with the best performance on each emotion of 
other systems\footnote{Systems participating in the `Affective Text' task plus the approaches in ~\cite{strapparava2008learning}. Other supervised approaches in the classification task \cite{mohammad2012emotional,bellegarda2010emotion,chaffar2011using}, reporting only overall performances, are not considered.} ($best_{se}$); the last column contains the upper bound of inter-annotator agreement.
%For 5 emotions out of 6 we improve over the best performing 
%systems. We do not outperform other systems only on \texttt{DISGUST}, that has no clear alignment with our labels (see Table \ref{tab:mapping}).
For all the 5 emotions we improve over the best performing 
systems (\texttt{DISGUST} has no alignment with our labels and was discarded).

Interestingly, even using a sub-optimal alignment for \texttt{SURPRISE} we still manage to outperform other systems. Considering the na\"{\i}ve approach we used, we can 
reasonably conclude that the quality and coverage of our resource are the reason of such results, and that adopting more complex approaches (i.e. compositionality) can possibly further improve 
performances in text-based emotion recognition.

\begin{table} [htb!] 	
	\begin{center} 	 	
		{\footnotesize 		
			\begin{tabular}{l|rrr|r|r} 		
				\hline 
& \multicolumn{3}{c|}{$DepecheMood$} & $best_{se}$ & upper\\
%\hline
 & \emph{f} & \emph{nf} & \emph{tfidf} &  &  \\
FEAR & \textbf{0.56} & 0.54 & 0.53 & 0.45 & 0.64 \\
ANGER & 0.36 & \textbf{0.38} & 0.36 & 0.32 & 0.50 \\
SURPRISE* & \textbf{0.25} & 0.21 & 0.24 & 0.16 & 0.36 \\
%DISGUST* & 0.05 & 0.06 & 0.07 & \textbf{0.18} & 0.44 \\
JOY & 0.39 & \textbf{0.40} & 0.39 & 0.26 & 0.60 \\
SADNESS & \textbf{0.48} & 0.47 & 0.46 & 0.41 & 0.68 \\

\hline 		
			\end{tabular} 		
		} 		 	
	\end{center}	 	
	\setlength{\belowcaptionskip}{-0.1cm} 	
	\caption{Regression results -- Pearson's correlation} 	%on SemEval2007 dataset.
	\label{tab:SemEval2007}  
\end{table}  

As a final test, we evaluate our resource in the classification task. 
The  na\"{\i}ve approach used in this case consists in mapping the average of the scores of 
all words in the headline to a binary decision with fixed threshold at 0.5 for each emotion (after min-max normalization on all test headlines scores). 
In Table~\ref{tab:SemEval2007_class} we report the results (F1 measure) of our 
approach along with the best performance of other systems on each emotion ($best_{se}$), as in the previous case. 
For 3 emotions out of 5 we improve over the best performing systems, for one emotion we obtain the same results, and for one emotion we do not outperform other systems. In this case the difference 
in performances among the various ways of representing the word-by-document matrix is more prominent: normalized frequencies  ($nf$) provide the best results. 

\begin{table} [htb!] 	
	\begin{center} 	 	
		{\footnotesize 		
			\begin{tabular}{l|rrr|r} 		
				\hline 
& \multicolumn{3}{c|}{$DepecheMood$} & $best_{se}$ \\
 & \emph{f} & \emph{nf} & \emph{tfidf} & \\
FEAR & 0.25 & \textbf{0.32} & 0.31 & 0.23 \\
ANGER & 0.00 & 0.00 & 0.00 & \textbf{0.17} \\
SURPRISE* & 0.13 & \textbf{0.16} & 0.09 & 0.15 \\
%DISGUST* & 0.03 & 0.02 & 0.01 & \textbf{0.04} \\
JOY & 0.22 & 0.30 & \textbf{0.32} & \textbf{0.32} \\
SADNESS & 0.36 & \textbf{0.40} & 0.38 & 0.30 \\
\hline 		
			\end{tabular} 		
		} 		 	
	\end{center}	 	
	\setlength{\belowcaptionskip}{-0.1cm} 	
	\caption{Classification results -- F1 measures} 	
	\label{tab:SemEval2007_class} 
\end{table} 

\section{Conclusions}
We presented %and provided to the community 
\texttt{DepecheMood}, an emotion lexicon built in a novel and totally automated way by harvesting crowd-sourced affective annotation from a social news network. Our experimental results indicate 
high-coverage and high-precision of the lexicon, showing significant improvements over state-of-the-art unsupervised approaches even when using the resource with
very na\"{\i}ve classification and regression strategies.
We believe that the wealth of information provided by social media can be harnessed to build models and resources for emotion recognition from text, going  a step beyond 
sentiment analysis.
Our future work will include testing Singular Value Decomposition on the word-by-document matrices, allowing to propagate emotions values for a document to similar 
words non present in the document itself, and the study of perceived mood effects on virality indices and readers engagement by exploiting tweets, likes, reshares and comments.\makeatletter{\renewcommand*\@makefnmark{}
\footnotetext{This work has been partially supported by the Trento RISE PerTe project.}
\makeatother} 

%\section*{Acknowledgments}

%The acknowledgments should go immediately before the references.  Do not number the acknowledgments section. Do not include this section when submitting your paper for review.

% include your own bib file like this:
\bibliographystyle{acl}
\bibliography{Persuasive}

\end{document}