\begin{tutorial}
  {Neural Approaches to Conversational AI}
  {tutorial-026}
  {\daydateyear, \tutorialmorningtime}
  {\TutLocB}
\end{tutorial}

This tutorial surveys neural approaches to conversational AI that were developed in the last few years. We group conversational systems into three categories: (1) question answering agents, (2) task-oriented dialogue agents, and (3) social bots. For each category,  we present a review of state-of-the-art neural approaches, draw the connection between neural approaches and traditional symbolic approaches, and discuss the progress we have made and challenges we are facing, using specific systems and models as case studies.


\vspace{2ex}\centerline{\rule{.5\linewidth}{.5pt}}\vspace{2ex}
\setlength{\parskip}{1ex}\setlength{\parindent}{0ex}




{\bfseries Jianfeng Gao} is Partner Research Manager at Microsoft
AI and Research, Redmond. He leads the
development of AI systems for machine reading
comprehension, question answering, chitchat bots,
task-oriented dialogue, and business applications.
From 2014 to 2017, he was Partner Research Manager
and Principal Researcher at Deep Learning
Technology Center at Microsoft Research, Redmond,
where he was leading the research on deep
learning for text and image processing. From
2006 to 2014, he was Researcher, Senior Researcher,
and Principal Researcher at Natural Language
Processing Group at Microsoft Research,
Redmond, where he worked on the Bing search
engine, improving its core relevance engine and
query spelling, understanding and reformulation
engines, MS ads relevance and prediction, and statistical
machine translation. From 2005 to 2006,
he was a Research Lead in Natural Interactive Services
Division at Microsoft, where he worked on
Project X, an effort of developing natural user interface
for Windows. From 2000 to 2005, he was
Research Lead in Natural Language Computing
Group at Microsoft Research Asia, where he and
his colleagues developed the first Chinese speech
recognition system released with Microsoft Office,
the Chinese/Japanese Input Method Editors (IME)
which were the leading products in the market, and
the natural language platform for Microsoft Windows.

{\bfseries Michel Galley} is a Senior Researcher at Microsoft
Research. His research interests are in
the areas of natural language processing and machine
learning, with a particular focus on conversational
AI, text generation, and machine translation.
From 2007 to 2010, he was a Postdoctoral
Scholar then Research Associate in the Computer
Science department at Stanford University, working
primarily on Machine Translation. In 2007,
he obtained his Ph.D. from the Computer Science
department at Columbia University, with research
focusing on summarization, discourse, and dialogue.
From 2003 to 2005, he did several internships
at USC/ISI and Language Weaver on machine
translation, which included work that won
several NIST MT competitions. From 2000-2001,
he did an 8-month internship and undergraduate
thesis work in the Spoken Dialog Systems group
at Bell Labs, working on generation for dialogue
systems.

{\bfseries Lihong Li} is a Research Scientist at Google
Inc. He obtained a PhD degree in Computer Science
from Rutgers University, specializing in reinforcement
learning theory and algorithms. After
that, he has held Researcher, Senior Researcher,
and Principal Researcher positions in Yahoo!
Research (2009-2012) and Microsoft Research
(2012-2017), before joining Google. His
main research interests are reinforcement learning
(in both Markov decision processes and contextual
bandits) and other related problems in AI (including
active leaning, online learning and large-scale
machine learning). His work has found applications
in recommendation, advertising, Web search
and conversation systems, and has won best paper
awards at ICML, AISTATS and WSDM. In recent
years, he served as area chairs or senior program
committee members at AAAI, AISTATS, ICML,
IJCAI and NIPS. More information can be found
on his homepage: \url{http://lihongli.github.io}.




