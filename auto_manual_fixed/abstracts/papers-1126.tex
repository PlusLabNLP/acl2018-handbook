Linguistic alignment between dialogue partners has been claimed to be affected by their relative social power. A common finding has been that interlocutors of higher power tend to receive more alignment than those of lower power. However, these studies overlook some low-level linguistic features that can also affect alignment, which casts doubts on these findings. This work characterizes the effect of power on alignment with logistic regression models in two datasets, finding that the effect vanishes or is reversed after controlling for low-level features such as utterance length. Thus, linguistic alignment is explained better by low-level features than by social power. We argue that a wider range of factors, especially cognitive factors, need to be taken into account for future studies on observational data when social factors of language use are in question.
