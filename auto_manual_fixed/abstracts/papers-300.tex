In neural machine translation, a source sequence of words is encoded into a vector from which a target sequence is generated in the decoding phase. Differently from statistical machine translation, the associations between source words and their possible target counterparts are not explicitly stored. Source and target words are at the two ends of a long information processing procedure, mediated by hidden states at both the source encoding and the target decoding phases. This makes it possible that a source word is incorrectly translated into a target word that is not any of its admissible equivalent counterparts in the target language. In this paper, we seek to somewhat shorten the distance between source and target words in that procedure, and thus strengthen their association, by means of a method we term bridging source and target word embeddings. We experiment with three strategies: (1) a source-side bridging model, where source word embeddings are moved one step closer to the output target sequence; (2) a target-side bridging model, which explores the more relevant source word embeddings for the prediction of the target sequence; and (3) a direct bridging model, which directly connects source and target word embeddings seeking to minimize errors in the translation of ones by the others. Experiments and analysis presented in this paper demonstrate that the proposed bridging models are able to significantly improve quality of both sentence translation, in general, and alignment and translation of individual source words with target words, in particular.
